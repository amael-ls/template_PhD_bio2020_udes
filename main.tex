
\documentclass[titlepage,oneside,letterpaper,openright,12pt]{report}

%%%%%%%%%%%%%%%%%%%    COMMENTS    %%%%%%%%%%%%%%%%%%%
% 1. To stop numbering the lines, deactivate \linenumbers (activated after loading the package lineno)
% 
% 2. To make the index, you need to compile first with lualatex and then use makeindex. Example:
%		lualatex main.tex
%		makeindex main.nlo -s nomencl.ist -o main.nls
%		lualatex main.tex
% 
% 3. For the bibliography, use the command biber (as I am using the package biblatex, much more modern than bibtex).
%		To use multiple bibliography, use the refsection environment (3.13.3 Multiple Bibliographies in biblatex doc Version 3.13).
%		Do not use the chapterbib package with biblatex, there is refsection environment for that!
%		Each article will need to use
%		\begin{refsection}
%			Intro, Methods, Results, Discussion
%			\printbibliography[heading=subbibnumbered]
%		\end{refsection}
% 
% 4. I defined two languages in the document, French and British english (which is the main language), use \begin{french} ... \end{french} to switch to French in a paragraph
% 
% 5. To avoid multi-defined labels, I recommend to apply this command line in your shell for each tex file of the folders ./article1/, ./article2/, ...:
% 		sed -i.tmp 's/\\label{/\\label{WHATEVER::/g' *.tex
% 			This will change all the labels by adding WHATEVER before. Example (adding article1 for each tex file in ./article1/ folder):
% 			\label{sec::intro} in ./article1/chap1.tex becomes \label{article1::sec::intro}
%		sed -i.tmp 's/\\ref{/\\ref{WHATEVER::/g' *.tex
%			This will change all the refs by adding WHATEVER before. Example (adding article1 for each tex file in ./article1/ folder):
% 			\ref{sec::intro} in ./article1/chap1.tex becomes \ref{article1::sec::intro}
% 		sed -i.tmp 's/\\eqref{/\\eqref{WHATEVER::/g' *.tex
%			This will change all the eqrefs by adding WHATEVER before. Example (adding article1 for each tex file in ./article1/ folder):
% 			\eqref{eq::R0_equation} in ./article1/chap1.tex becomes \eqref{article1::eq::R0_equation}
% 
%	For safety reasons, the original file will be kept in a temporary file '.tmp'. Example:
%		sed -i.tmp 'Your instructions here' introduction.tex will create two files:
% 			introduction.tex which should be the one to keep (if no mistakes in the sed instructions)
% 			introduction.tex.tmp which is the original file (that you can dump if sed behaved the way you expected)
% 	For an introduction to the power of sed and regular expressions: https://www.grymoire.com/Unix/Sed.html#uh-1
% 
% 6. I defined some colours in this document. Except RdeepBlue, none of them is used in this template and can be deleted.
% 	If you delete RdeepBlue, do not forget to adapt all the commands using this colour

%%%%%%%%%%%%%%%%%%%    PACKAGES    %%%%%%%%%%%%%%%%%%%
%%%% font
\usepackage{fontspec}
\usepackage{marvosym}

%%%% Language
\usepackage{polyglossia}
	\setdefaultlanguage[variant=british]{english}
	\setotherlanguages{french}

%%%% Date
\usepackage{datetime}
	\newdateformat{monthyeardate}{\monthname[\THEMONTH] \THEYEAR}

%%%% marges, space, titles...
\usepackage[top=3.75cm, bottom=2.5cm, left=3cm, right=2.5cm]{geometry}
% \usepackage{setspace}
\setlength{\parindent}{0ex}   % indentation au début de chaque paragraphe
\setlength{\parskip}{3ex plus 0.5ex minus 0.2ex} % espace vertical entre paragraphes

\usepackage{titlesec}
\newcommand\chapterstring{CHAPITRE}
\titleformat{\chapter}[display]{\vspace{-8em}\center\bfseries}{\chapterstring~\thechapter}{12pt}{}[\vspace{1.5ex}]
\titleformat{\section}{\vspace{0mm plus 2cm}\addpenalty{-1000}\bfseries\sffamily}{\thesection}{1em}{}
\titleformat{\subsection}{\addpenalty{-500}\bfseries\sffamily}{\thesubsection}{1em}{}
\titleformat{\subsubsection}{\penalty-500\vspace{0pt plus 2pt}{}\bfseries\sffamily}{\thesubsubsection}{}{}[\vspace{-14pt}]

\usepackage{titletoc}
	\titlecontents{chapter}[0pt]{}{\bfseries\chapterstring~\thecontentslabel~}{\bfseries}{\hspace{1em plus 1fill}\contentspage}

%%%% Appendix
\usepackage{appendix}
%% Start of subappendices environment
\AtBeginEnvironment{subappendices}{%
	\chapter*{Appendix}
	\addcontentsline{toc}{chapter}{Appendices}
	\counterwithin{figure}{section}
	\counterwithin{table}{section}
}

%% End of subappendices environment
\AtEndEnvironment{subappendices}{%
	\counterwithout{figure}{section}
	\counterwithout{table}{section}
}

%%%% Graphics
%% To compile with lualatex
\usepackage[luatex]{graphicx}

%% Graphics' path, but I prefer to use absolute path for safety
% \graphicspath{{./article1/}{./article2/}{article3/}}

%% Numbering figures in sections rather than continuously
\usepackage{chngcntr}
\counterwithin{figure}{section}

%% Put figures within their declaration section 
\usepackage[section]{placeins}

%% To create multi-figures
\usepackage{caption}
\usepackage{subcaption}

%% Colours
\usepackage[dvipsnames, svgnames]{xcolor}
	\definecolor{Rblack}{RGB}{0,0,0}
	\definecolor{Rgrey}{RGB}{42,51,68}
	\definecolor{RdeepBlue}{RGB}{17,34,170}
	\definecolor{RmidBlue}{RGB}{32,88,220}
	\definecolor{RlightBlue}{RGB}{90,130,234}
	\definecolor{RuglyGreen}{RGB}{253,236,190}
	\definecolor{RyellowPea}{RGB}{253,219,91}
	\definecolor{RdeepYellow}{RGB}{250,185,53}
	\definecolor{Rsalmon}{RGB}{253,152,89}
	\definecolor{Rred}{RGB}{182,52,58}

	\definecolor{lightAvailability_grey}{RGB}{20,20,20}

%% Tikz
\usepackage{tikz}

%%%% Tab, list...
%% Best package for tables
\usepackage{booktabs}

%% List
\usepackage{enumerate}
\usepackage{enumitem}% http://ctan.org/pkg/enumitem

%% Nomenclature (containing three sections: Acronyms, greek symbols, roman symbols) https://tex.stackexchange.com/questions/452107/nomenclature-having-two-or-more-descriptions-and-units-for-a-symbol
\usepackage[intoc]{nomencl}

%% Change caption entries to whatever you want (also done in the ToC)
\addto\captionsenglish{ % Replace "english" with the language you use, if you use babel you might have to replace english by british or USenglish
	\renewcommand{\contentsname}{TABLE DES MATIÈRES}
	\renewcommand{\listtablename}{LISTE DES TABLEAUX}
	\renewcommand{\listfigurename}{LISTE DES FIGURES}
}

%%%% Links
\usepackage{url}
\usepackage[luatex, colorlinks=true, linkcolor=black, urlcolor=RdeepBlue, citecolor=RdeepBlue]{hyperref}

%%%% Bibliography and table of contents
%% Bibliography output using biblatex
\usepackage{csquotes}
\usepackage[style=apa, natbib=true, sorting=ynt]{biblatex}
% \usepackage[uniquelist=false, backend=biber, citestyle=authoryear-comp, sorting=nyt, sortcites=true, giveninits=true, hyperref, natbib, url=false, doi=false, eprint=false, isbn=false, maxbibnames=25, maxcitenames=2, mincitenames=1, uniquename=init, refsection=chapter]{biblatex}
\addbibresource{phd.bib}

%% Table of content (= ToC)
\usepackage[nottoc]{tocbibind}

%%%% Mathematics
\usepackage{amsthm}
\usepackage{amsmath}
	\allowdisplaybreaks % allows breaks between pages in formula
\usepackage{amssymb}
\usepackage{bbold}
\usepackage{dsfont}
\usepackage{mathrsfs}
\usepackage{bm}
\usepackage{xfrac}
\usepackage{etoolbox} % For renumbering
\usepackage{thmbox} % For "theorem" definitions
\usepackage{gensymb}
\usepackage{siunitx}
	\sisetup{
		inter-unit-product=\ensuremath{{}\cdot{}},
		per-mode=symbol
	}

\DeclareMathOperator{\logit}{logit}

%%%% Line numbers --conflict with amsthm-- pdf http://texblog.org/2012/02/08/adding-line-numbers-to-documents/
\usepackage{lineno}
	\linenumbers

%%%%%%%%%%%%%%%%%%%%%   OTHERS   %%%%%%%%%%%%%%%%%%%%%
% http://tex.stackexchange.com/questions/13304/which-package-version-am-i-using
\listfiles % Add \listfiles to your preamble and then look at the .log file. This will tell you the current version of all the packages loaded

%%%%%%%%%%%%%%%%%%%%%   NOMENCLATURE CODING   %%%%%%%%%%%%%%%%%%%%%
\makenomenclature
\renewcommand{\nomname}{LISTE DES ABBRÉVIATIONS}

\ExplSyntaxOn
\RenewDocumentCommand\nomgroup{m}
{
	\item[\textbf{\textcolor{RdeepBlue}{\choosenomgroup{#1}}}]
}
\NewDocumentCommand{\choosenomgroup}{m}
{
	\str_case:nn { #1 }
	{
		{A}{Acronymes}
		{L}{Symboles~latins}
		{G}{Symboles~grecs}
	}
}
\ExplSyntaxOff
\renewcommand*{\nompreamble}{\markboth{\nomname}{\nomname}}

\newcommand{\nomunit}[1]{%
	\renewcommand{\nomentryend}{\hspace*{\fill}\si{#1}}%
}

%%%%%%%%%%%%%%%%%%   HYPHENATION    %%%%%%%%%%%%%%%%%%
%%%% If latex has troubles to split a word, you can teach it here
% \hyphenation{cal-cu-lus}

%%%%%%%%%%%%%%%%%%%%%%%%%%%%%%%%%%%%%%%%%%%%%%%%%%%%%%%%%%%%%%%%%%%%
%%%%%%%%%%%%%%%%%   NEW COMMANDS USER'S SPECIFIC   %%%%%%%%%%%%%%%%%
%%%%%%%%%%%%%%%%%%%%%%%%%%%%%%%%%%%%%%%%%%%%%%%%%%%%%%%%%%%%%%%%%%%%
%%%%%%%% Commands that will be apply to the whole Ph.D. For chapter specific commands, declare them in the chapters (and use renewcommand if a name is used twice)
%%%% Text commands
\newcommand {\ie}{\textit{i.e., }}
\newcommand {\eg}{\textit{e.g., }}
\newcommand {\cf}{\textit{cf} }
\providecommand{\keywords}[1]{\textbf{Mots-clefs:} #1}


%%%% Tikz
%% Libraries (specific tikz commands are in article*/main.tex files)
\usetikzlibrary{arrows, plotmarks, decorations.markings}
\tikzstyle{arrow} = [->,>=stealth,thick,rounded corners=4pt,line width=1pt]
\usetikzlibrary{shadows}
\usetikzlibrary{shadings}
\usetikzlibrary{positioning} % relative coodinate
\usetikzlibrary{tikzmark, calc} % calc, to calculate coordinate
\usetikzlibrary{decorations.pathmorphing} % to snake an arrow
\usetikzlibrary{shapes.arrows}
\usetikzlibrary{patterns}

%% Gradient shadings command
\makeatletter
\def\createshadingfromlist#1#2#3{%
	\pgfutil@tempcnta=0\relax
	\pgfutil@for\pgf@tmp:={#3}\do{\advance\pgfutil@tempcnta by1}%
	\ifnum\pgfutil@tempcnta=1\relax%
		\edef\pgf@spec{color(0)=(#3);color(100)=(#3)}%
	\else%
		\pgfmathparse{50/(\pgfutil@tempcnta-1)}\let\pgf@step=\pgfmathresult%
		\pgfutil@tempcntb=1\relax%
		\pgfutil@for\pgf@tmp:={#3}\do{%
			\ifnum\pgfutil@tempcntb=1\relax%
				\edef\pgf@spec{color(0)=(\pgf@tmp);color(25)=(\pgf@tmp)}%
			\else%
	        \ifnum\pgfutil@tempcntb<\pgfutil@tempcnta\relax%
				\pgfmathparse{25+\pgf@step/4+(\pgfutil@tempcntb-1)*\pgf@step}%
				\edef\pgf@spec{\pgf@spec;color(\pgfmathresult)=(\pgf@tmp)}%
	        \else%
	        	\edef\pgf@spec{\pgf@spec;color(75)=(\pgf@tmp);color(100)=(\pgf@tmp)}%
	        \fi%
	    \fi%
	    \advance\pgfutil@tempcntb by1\relax%
	    }%
	\fi%
	\csname pgfdeclare#2shading\endcsname{#1}{100}\pgf@spec%
}

%%%%%%%% Math commands
%%%% Maths
\newcommand {\s}{{s}^{*}}
\newcommand {\sst}{\tilde{s}^{*}} % s* stable d'où le st
\newcommand {\N}{\tilde{N}}
\newcommand {\A}{\mathscr{A}}
\newcommand {\K}{\mathcal{K}}
\renewcommand{\S}{\mathscr{S}}
\newcommand{\R}{\mathds{R}}
\newcommand{\Prob}{\mathds{P}}
\newcommand{\F}{\mathcal{F}}

%%%% Theorem style
\newtheoremstyle{theo}{\topsep}{\topsep}{\itshape}{}{\bfseries}{.}{\newline}{\thmname{#1} \thmnumber{#2} \thmnote{~: \textit{#3}}}
\theoremstyle{theo}
\newtheorem{rem}{Remark}[section]
\newtheorem{defi}{Definition}[section]
\newtheorem{assum}{Assumption}[section]
\newtheorem{nota}{Notation}[section]

%%%%%%%%%%%%%%%%%%%%%   BEGIN DOCUMENT   %%%%%%%%%%%%%%%%%%%%%
\begin{document}
%%%% Starts by a blank page
\newpage
\thispagestyle{empty}
\mbox{}

%%%% Title page
\begin{titlepage}
	\begin{center}
		\textsc{Your title small caps} \\
	\vspace{2cm}
	par \\
	\vspace{2cm}
	Amaël Le Squin
	\vspace{1cm}
	Université de Sherbrooke \\
	\vspace{1cm}
	Thèse présentée au Département de biologie en vue \\
	de l'obtention du grade de docteur ès sciences (Ph.D.)
	\vfill
	\textsc{faculté des sciences} \\
	\textsc{université de sherbrooke} \\
	\vspace{2cm}
	Sherbrooke, Québec, Canada, \begin{french} \monthyeardate\today \end{french}
	\end{center}	
\end{titlepage}

%%%% Jury page (and not Jimmy page...)
\thispagestyle{empty}
\begin{center}
\begin{french}
	Le \today \\
	\vspace{1cm}
	\textit{Le jury a accepté la thèse d'Amaël Le Squin dans sa version finale} \\
	\vspace{1cm}
	Membre du jury \\
	\vspace{1cm}
	Professeur Dominique Gravel \\
	Directeur de recherche \\
	Département de biologie \\
	Université de Sherbrooke \\
	\vspace{1cm}
	Professeure Isabelle Boulangeat \\
	Codirectrice de recherche \\
	IRSTEA \\
	\vspace{1cm}
	Professeur Prénom et nom \\
	Évaluatrice ou Évaluateur externe \\
	Département nom \\
	Nom de l'institution \\
	\vspace{1cm}
	Professeure Prénom et nom \\
	Évaluatrice ou Évaluateur interne \\
	Département nom \\
	Nom de l'institution \\
	\vspace{1cm}
	Professeure Prénom et nom \\
	Président-rapporteur \\
	Département de biologie \\
	Université de Sherbrooke
\end{french}
\end{center}

%%%% From abstract (included) to Introduction (excluded)
%% Start roman numbering
\pagenumbering{roman}
\input{abstract_fr}

\input{abstract_en}

\input{acknowledgements}

%% Table of contents
\tableofcontents

%% Nomencalture
\printnomenclature

%% List of tables
\listoftables

%% List of figures
\listoffigures

\newpage
%%%% Main body
%% Start arabic numbering
\pagenumbering{arabic}

%% Input chapters
\input{./introduction}

\documentclass[titlepage,oneside,letterpaper,openright,12pt]{report}

%%%%%%%%%%%%%%%%%%%    COMMENTS    %%%%%%%%%%%%%%%%%%%
% 1. To stop numbering the lines, deactivate \linenumbers (activated after loading the package lineno)
% 
% 2. To make the index, you need to compile first with lualatex and then use makeindex. Example:
%		lualatex main.tex
%		makeindex main.nlo -s nomencl.ist -o main.nls
%		lualatex main.tex
% 
% 3. For the bibliography, use the command biber (as I am using the package biblatex, much more modern than bibtex).
%		To use multiple bibliography, use the refsection environment (3.13.3 Multiple Bibliographies in biblatex doc Version 3.13).
%		Do not use the chapterbib package with biblatex, there is refsection environment for that!
%		Each article will need to use
%		\begin{refsection}
%			Intro, Methods, Results, Discussion
%			\printbibliography[heading=subbibnumbered]
%		\end{refsection}
% 
% 4. I defined two languages in the document, French and British english (which is the main language), use \begin{french} ... \end{french} to switch to French in a paragraph
% 
% 5. To avoid multi-defined labels, I recommend to apply this command line in your shell for each tex file of the folders ./article1/, ./article2/, ...:
% 		sed -i.tmp 's/\\label{/\\label{WHATEVER::/g' *.tex
% 			This will change all the labels by adding WHATEVER before. Example (adding article1 for each tex file in ./article1/ folder):
% 			\label{sec::intro} in ./article1/chap1.tex becomes \label{article1::sec::intro}
%		sed -i.tmp 's/\\ref{/\\ref{WHATEVER::/g' *.tex
%			This will change all the refs by adding WHATEVER before. Example (adding article1 for each tex file in ./article1/ folder):
% 			\ref{sec::intro} in ./article1/chap1.tex becomes \ref{article1::sec::intro}
% 		sed -i.tmp 's/\\eqref{/\\eqref{WHATEVER::/g' *.tex
%			This will change all the eqrefs by adding WHATEVER before. Example (adding article1 for each tex file in ./article1/ folder):
% 			\eqref{eq::R0_equation} in ./article1/chap1.tex becomes \eqref{article1::eq::R0_equation}
% 
%	For safety reasons, the original file will be kept in a temporary file '.tmp'. Example:
%		sed -i.tmp 'Your instructions here' introduction.tex will create two files:
% 			introduction.tex which should be the one to keep (if no mistakes in the sed instructions)
% 			introduction.tex.tmp which is the original file (that you can dump if sed behaved the way you expected)
% 	For an introduction to the power of sed and regular expressions: https://www.grymoire.com/Unix/Sed.html#uh-1
% 
% 6. I defined some colours in this document. Except RdeepBlue, none of them is used in this template and can be deleted.
% 	If you delete RdeepBlue, do not forget to adapt all the commands using this colour

%%%%%%%%%%%%%%%%%%%    PACKAGES    %%%%%%%%%%%%%%%%%%%
%%%% font
\usepackage{fontspec}
\usepackage{marvosym}

%%%% Language
\usepackage{polyglossia}
	\setdefaultlanguage[variant=british]{english}
	\setotherlanguages{french}

%%%% Date
\usepackage{datetime}
	\newdateformat{monthyeardate}{\monthname[\THEMONTH] \THEYEAR}

%%%% marges, space, titles...
\usepackage[top=3.75cm, bottom=2.5cm, left=3cm, right=2.5cm]{geometry}
% \usepackage{setspace}
\setlength{\parindent}{0ex}   % indentation au début de chaque paragraphe
\setlength{\parskip}{3ex plus 0.5ex minus 0.2ex} % espace vertical entre paragraphes

\usepackage{titlesec}
\newcommand\chapterstring{CHAPITRE}
\titleformat{\chapter}[display]{\vspace{-8em}\center\bfseries}{\chapterstring~\thechapter}{12pt}{}[\vspace{1.5ex}]
\titleformat{\section}{\vspace{0mm plus 2cm}\addpenalty{-1000}\bfseries\sffamily}{\thesection}{1em}{}
\titleformat{\subsection}{\addpenalty{-500}\bfseries\sffamily}{\thesubsection}{1em}{}
\titleformat{\subsubsection}{\penalty-500\vspace{0pt plus 2pt}{}\bfseries\sffamily}{\thesubsubsection}{}{}[\vspace{-14pt}]

\usepackage{titletoc}
	\titlecontents{chapter}[0pt]{}{\bfseries\chapterstring~\thecontentslabel~}{\bfseries}{\hspace{1em plus 1fill}\contentspage}

%%%% Appendix
\usepackage{appendix}
%% Start of subappendices environment
\AtBeginEnvironment{subappendices}{%
	\chapter*{Appendix}
	\addcontentsline{toc}{chapter}{Appendices}
	\counterwithin{figure}{section}
	\counterwithin{table}{section}
}

%% End of subappendices environment
\AtEndEnvironment{subappendices}{%
	\counterwithout{figure}{section}
	\counterwithout{table}{section}
}

%%%% Graphics
%% To compile with lualatex
\usepackage[luatex]{graphicx}

%% Graphics' path, but I prefer to use absolute path for safety
% \graphicspath{{./article1/}{./article2/}{article3/}}

%% Numbering figures in sections rather than continuously
\usepackage{chngcntr}
\counterwithin{figure}{section}

%% Put figures within their declaration section 
\usepackage[section]{placeins}

%% To create multi-figures
\usepackage{caption}
\usepackage{subcaption}

%% Colours
\usepackage[dvipsnames, svgnames]{xcolor}
	\definecolor{Rblack}{RGB}{0,0,0}
	\definecolor{Rgrey}{RGB}{42,51,68}
	\definecolor{RdeepBlue}{RGB}{17,34,170}
	\definecolor{RmidBlue}{RGB}{32,88,220}
	\definecolor{RlightBlue}{RGB}{90,130,234}
	\definecolor{RuglyGreen}{RGB}{253,236,190}
	\definecolor{RyellowPea}{RGB}{253,219,91}
	\definecolor{RdeepYellow}{RGB}{250,185,53}
	\definecolor{Rsalmon}{RGB}{253,152,89}
	\definecolor{Rred}{RGB}{182,52,58}

	\definecolor{lightAvailability_grey}{RGB}{20,20,20}

%% Tikz
\usepackage{tikz}

%%%% Tab, list...
%% Best package for tables
\usepackage{booktabs}

%% List
\usepackage{enumerate}
\usepackage{enumitem}% http://ctan.org/pkg/enumitem

%% Nomenclature (containing three sections: Acronyms, greek symbols, roman symbols) https://tex.stackexchange.com/questions/452107/nomenclature-having-two-or-more-descriptions-and-units-for-a-symbol
\usepackage[intoc]{nomencl}

%% Change caption entries to whatever you want (also done in the ToC)
\addto\captionsenglish{ % Replace "english" with the language you use, if you use babel you might have to replace english by british or USenglish
	\renewcommand{\contentsname}{TABLE DES MATIÈRES}
	\renewcommand{\listtablename}{LISTE DES TABLEAUX}
	\renewcommand{\listfigurename}{LISTE DES FIGURES}
}

%%%% Links
\usepackage{url}
\usepackage[luatex, colorlinks=true, linkcolor=black, urlcolor=RdeepBlue, citecolor=RdeepBlue]{hyperref}

%%%% Bibliography and table of contents
%% Bibliography output using biblatex
\usepackage{csquotes}
\usepackage[style=apa, natbib=true, sorting=ynt]{biblatex}
% \usepackage[uniquelist=false, backend=biber, citestyle=authoryear-comp, sorting=nyt, sortcites=true, giveninits=true, hyperref, natbib, url=false, doi=false, eprint=false, isbn=false, maxbibnames=25, maxcitenames=2, mincitenames=1, uniquename=init, refsection=chapter]{biblatex}
\addbibresource{phd.bib}

%% Table of content (= ToC)
\usepackage[nottoc]{tocbibind}

%%%% Mathematics
\usepackage{amsthm}
\usepackage{amsmath}
	\allowdisplaybreaks % allows breaks between pages in formula
\usepackage{amssymb}
\usepackage{bbold}
\usepackage{dsfont}
\usepackage{mathrsfs}
\usepackage{bm}
\usepackage{xfrac}
\usepackage{etoolbox} % For renumbering
\usepackage{thmbox} % For "theorem" definitions
\usepackage{gensymb}
\usepackage{siunitx}
	\sisetup{
		inter-unit-product=\ensuremath{{}\cdot{}},
		per-mode=symbol
	}

\DeclareMathOperator{\logit}{logit}

%%%% Line numbers --conflict with amsthm-- pdf http://texblog.org/2012/02/08/adding-line-numbers-to-documents/
\usepackage{lineno}
	\linenumbers

%%%%%%%%%%%%%%%%%%%%%   OTHERS   %%%%%%%%%%%%%%%%%%%%%
% http://tex.stackexchange.com/questions/13304/which-package-version-am-i-using
\listfiles % Add \listfiles to your preamble and then look at the .log file. This will tell you the current version of all the packages loaded

%%%%%%%%%%%%%%%%%%%%%   NOMENCLATURE CODING   %%%%%%%%%%%%%%%%%%%%%
\makenomenclature
\renewcommand{\nomname}{LISTE DES ABBRÉVIATIONS}

\ExplSyntaxOn
\RenewDocumentCommand\nomgroup{m}
{
	\item[\textbf{\textcolor{RdeepBlue}{\choosenomgroup{#1}}}]
}
\NewDocumentCommand{\choosenomgroup}{m}
{
	\str_case:nn { #1 }
	{
		{A}{Acronymes}
		{L}{Symboles~latins}
		{G}{Symboles~grecs}
	}
}
\ExplSyntaxOff
\renewcommand*{\nompreamble}{\markboth{\nomname}{\nomname}}

\newcommand{\nomunit}[1]{%
	\renewcommand{\nomentryend}{\hspace*{\fill}\si{#1}}%
}

%%%%%%%%%%%%%%%%%%   HYPHENATION    %%%%%%%%%%%%%%%%%%
%%%% If latex has troubles to split a word, you can teach it here
% \hyphenation{cal-cu-lus}

%%%%%%%%%%%%%%%%%%%%%%%%%%%%%%%%%%%%%%%%%%%%%%%%%%%%%%%%%%%%%%%%%%%%
%%%%%%%%%%%%%%%%%   NEW COMMANDS USER'S SPECIFIC   %%%%%%%%%%%%%%%%%
%%%%%%%%%%%%%%%%%%%%%%%%%%%%%%%%%%%%%%%%%%%%%%%%%%%%%%%%%%%%%%%%%%%%
%%%%%%%% Commands that will be apply to the whole Ph.D. For chapter specific commands, declare them in the chapters (and use renewcommand if a name is used twice)
%%%% Text commands
\newcommand {\ie}{\textit{i.e., }}
\newcommand {\eg}{\textit{e.g., }}
\newcommand {\cf}{\textit{cf} }
\providecommand{\keywords}[1]{\textbf{Mots-clefs:} #1}


%%%% Tikz
%% Libraries (specific tikz commands are in article*/main.tex files)
\usetikzlibrary{arrows, plotmarks, decorations.markings}
\tikzstyle{arrow} = [->,>=stealth,thick,rounded corners=4pt,line width=1pt]
\usetikzlibrary{shadows}
\usetikzlibrary{shadings}
\usetikzlibrary{positioning} % relative coodinate
\usetikzlibrary{tikzmark, calc} % calc, to calculate coordinate
\usetikzlibrary{decorations.pathmorphing} % to snake an arrow
\usetikzlibrary{shapes.arrows}
\usetikzlibrary{patterns}

%% Gradient shadings command
\makeatletter
\def\createshadingfromlist#1#2#3{%
	\pgfutil@tempcnta=0\relax
	\pgfutil@for\pgf@tmp:={#3}\do{\advance\pgfutil@tempcnta by1}%
	\ifnum\pgfutil@tempcnta=1\relax%
		\edef\pgf@spec{color(0)=(#3);color(100)=(#3)}%
	\else%
		\pgfmathparse{50/(\pgfutil@tempcnta-1)}\let\pgf@step=\pgfmathresult%
		\pgfutil@tempcntb=1\relax%
		\pgfutil@for\pgf@tmp:={#3}\do{%
			\ifnum\pgfutil@tempcntb=1\relax%
				\edef\pgf@spec{color(0)=(\pgf@tmp);color(25)=(\pgf@tmp)}%
			\else%
	        \ifnum\pgfutil@tempcntb<\pgfutil@tempcnta\relax%
				\pgfmathparse{25+\pgf@step/4+(\pgfutil@tempcntb-1)*\pgf@step}%
				\edef\pgf@spec{\pgf@spec;color(\pgfmathresult)=(\pgf@tmp)}%
	        \else%
	        	\edef\pgf@spec{\pgf@spec;color(75)=(\pgf@tmp);color(100)=(\pgf@tmp)}%
	        \fi%
	    \fi%
	    \advance\pgfutil@tempcntb by1\relax%
	    }%
	\fi%
	\csname pgfdeclare#2shading\endcsname{#1}{100}\pgf@spec%
}

%%%%%%%% Math commands
%%%% Maths
\newcommand {\s}{{s}^{*}}
\newcommand {\sst}{\tilde{s}^{*}} % s* stable d'où le st
\newcommand {\N}{\tilde{N}}
\newcommand {\A}{\mathscr{A}}
\newcommand {\K}{\mathcal{K}}
\renewcommand{\S}{\mathscr{S}}
\newcommand{\R}{\mathds{R}}
\newcommand{\Prob}{\mathds{P}}
\newcommand{\F}{\mathcal{F}}

%%%% Theorem style
\newtheoremstyle{theo}{\topsep}{\topsep}{\itshape}{}{\bfseries}{.}{\newline}{\thmname{#1} \thmnumber{#2} \thmnote{~: \textit{#3}}}
\theoremstyle{theo}
\newtheorem{rem}{Remark}[section]
\newtheorem{defi}{Definition}[section]
\newtheorem{assum}{Assumption}[section]
\newtheorem{nota}{Notation}[section]

%%%%%%%%%%%%%%%%%%%%%   BEGIN DOCUMENT   %%%%%%%%%%%%%%%%%%%%%
\begin{document}
%%%% Starts by a blank page
\newpage
\thispagestyle{empty}
\mbox{}

%%%% Title page
\begin{titlepage}
	\begin{center}
		\textsc{Your title small caps} \\
	\vspace{2cm}
	par \\
	\vspace{2cm}
	Amaël Le Squin
	\vspace{1cm}
	Université de Sherbrooke \\
	\vspace{1cm}
	Thèse présentée au Département de biologie en vue \\
	de l'obtention du grade de docteur ès sciences (Ph.D.)
	\vfill
	\textsc{faculté des sciences} \\
	\textsc{université de sherbrooke} \\
	\vspace{2cm}
	Sherbrooke, Québec, Canada, \begin{french} \monthyeardate\today \end{french}
	\end{center}	
\end{titlepage}

%%%% Jury page (and not Jimmy page...)
\thispagestyle{empty}
\begin{center}
\begin{french}
	Le \today \\
	\vspace{1cm}
	\textit{Le jury a accepté la thèse d'Amaël Le Squin dans sa version finale} \\
	\vspace{1cm}
	Membre du jury \\
	\vspace{1cm}
	Professeur Dominique Gravel \\
	Directeur de recherche \\
	Département de biologie \\
	Université de Sherbrooke \\
	\vspace{1cm}
	Professeure Isabelle Boulangeat \\
	Codirectrice de recherche \\
	IRSTEA \\
	\vspace{1cm}
	Professeur Prénom et nom \\
	Évaluatrice ou Évaluateur externe \\
	Département nom \\
	Nom de l'institution \\
	\vspace{1cm}
	Professeure Prénom et nom \\
	Évaluatrice ou Évaluateur interne \\
	Département nom \\
	Nom de l'institution \\
	\vspace{1cm}
	Professeure Prénom et nom \\
	Président-rapporteur \\
	Département de biologie \\
	Université de Sherbrooke
\end{french}
\end{center}

%%%% From abstract (included) to Introduction (excluded)
%% Start roman numbering
\pagenumbering{roman}
\input{abstract_fr}

\input{abstract_en}

\input{acknowledgements}

%% Table of contents
\tableofcontents

%% Nomencalture
\printnomenclature

%% List of tables
\listoftables

%% List of figures
\listoffigures

\newpage
%%%% Main body
%% Start arabic numbering
\pagenumbering{arabic}

%% Input chapters
\input{./introduction}

\documentclass[titlepage,oneside,letterpaper,openright,12pt]{report}

%%%%%%%%%%%%%%%%%%%    COMMENTS    %%%%%%%%%%%%%%%%%%%
% 1. To stop numbering the lines, deactivate \linenumbers (activated after loading the package lineno)
% 
% 2. To make the index, you need to compile first with lualatex and then use makeindex. Example:
%		lualatex main.tex
%		makeindex main.nlo -s nomencl.ist -o main.nls
%		lualatex main.tex
% 
% 3. For the bibliography, use the command biber (as I am using the package biblatex, much more modern than bibtex).
%		To use multiple bibliography, use the refsection environment (3.13.3 Multiple Bibliographies in biblatex doc Version 3.13).
%		Do not use the chapterbib package with biblatex, there is refsection environment for that!
%		Each article will need to use
%		\begin{refsection}
%			Intro, Methods, Results, Discussion
%			\printbibliography[heading=subbibnumbered]
%		\end{refsection}
% 
% 4. I defined two languages in the document, French and British english (which is the main language), use \begin{french} ... \end{french} to switch to French in a paragraph
% 
% 5. To avoid multi-defined labels, I recommend to apply this command line in your shell for each tex file of the folders ./article1/, ./article2/, ...:
% 		sed -i.tmp 's/\\label{/\\label{WHATEVER::/g' *.tex
% 			This will change all the labels by adding WHATEVER before. Example (adding article1 for each tex file in ./article1/ folder):
% 			\label{sec::intro} in ./article1/chap1.tex becomes \label{article1::sec::intro}
%		sed -i.tmp 's/\\ref{/\\ref{WHATEVER::/g' *.tex
%			This will change all the refs by adding WHATEVER before. Example (adding article1 for each tex file in ./article1/ folder):
% 			\ref{sec::intro} in ./article1/chap1.tex becomes \ref{article1::sec::intro}
% 		sed -i.tmp 's/\\eqref{/\\eqref{WHATEVER::/g' *.tex
%			This will change all the eqrefs by adding WHATEVER before. Example (adding article1 for each tex file in ./article1/ folder):
% 			\eqref{eq::R0_equation} in ./article1/chap1.tex becomes \eqref{article1::eq::R0_equation}
% 
%	For safety reasons, the original file will be kept in a temporary file '.tmp'. Example:
%		sed -i.tmp 'Your instructions here' introduction.tex will create two files:
% 			introduction.tex which should be the one to keep (if no mistakes in the sed instructions)
% 			introduction.tex.tmp which is the original file (that you can dump if sed behaved the way you expected)
% 	For an introduction to the power of sed and regular expressions: https://www.grymoire.com/Unix/Sed.html#uh-1
% 
% 6. I defined some colours in this document. Except RdeepBlue, none of them is used in this template and can be deleted.
% 	If you delete RdeepBlue, do not forget to adapt all the commands using this colour

%%%%%%%%%%%%%%%%%%%    PACKAGES    %%%%%%%%%%%%%%%%%%%
%%%% font
\usepackage{fontspec}
\usepackage{marvosym}

%%%% Language
\usepackage{polyglossia}
	\setdefaultlanguage[variant=british]{english}
	\setotherlanguages{french}

%%%% Date
\usepackage{datetime}
	\newdateformat{monthyeardate}{\monthname[\THEMONTH] \THEYEAR}

%%%% marges, space, titles...
\usepackage[top=3.75cm, bottom=2.5cm, left=3cm, right=2.5cm]{geometry}
% \usepackage{setspace}
\setlength{\parindent}{0ex}   % indentation au début de chaque paragraphe
\setlength{\parskip}{3ex plus 0.5ex minus 0.2ex} % espace vertical entre paragraphes

\usepackage{titlesec}
\newcommand\chapterstring{CHAPITRE}
\titleformat{\chapter}[display]{\vspace{-8em}\center\bfseries}{\chapterstring~\thechapter}{12pt}{}[\vspace{1.5ex}]
\titleformat{\section}{\vspace{0mm plus 2cm}\addpenalty{-1000}\bfseries\sffamily}{\thesection}{1em}{}
\titleformat{\subsection}{\addpenalty{-500}\bfseries\sffamily}{\thesubsection}{1em}{}
\titleformat{\subsubsection}{\penalty-500\vspace{0pt plus 2pt}{}\bfseries\sffamily}{\thesubsubsection}{}{}[\vspace{-14pt}]

\usepackage{titletoc}
	\titlecontents{chapter}[0pt]{}{\bfseries\chapterstring~\thecontentslabel~}{\bfseries}{\hspace{1em plus 1fill}\contentspage}

%%%% Appendix
\usepackage{appendix}
%% Start of subappendices environment
\AtBeginEnvironment{subappendices}{%
	\chapter*{Appendix}
	\addcontentsline{toc}{chapter}{Appendices}
	\counterwithin{figure}{section}
	\counterwithin{table}{section}
}

%% End of subappendices environment
\AtEndEnvironment{subappendices}{%
	\counterwithout{figure}{section}
	\counterwithout{table}{section}
}

%%%% Graphics
%% To compile with lualatex
\usepackage[luatex]{graphicx}

%% Graphics' path, but I prefer to use absolute path for safety
% \graphicspath{{./article1/}{./article2/}{article3/}}

%% Numbering figures in sections rather than continuously
\usepackage{chngcntr}
\counterwithin{figure}{section}

%% Put figures within their declaration section 
\usepackage[section]{placeins}

%% To create multi-figures
\usepackage{caption}
\usepackage{subcaption}

%% Colours
\usepackage[dvipsnames, svgnames]{xcolor}
	\definecolor{Rblack}{RGB}{0,0,0}
	\definecolor{Rgrey}{RGB}{42,51,68}
	\definecolor{RdeepBlue}{RGB}{17,34,170}
	\definecolor{RmidBlue}{RGB}{32,88,220}
	\definecolor{RlightBlue}{RGB}{90,130,234}
	\definecolor{RuglyGreen}{RGB}{253,236,190}
	\definecolor{RyellowPea}{RGB}{253,219,91}
	\definecolor{RdeepYellow}{RGB}{250,185,53}
	\definecolor{Rsalmon}{RGB}{253,152,89}
	\definecolor{Rred}{RGB}{182,52,58}

	\definecolor{lightAvailability_grey}{RGB}{20,20,20}

%% Tikz
\usepackage{tikz}

%%%% Tab, list...
%% Best package for tables
\usepackage{booktabs}

%% List
\usepackage{enumerate}
\usepackage{enumitem}% http://ctan.org/pkg/enumitem

%% Nomenclature (containing three sections: Acronyms, greek symbols, roman symbols) https://tex.stackexchange.com/questions/452107/nomenclature-having-two-or-more-descriptions-and-units-for-a-symbol
\usepackage[intoc]{nomencl}

%% Change caption entries to whatever you want (also done in the ToC)
\addto\captionsenglish{ % Replace "english" with the language you use, if you use babel you might have to replace english by british or USenglish
	\renewcommand{\contentsname}{TABLE DES MATIÈRES}
	\renewcommand{\listtablename}{LISTE DES TABLEAUX}
	\renewcommand{\listfigurename}{LISTE DES FIGURES}
}

%%%% Links
\usepackage{url}
\usepackage[luatex, colorlinks=true, linkcolor=black, urlcolor=RdeepBlue, citecolor=RdeepBlue]{hyperref}

%%%% Bibliography and table of contents
%% Bibliography output using biblatex
\usepackage{csquotes}
\usepackage[style=apa, natbib=true, sorting=ynt]{biblatex}
% \usepackage[uniquelist=false, backend=biber, citestyle=authoryear-comp, sorting=nyt, sortcites=true, giveninits=true, hyperref, natbib, url=false, doi=false, eprint=false, isbn=false, maxbibnames=25, maxcitenames=2, mincitenames=1, uniquename=init, refsection=chapter]{biblatex}
\addbibresource{phd.bib}

%% Table of content (= ToC)
\usepackage[nottoc]{tocbibind}

%%%% Mathematics
\usepackage{amsthm}
\usepackage{amsmath}
	\allowdisplaybreaks % allows breaks between pages in formula
\usepackage{amssymb}
\usepackage{bbold}
\usepackage{dsfont}
\usepackage{mathrsfs}
\usepackage{bm}
\usepackage{xfrac}
\usepackage{etoolbox} % For renumbering
\usepackage{thmbox} % For "theorem" definitions
\usepackage{gensymb}
\usepackage{siunitx}
	\sisetup{
		inter-unit-product=\ensuremath{{}\cdot{}},
		per-mode=symbol
	}

\DeclareMathOperator{\logit}{logit}

%%%% Line numbers --conflict with amsthm-- pdf http://texblog.org/2012/02/08/adding-line-numbers-to-documents/
\usepackage{lineno}
	\linenumbers

%%%%%%%%%%%%%%%%%%%%%   OTHERS   %%%%%%%%%%%%%%%%%%%%%
% http://tex.stackexchange.com/questions/13304/which-package-version-am-i-using
\listfiles % Add \listfiles to your preamble and then look at the .log file. This will tell you the current version of all the packages loaded

%%%%%%%%%%%%%%%%%%%%%   NOMENCLATURE CODING   %%%%%%%%%%%%%%%%%%%%%
\makenomenclature
\renewcommand{\nomname}{LISTE DES ABBRÉVIATIONS}

\ExplSyntaxOn
\RenewDocumentCommand\nomgroup{m}
{
	\item[\textbf{\textcolor{RdeepBlue}{\choosenomgroup{#1}}}]
}
\NewDocumentCommand{\choosenomgroup}{m}
{
	\str_case:nn { #1 }
	{
		{A}{Acronymes}
		{L}{Symboles~latins}
		{G}{Symboles~grecs}
	}
}
\ExplSyntaxOff
\renewcommand*{\nompreamble}{\markboth{\nomname}{\nomname}}

\newcommand{\nomunit}[1]{%
	\renewcommand{\nomentryend}{\hspace*{\fill}\si{#1}}%
}

%%%%%%%%%%%%%%%%%%   HYPHENATION    %%%%%%%%%%%%%%%%%%
%%%% If latex has troubles to split a word, you can teach it here
% \hyphenation{cal-cu-lus}

%%%%%%%%%%%%%%%%%%%%%%%%%%%%%%%%%%%%%%%%%%%%%%%%%%%%%%%%%%%%%%%%%%%%
%%%%%%%%%%%%%%%%%   NEW COMMANDS USER'S SPECIFIC   %%%%%%%%%%%%%%%%%
%%%%%%%%%%%%%%%%%%%%%%%%%%%%%%%%%%%%%%%%%%%%%%%%%%%%%%%%%%%%%%%%%%%%
%%%%%%%% Commands that will be apply to the whole Ph.D. For chapter specific commands, declare them in the chapters (and use renewcommand if a name is used twice)
%%%% Text commands
\newcommand {\ie}{\textit{i.e., }}
\newcommand {\eg}{\textit{e.g., }}
\newcommand {\cf}{\textit{cf} }
\providecommand{\keywords}[1]{\textbf{Mots-clefs:} #1}


%%%% Tikz
%% Libraries (specific tikz commands are in article*/main.tex files)
\usetikzlibrary{arrows, plotmarks, decorations.markings}
\tikzstyle{arrow} = [->,>=stealth,thick,rounded corners=4pt,line width=1pt]
\usetikzlibrary{shadows}
\usetikzlibrary{shadings}
\usetikzlibrary{positioning} % relative coodinate
\usetikzlibrary{tikzmark, calc} % calc, to calculate coordinate
\usetikzlibrary{decorations.pathmorphing} % to snake an arrow
\usetikzlibrary{shapes.arrows}
\usetikzlibrary{patterns}

%% Gradient shadings command
\makeatletter
\def\createshadingfromlist#1#2#3{%
	\pgfutil@tempcnta=0\relax
	\pgfutil@for\pgf@tmp:={#3}\do{\advance\pgfutil@tempcnta by1}%
	\ifnum\pgfutil@tempcnta=1\relax%
		\edef\pgf@spec{color(0)=(#3);color(100)=(#3)}%
	\else%
		\pgfmathparse{50/(\pgfutil@tempcnta-1)}\let\pgf@step=\pgfmathresult%
		\pgfutil@tempcntb=1\relax%
		\pgfutil@for\pgf@tmp:={#3}\do{%
			\ifnum\pgfutil@tempcntb=1\relax%
				\edef\pgf@spec{color(0)=(\pgf@tmp);color(25)=(\pgf@tmp)}%
			\else%
	        \ifnum\pgfutil@tempcntb<\pgfutil@tempcnta\relax%
				\pgfmathparse{25+\pgf@step/4+(\pgfutil@tempcntb-1)*\pgf@step}%
				\edef\pgf@spec{\pgf@spec;color(\pgfmathresult)=(\pgf@tmp)}%
	        \else%
	        	\edef\pgf@spec{\pgf@spec;color(75)=(\pgf@tmp);color(100)=(\pgf@tmp)}%
	        \fi%
	    \fi%
	    \advance\pgfutil@tempcntb by1\relax%
	    }%
	\fi%
	\csname pgfdeclare#2shading\endcsname{#1}{100}\pgf@spec%
}

%%%%%%%% Math commands
%%%% Maths
\newcommand {\s}{{s}^{*}}
\newcommand {\sst}{\tilde{s}^{*}} % s* stable d'où le st
\newcommand {\N}{\tilde{N}}
\newcommand {\A}{\mathscr{A}}
\newcommand {\K}{\mathcal{K}}
\renewcommand{\S}{\mathscr{S}}
\newcommand{\R}{\mathds{R}}
\newcommand{\Prob}{\mathds{P}}
\newcommand{\F}{\mathcal{F}}

%%%% Theorem style
\newtheoremstyle{theo}{\topsep}{\topsep}{\itshape}{}{\bfseries}{.}{\newline}{\thmname{#1} \thmnumber{#2} \thmnote{~: \textit{#3}}}
\theoremstyle{theo}
\newtheorem{rem}{Remark}[section]
\newtheorem{defi}{Definition}[section]
\newtheorem{assum}{Assumption}[section]
\newtheorem{nota}{Notation}[section]

%%%%%%%%%%%%%%%%%%%%%   BEGIN DOCUMENT   %%%%%%%%%%%%%%%%%%%%%
\begin{document}
%%%% Starts by a blank page
\newpage
\thispagestyle{empty}
\mbox{}

%%%% Title page
\begin{titlepage}
	\begin{center}
		\textsc{Your title small caps} \\
	\vspace{2cm}
	par \\
	\vspace{2cm}
	Amaël Le Squin
	\vspace{1cm}
	Université de Sherbrooke \\
	\vspace{1cm}
	Thèse présentée au Département de biologie en vue \\
	de l'obtention du grade de docteur ès sciences (Ph.D.)
	\vfill
	\textsc{faculté des sciences} \\
	\textsc{université de sherbrooke} \\
	\vspace{2cm}
	Sherbrooke, Québec, Canada, \begin{french} \monthyeardate\today \end{french}
	\end{center}	
\end{titlepage}

%%%% Jury page (and not Jimmy page...)
\thispagestyle{empty}
\begin{center}
\begin{french}
	Le \today \\
	\vspace{1cm}
	\textit{Le jury a accepté la thèse d'Amaël Le Squin dans sa version finale} \\
	\vspace{1cm}
	Membre du jury \\
	\vspace{1cm}
	Professeur Dominique Gravel \\
	Directeur de recherche \\
	Département de biologie \\
	Université de Sherbrooke \\
	\vspace{1cm}
	Professeure Isabelle Boulangeat \\
	Codirectrice de recherche \\
	IRSTEA \\
	\vspace{1cm}
	Professeur Prénom et nom \\
	Évaluatrice ou Évaluateur externe \\
	Département nom \\
	Nom de l'institution \\
	\vspace{1cm}
	Professeure Prénom et nom \\
	Évaluatrice ou Évaluateur interne \\
	Département nom \\
	Nom de l'institution \\
	\vspace{1cm}
	Professeure Prénom et nom \\
	Président-rapporteur \\
	Département de biologie \\
	Université de Sherbrooke
\end{french}
\end{center}

%%%% From abstract (included) to Introduction (excluded)
%% Start roman numbering
\pagenumbering{roman}
\input{abstract_fr}

\input{abstract_en}

\input{acknowledgements}

%% Table of contents
\tableofcontents

%% Nomencalture
\printnomenclature

%% List of tables
\listoftables

%% List of figures
\listoffigures

\newpage
%%%% Main body
%% Start arabic numbering
\pagenumbering{arabic}

%% Input chapters
\input{./introduction}

\documentclass[titlepage,oneside,letterpaper,openright,12pt]{report}

%%%%%%%%%%%%%%%%%%%    COMMENTS    %%%%%%%%%%%%%%%%%%%
% 1. To stop numbering the lines, deactivate \linenumbers (activated after loading the package lineno)
% 
% 2. To make the index, you need to compile first with lualatex and then use makeindex. Example:
%		lualatex main.tex
%		makeindex main.nlo -s nomencl.ist -o main.nls
%		lualatex main.tex
% 
% 3. For the bibliography, use the command biber (as I am using the package biblatex, much more modern than bibtex).
%		To use multiple bibliography, use the refsection environment (3.13.3 Multiple Bibliographies in biblatex doc Version 3.13).
%		Do not use the chapterbib package with biblatex, there is refsection environment for that!
%		Each article will need to use
%		\begin{refsection}
%			Intro, Methods, Results, Discussion
%			\printbibliography[heading=subbibnumbered]
%		\end{refsection}
% 
% 4. I defined two languages in the document, French and British english (which is the main language), use \begin{french} ... \end{french} to switch to French in a paragraph
% 
% 5. To avoid multi-defined labels, I recommend to apply this command line in your shell for each tex file of the folders ./article1/, ./article2/, ...:
% 		sed -i.tmp 's/\\label{/\\label{WHATEVER::/g' *.tex
% 			This will change all the labels by adding WHATEVER before. Example (adding article1 for each tex file in ./article1/ folder):
% 			\label{sec::intro} in ./article1/chap1.tex becomes \label{article1::sec::intro}
%		sed -i.tmp 's/\\ref{/\\ref{WHATEVER::/g' *.tex
%			This will change all the refs by adding WHATEVER before. Example (adding article1 for each tex file in ./article1/ folder):
% 			\ref{sec::intro} in ./article1/chap1.tex becomes \ref{article1::sec::intro}
% 		sed -i.tmp 's/\\eqref{/\\eqref{WHATEVER::/g' *.tex
%			This will change all the eqrefs by adding WHATEVER before. Example (adding article1 for each tex file in ./article1/ folder):
% 			\eqref{eq::R0_equation} in ./article1/chap1.tex becomes \eqref{article1::eq::R0_equation}
% 
%	For safety reasons, the original file will be kept in a temporary file '.tmp'. Example:
%		sed -i.tmp 'Your instructions here' introduction.tex will create two files:
% 			introduction.tex which should be the one to keep (if no mistakes in the sed instructions)
% 			introduction.tex.tmp which is the original file (that you can dump if sed behaved the way you expected)
% 	For an introduction to the power of sed and regular expressions: https://www.grymoire.com/Unix/Sed.html#uh-1
% 
% 6. I defined some colours in this document. Except RdeepBlue, none of them is used in this template and can be deleted.
% 	If you delete RdeepBlue, do not forget to adapt all the commands using this colour

%%%%%%%%%%%%%%%%%%%    PACKAGES    %%%%%%%%%%%%%%%%%%%
%%%% font
\usepackage{fontspec}
\usepackage{marvosym}

%%%% Language
\usepackage{polyglossia}
	\setdefaultlanguage[variant=british]{english}
	\setotherlanguages{french}

%%%% Date
\usepackage{datetime}
	\newdateformat{monthyeardate}{\monthname[\THEMONTH] \THEYEAR}

%%%% marges, space, titles...
\usepackage[top=3.75cm, bottom=2.5cm, left=3cm, right=2.5cm]{geometry}
% \usepackage{setspace}
\setlength{\parindent}{0ex}   % indentation au début de chaque paragraphe
\setlength{\parskip}{3ex plus 0.5ex minus 0.2ex} % espace vertical entre paragraphes

\usepackage{titlesec}
\newcommand\chapterstring{CHAPITRE}
\titleformat{\chapter}[display]{\vspace{-8em}\center\bfseries}{\chapterstring~\thechapter}{12pt}{}[\vspace{1.5ex}]
\titleformat{\section}{\vspace{0mm plus 2cm}\addpenalty{-1000}\bfseries\sffamily}{\thesection}{1em}{}
\titleformat{\subsection}{\addpenalty{-500}\bfseries\sffamily}{\thesubsection}{1em}{}
\titleformat{\subsubsection}{\penalty-500\vspace{0pt plus 2pt}{}\bfseries\sffamily}{\thesubsubsection}{}{}[\vspace{-14pt}]

\usepackage{titletoc}
	\titlecontents{chapter}[0pt]{}{\bfseries\chapterstring~\thecontentslabel~}{\bfseries}{\hspace{1em plus 1fill}\contentspage}

%%%% Appendix
\usepackage{appendix}
%% Start of subappendices environment
\AtBeginEnvironment{subappendices}{%
	\chapter*{Appendix}
	\addcontentsline{toc}{chapter}{Appendices}
	\counterwithin{figure}{section}
	\counterwithin{table}{section}
}

%% End of subappendices environment
\AtEndEnvironment{subappendices}{%
	\counterwithout{figure}{section}
	\counterwithout{table}{section}
}

%%%% Graphics
%% To compile with lualatex
\usepackage[luatex]{graphicx}

%% Graphics' path, but I prefer to use absolute path for safety
% \graphicspath{{./article1/}{./article2/}{article3/}}

%% Numbering figures in sections rather than continuously
\usepackage{chngcntr}
\counterwithin{figure}{section}

%% Put figures within their declaration section 
\usepackage[section]{placeins}

%% To create multi-figures
\usepackage{caption}
\usepackage{subcaption}

%% Colours
\usepackage[dvipsnames, svgnames]{xcolor}
	\definecolor{Rblack}{RGB}{0,0,0}
	\definecolor{Rgrey}{RGB}{42,51,68}
	\definecolor{RdeepBlue}{RGB}{17,34,170}
	\definecolor{RmidBlue}{RGB}{32,88,220}
	\definecolor{RlightBlue}{RGB}{90,130,234}
	\definecolor{RuglyGreen}{RGB}{253,236,190}
	\definecolor{RyellowPea}{RGB}{253,219,91}
	\definecolor{RdeepYellow}{RGB}{250,185,53}
	\definecolor{Rsalmon}{RGB}{253,152,89}
	\definecolor{Rred}{RGB}{182,52,58}

	\definecolor{lightAvailability_grey}{RGB}{20,20,20}

%% Tikz
\usepackage{tikz}

%%%% Tab, list...
%% Best package for tables
\usepackage{booktabs}

%% List
\usepackage{enumerate}
\usepackage{enumitem}% http://ctan.org/pkg/enumitem

%% Nomenclature (containing three sections: Acronyms, greek symbols, roman symbols) https://tex.stackexchange.com/questions/452107/nomenclature-having-two-or-more-descriptions-and-units-for-a-symbol
\usepackage[intoc]{nomencl}

%% Change caption entries to whatever you want (also done in the ToC)
\addto\captionsenglish{ % Replace "english" with the language you use, if you use babel you might have to replace english by british or USenglish
	\renewcommand{\contentsname}{TABLE DES MATIÈRES}
	\renewcommand{\listtablename}{LISTE DES TABLEAUX}
	\renewcommand{\listfigurename}{LISTE DES FIGURES}
}

%%%% Links
\usepackage{url}
\usepackage[luatex, colorlinks=true, linkcolor=black, urlcolor=RdeepBlue, citecolor=RdeepBlue]{hyperref}

%%%% Bibliography and table of contents
%% Bibliography output using biblatex
\usepackage{csquotes}
\usepackage[style=apa, natbib=true, sorting=ynt]{biblatex}
% \usepackage[uniquelist=false, backend=biber, citestyle=authoryear-comp, sorting=nyt, sortcites=true, giveninits=true, hyperref, natbib, url=false, doi=false, eprint=false, isbn=false, maxbibnames=25, maxcitenames=2, mincitenames=1, uniquename=init, refsection=chapter]{biblatex}
\addbibresource{phd.bib}

%% Table of content (= ToC)
\usepackage[nottoc]{tocbibind}

%%%% Mathematics
\usepackage{amsthm}
\usepackage{amsmath}
	\allowdisplaybreaks % allows breaks between pages in formula
\usepackage{amssymb}
\usepackage{bbold}
\usepackage{dsfont}
\usepackage{mathrsfs}
\usepackage{bm}
\usepackage{xfrac}
\usepackage{etoolbox} % For renumbering
\usepackage{thmbox} % For "theorem" definitions
\usepackage{gensymb}
\usepackage{siunitx}
	\sisetup{
		inter-unit-product=\ensuremath{{}\cdot{}},
		per-mode=symbol
	}

\DeclareMathOperator{\logit}{logit}

%%%% Line numbers --conflict with amsthm-- pdf http://texblog.org/2012/02/08/adding-line-numbers-to-documents/
\usepackage{lineno}
	\linenumbers

%%%%%%%%%%%%%%%%%%%%%   OTHERS   %%%%%%%%%%%%%%%%%%%%%
% http://tex.stackexchange.com/questions/13304/which-package-version-am-i-using
\listfiles % Add \listfiles to your preamble and then look at the .log file. This will tell you the current version of all the packages loaded

%%%%%%%%%%%%%%%%%%%%%   NOMENCLATURE CODING   %%%%%%%%%%%%%%%%%%%%%
\makenomenclature
\renewcommand{\nomname}{LISTE DES ABBRÉVIATIONS}

\ExplSyntaxOn
\RenewDocumentCommand\nomgroup{m}
{
	\item[\textbf{\textcolor{RdeepBlue}{\choosenomgroup{#1}}}]
}
\NewDocumentCommand{\choosenomgroup}{m}
{
	\str_case:nn { #1 }
	{
		{A}{Acronymes}
		{L}{Symboles~latins}
		{G}{Symboles~grecs}
	}
}
\ExplSyntaxOff
\renewcommand*{\nompreamble}{\markboth{\nomname}{\nomname}}

\newcommand{\nomunit}[1]{%
	\renewcommand{\nomentryend}{\hspace*{\fill}\si{#1}}%
}

%%%%%%%%%%%%%%%%%%   HYPHENATION    %%%%%%%%%%%%%%%%%%
%%%% If latex has troubles to split a word, you can teach it here
% \hyphenation{cal-cu-lus}

%%%%%%%%%%%%%%%%%%%%%%%%%%%%%%%%%%%%%%%%%%%%%%%%%%%%%%%%%%%%%%%%%%%%
%%%%%%%%%%%%%%%%%   NEW COMMANDS USER'S SPECIFIC   %%%%%%%%%%%%%%%%%
%%%%%%%%%%%%%%%%%%%%%%%%%%%%%%%%%%%%%%%%%%%%%%%%%%%%%%%%%%%%%%%%%%%%
%%%%%%%% Commands that will be apply to the whole Ph.D. For chapter specific commands, declare them in the chapters (and use renewcommand if a name is used twice)
%%%% Text commands
\newcommand {\ie}{\textit{i.e., }}
\newcommand {\eg}{\textit{e.g., }}
\newcommand {\cf}{\textit{cf} }
\providecommand{\keywords}[1]{\textbf{Mots-clefs:} #1}


%%%% Tikz
%% Libraries (specific tikz commands are in article*/main.tex files)
\usetikzlibrary{arrows, plotmarks, decorations.markings}
\tikzstyle{arrow} = [->,>=stealth,thick,rounded corners=4pt,line width=1pt]
\usetikzlibrary{shadows}
\usetikzlibrary{shadings}
\usetikzlibrary{positioning} % relative coodinate
\usetikzlibrary{tikzmark, calc} % calc, to calculate coordinate
\usetikzlibrary{decorations.pathmorphing} % to snake an arrow
\usetikzlibrary{shapes.arrows}
\usetikzlibrary{patterns}

%% Gradient shadings command
\makeatletter
\def\createshadingfromlist#1#2#3{%
	\pgfutil@tempcnta=0\relax
	\pgfutil@for\pgf@tmp:={#3}\do{\advance\pgfutil@tempcnta by1}%
	\ifnum\pgfutil@tempcnta=1\relax%
		\edef\pgf@spec{color(0)=(#3);color(100)=(#3)}%
	\else%
		\pgfmathparse{50/(\pgfutil@tempcnta-1)}\let\pgf@step=\pgfmathresult%
		\pgfutil@tempcntb=1\relax%
		\pgfutil@for\pgf@tmp:={#3}\do{%
			\ifnum\pgfutil@tempcntb=1\relax%
				\edef\pgf@spec{color(0)=(\pgf@tmp);color(25)=(\pgf@tmp)}%
			\else%
	        \ifnum\pgfutil@tempcntb<\pgfutil@tempcnta\relax%
				\pgfmathparse{25+\pgf@step/4+(\pgfutil@tempcntb-1)*\pgf@step}%
				\edef\pgf@spec{\pgf@spec;color(\pgfmathresult)=(\pgf@tmp)}%
	        \else%
	        	\edef\pgf@spec{\pgf@spec;color(75)=(\pgf@tmp);color(100)=(\pgf@tmp)}%
	        \fi%
	    \fi%
	    \advance\pgfutil@tempcntb by1\relax%
	    }%
	\fi%
	\csname pgfdeclare#2shading\endcsname{#1}{100}\pgf@spec%
}

%%%%%%%% Math commands
%%%% Maths
\newcommand {\s}{{s}^{*}}
\newcommand {\sst}{\tilde{s}^{*}} % s* stable d'où le st
\newcommand {\N}{\tilde{N}}
\newcommand {\A}{\mathscr{A}}
\newcommand {\K}{\mathcal{K}}
\renewcommand{\S}{\mathscr{S}}
\newcommand{\R}{\mathds{R}}
\newcommand{\Prob}{\mathds{P}}
\newcommand{\F}{\mathcal{F}}

%%%% Theorem style
\newtheoremstyle{theo}{\topsep}{\topsep}{\itshape}{}{\bfseries}{.}{\newline}{\thmname{#1} \thmnumber{#2} \thmnote{~: \textit{#3}}}
\theoremstyle{theo}
\newtheorem{rem}{Remark}[section]
\newtheorem{defi}{Definition}[section]
\newtheorem{assum}{Assumption}[section]
\newtheorem{nota}{Notation}[section]

%%%%%%%%%%%%%%%%%%%%%   BEGIN DOCUMENT   %%%%%%%%%%%%%%%%%%%%%
\begin{document}
%%%% Starts by a blank page
\newpage
\thispagestyle{empty}
\mbox{}

%%%% Title page
\begin{titlepage}
	\begin{center}
		\textsc{Your title small caps} \\
	\vspace{2cm}
	par \\
	\vspace{2cm}
	Amaël Le Squin
	\vspace{1cm}
	Université de Sherbrooke \\
	\vspace{1cm}
	Thèse présentée au Département de biologie en vue \\
	de l'obtention du grade de docteur ès sciences (Ph.D.)
	\vfill
	\textsc{faculté des sciences} \\
	\textsc{université de sherbrooke} \\
	\vspace{2cm}
	Sherbrooke, Québec, Canada, \begin{french} \monthyeardate\today \end{french}
	\end{center}	
\end{titlepage}

%%%% Jury page (and not Jimmy page...)
\thispagestyle{empty}
\begin{center}
\begin{french}
	Le \today \\
	\vspace{1cm}
	\textit{Le jury a accepté la thèse d'Amaël Le Squin dans sa version finale} \\
	\vspace{1cm}
	Membre du jury \\
	\vspace{1cm}
	Professeur Dominique Gravel \\
	Directeur de recherche \\
	Département de biologie \\
	Université de Sherbrooke \\
	\vspace{1cm}
	Professeure Isabelle Boulangeat \\
	Codirectrice de recherche \\
	IRSTEA \\
	\vspace{1cm}
	Professeur Prénom et nom \\
	Évaluatrice ou Évaluateur externe \\
	Département nom \\
	Nom de l'institution \\
	\vspace{1cm}
	Professeure Prénom et nom \\
	Évaluatrice ou Évaluateur interne \\
	Département nom \\
	Nom de l'institution \\
	\vspace{1cm}
	Professeure Prénom et nom \\
	Président-rapporteur \\
	Département de biologie \\
	Université de Sherbrooke
\end{french}
\end{center}

%%%% From abstract (included) to Introduction (excluded)
%% Start roman numbering
\pagenumbering{roman}
\input{abstract_fr}

\input{abstract_en}

\input{acknowledgements}

%% Table of contents
\tableofcontents

%% Nomencalture
\printnomenclature

%% List of tables
\listoftables

%% List of figures
\listoffigures

\newpage
%%%% Main body
%% Start arabic numbering
\pagenumbering{arabic}

%% Input chapters
\input{./introduction}
\input{./article1/main}
% \input{./article3/main}

\nomenclature[a]{TPS}{Thermal protection System}
\nomenclature[L]{$v$}{Fluid velocity\nomunit{\metre\per\second}}
\nomenclature[G]{$\tau$}{Longitude \nomunit{\radian}}
\nomenclature[g]{$\tau$}{Dimensional time \nomunit{\second}}
\nomenclature[g]{$\delta$}{Geocentric Latitude \nomunit{\radian}}
\nomenclature[g]{$\delta$}{Differential Operator}
\nomenclature[g]{$\delta$}{Radius of Trust Region \nomunit{\meter}}
\nomenclature[g]{$\delta^*$}{Geodetic Latitude.  \nomunit{\radian}}

%%%% Ends by a blank page, congrats!
\newpage
\thispagestyle{empty}
\mbox{}

\end{document}
% 
\documentclass[titlepage,oneside,letterpaper,openright,12pt]{report}

%%%%%%%%%%%%%%%%%%%    COMMENTS    %%%%%%%%%%%%%%%%%%%
% 1. To stop numbering the lines, deactivate \linenumbers (activated after loading the package lineno)
% 
% 2. To make the index, you need to compile first with lualatex and then use makeindex. Example:
%		lualatex main.tex
%		makeindex main.nlo -s nomencl.ist -o main.nls
%		lualatex main.tex
% 
% 3. For the bibliography, use the command biber (as I am using the package biblatex, much more modern than bibtex).
%		To use multiple bibliography, use the refsection environment (3.13.3 Multiple Bibliographies in biblatex doc Version 3.13).
%		Do not use the chapterbib package with biblatex, there is refsection environment for that!
%		Each article will need to use
%		\begin{refsection}
%			Intro, Methods, Results, Discussion
%			\printbibliography[heading=subbibnumbered]
%		\end{refsection}
% 
% 4. I defined two languages in the document, French and British english (which is the main language), use \begin{french} ... \end{french} to switch to French in a paragraph
% 
% 5. To avoid multi-defined labels, I recommend to apply this command line in your shell for each tex file of the folders ./article1/, ./article2/, ...:
% 		sed -i.tmp 's/\\label{/\\label{WHATEVER::/g' *.tex
% 			This will change all the labels by adding WHATEVER before. Example (adding article1 for each tex file in ./article1/ folder):
% 			\label{sec::intro} in ./article1/chap1.tex becomes \label{article1::sec::intro}
%		sed -i.tmp 's/\\ref{/\\ref{WHATEVER::/g' *.tex
%			This will change all the refs by adding WHATEVER before. Example (adding article1 for each tex file in ./article1/ folder):
% 			\ref{sec::intro} in ./article1/chap1.tex becomes \ref{article1::sec::intro}
% 		sed -i.tmp 's/\\eqref{/\\eqref{WHATEVER::/g' *.tex
%			This will change all the eqrefs by adding WHATEVER before. Example (adding article1 for each tex file in ./article1/ folder):
% 			\eqref{eq::R0_equation} in ./article1/chap1.tex becomes \eqref{article1::eq::R0_equation}
% 
%	For safety reasons, the original file will be kept in a temporary file '.tmp'. Example:
%		sed -i.tmp 'Your instructions here' introduction.tex will create two files:
% 			introduction.tex which should be the one to keep (if no mistakes in the sed instructions)
% 			introduction.tex.tmp which is the original file (that you can dump if sed behaved the way you expected)
% 	For an introduction to the power of sed and regular expressions: https://www.grymoire.com/Unix/Sed.html#uh-1
% 
% 6. I defined some colours in this document. Except RdeepBlue, none of them is used in this template and can be deleted.
% 	If you delete RdeepBlue, do not forget to adapt all the commands using this colour

%%%%%%%%%%%%%%%%%%%    PACKAGES    %%%%%%%%%%%%%%%%%%%
%%%% font
\usepackage{fontspec}
\usepackage{marvosym}

%%%% Language
\usepackage{polyglossia}
	\setdefaultlanguage[variant=british]{english}
	\setotherlanguages{french}

%%%% Date
\usepackage{datetime}
	\newdateformat{monthyeardate}{\monthname[\THEMONTH] \THEYEAR}

%%%% marges, space, titles...
\usepackage[top=3.75cm, bottom=2.5cm, left=3cm, right=2.5cm]{geometry}
% \usepackage{setspace}
\setlength{\parindent}{0ex}   % indentation au début de chaque paragraphe
\setlength{\parskip}{3ex plus 0.5ex minus 0.2ex} % espace vertical entre paragraphes

\usepackage{titlesec}
\newcommand\chapterstring{CHAPITRE}
\titleformat{\chapter}[display]{\vspace{-8em}\center\bfseries}{\chapterstring~\thechapter}{12pt}{}[\vspace{1.5ex}]
\titleformat{\section}{\vspace{0mm plus 2cm}\addpenalty{-1000}\bfseries\sffamily}{\thesection}{1em}{}
\titleformat{\subsection}{\addpenalty{-500}\bfseries\sffamily}{\thesubsection}{1em}{}
\titleformat{\subsubsection}{\penalty-500\vspace{0pt plus 2pt}{}\bfseries\sffamily}{\thesubsubsection}{}{}[\vspace{-14pt}]

\usepackage{titletoc}
	\titlecontents{chapter}[0pt]{}{\bfseries\chapterstring~\thecontentslabel~}{\bfseries}{\hspace{1em plus 1fill}\contentspage}

%%%% Appendix
\usepackage{appendix}
%% Start of subappendices environment
\AtBeginEnvironment{subappendices}{%
	\chapter*{Appendix}
	\addcontentsline{toc}{chapter}{Appendices}
	\counterwithin{figure}{section}
	\counterwithin{table}{section}
}

%% End of subappendices environment
\AtEndEnvironment{subappendices}{%
	\counterwithout{figure}{section}
	\counterwithout{table}{section}
}

%%%% Graphics
%% To compile with lualatex
\usepackage[luatex]{graphicx}

%% Graphics' path, but I prefer to use absolute path for safety
% \graphicspath{{./article1/}{./article2/}{article3/}}

%% Numbering figures in sections rather than continuously
\usepackage{chngcntr}
\counterwithin{figure}{section}

%% Put figures within their declaration section 
\usepackage[section]{placeins}

%% To create multi-figures
\usepackage{caption}
\usepackage{subcaption}

%% Colours
\usepackage[dvipsnames, svgnames]{xcolor}
	\definecolor{Rblack}{RGB}{0,0,0}
	\definecolor{Rgrey}{RGB}{42,51,68}
	\definecolor{RdeepBlue}{RGB}{17,34,170}
	\definecolor{RmidBlue}{RGB}{32,88,220}
	\definecolor{RlightBlue}{RGB}{90,130,234}
	\definecolor{RuglyGreen}{RGB}{253,236,190}
	\definecolor{RyellowPea}{RGB}{253,219,91}
	\definecolor{RdeepYellow}{RGB}{250,185,53}
	\definecolor{Rsalmon}{RGB}{253,152,89}
	\definecolor{Rred}{RGB}{182,52,58}

	\definecolor{lightAvailability_grey}{RGB}{20,20,20}

%% Tikz
\usepackage{tikz}

%%%% Tab, list...
%% Best package for tables
\usepackage{booktabs}

%% List
\usepackage{enumerate}
\usepackage{enumitem}% http://ctan.org/pkg/enumitem

%% Nomenclature (containing three sections: Acronyms, greek symbols, roman symbols) https://tex.stackexchange.com/questions/452107/nomenclature-having-two-or-more-descriptions-and-units-for-a-symbol
\usepackage[intoc]{nomencl}

%% Change caption entries to whatever you want (also done in the ToC)
\addto\captionsenglish{ % Replace "english" with the language you use, if you use babel you might have to replace english by british or USenglish
	\renewcommand{\contentsname}{TABLE DES MATIÈRES}
	\renewcommand{\listtablename}{LISTE DES TABLEAUX}
	\renewcommand{\listfigurename}{LISTE DES FIGURES}
}

%%%% Links
\usepackage{url}
\usepackage[luatex, colorlinks=true, linkcolor=black, urlcolor=RdeepBlue, citecolor=RdeepBlue]{hyperref}

%%%% Bibliography and table of contents
%% Bibliography output using biblatex
\usepackage{csquotes}
\usepackage[style=apa, natbib=true, sorting=ynt]{biblatex}
% \usepackage[uniquelist=false, backend=biber, citestyle=authoryear-comp, sorting=nyt, sortcites=true, giveninits=true, hyperref, natbib, url=false, doi=false, eprint=false, isbn=false, maxbibnames=25, maxcitenames=2, mincitenames=1, uniquename=init, refsection=chapter]{biblatex}
\addbibresource{phd.bib}

%% Table of content (= ToC)
\usepackage[nottoc]{tocbibind}

%%%% Mathematics
\usepackage{amsthm}
\usepackage{amsmath}
	\allowdisplaybreaks % allows breaks between pages in formula
\usepackage{amssymb}
\usepackage{bbold}
\usepackage{dsfont}
\usepackage{mathrsfs}
\usepackage{bm}
\usepackage{xfrac}
\usepackage{etoolbox} % For renumbering
\usepackage{thmbox} % For "theorem" definitions
\usepackage{gensymb}
\usepackage{siunitx}
	\sisetup{
		inter-unit-product=\ensuremath{{}\cdot{}},
		per-mode=symbol
	}

\DeclareMathOperator{\logit}{logit}

%%%% Line numbers --conflict with amsthm-- pdf http://texblog.org/2012/02/08/adding-line-numbers-to-documents/
\usepackage{lineno}
	\linenumbers

%%%%%%%%%%%%%%%%%%%%%   OTHERS   %%%%%%%%%%%%%%%%%%%%%
% http://tex.stackexchange.com/questions/13304/which-package-version-am-i-using
\listfiles % Add \listfiles to your preamble and then look at the .log file. This will tell you the current version of all the packages loaded

%%%%%%%%%%%%%%%%%%%%%   NOMENCLATURE CODING   %%%%%%%%%%%%%%%%%%%%%
\makenomenclature
\renewcommand{\nomname}{LISTE DES ABBRÉVIATIONS}

\ExplSyntaxOn
\RenewDocumentCommand\nomgroup{m}
{
	\item[\textbf{\textcolor{RdeepBlue}{\choosenomgroup{#1}}}]
}
\NewDocumentCommand{\choosenomgroup}{m}
{
	\str_case:nn { #1 }
	{
		{A}{Acronymes}
		{L}{Symboles~latins}
		{G}{Symboles~grecs}
	}
}
\ExplSyntaxOff
\renewcommand*{\nompreamble}{\markboth{\nomname}{\nomname}}

\newcommand{\nomunit}[1]{%
	\renewcommand{\nomentryend}{\hspace*{\fill}\si{#1}}%
}

%%%%%%%%%%%%%%%%%%   HYPHENATION    %%%%%%%%%%%%%%%%%%
%%%% If latex has troubles to split a word, you can teach it here
% \hyphenation{cal-cu-lus}

%%%%%%%%%%%%%%%%%%%%%%%%%%%%%%%%%%%%%%%%%%%%%%%%%%%%%%%%%%%%%%%%%%%%
%%%%%%%%%%%%%%%%%   NEW COMMANDS USER'S SPECIFIC   %%%%%%%%%%%%%%%%%
%%%%%%%%%%%%%%%%%%%%%%%%%%%%%%%%%%%%%%%%%%%%%%%%%%%%%%%%%%%%%%%%%%%%
%%%%%%%% Commands that will be apply to the whole Ph.D. For chapter specific commands, declare them in the chapters (and use renewcommand if a name is used twice)
%%%% Text commands
\newcommand {\ie}{\textit{i.e., }}
\newcommand {\eg}{\textit{e.g., }}
\newcommand {\cf}{\textit{cf} }
\providecommand{\keywords}[1]{\textbf{Mots-clefs:} #1}


%%%% Tikz
%% Libraries (specific tikz commands are in article*/main.tex files)
\usetikzlibrary{arrows, plotmarks, decorations.markings}
\tikzstyle{arrow} = [->,>=stealth,thick,rounded corners=4pt,line width=1pt]
\usetikzlibrary{shadows}
\usetikzlibrary{shadings}
\usetikzlibrary{positioning} % relative coodinate
\usetikzlibrary{tikzmark, calc} % calc, to calculate coordinate
\usetikzlibrary{decorations.pathmorphing} % to snake an arrow
\usetikzlibrary{shapes.arrows}
\usetikzlibrary{patterns}

%% Gradient shadings command
\makeatletter
\def\createshadingfromlist#1#2#3{%
	\pgfutil@tempcnta=0\relax
	\pgfutil@for\pgf@tmp:={#3}\do{\advance\pgfutil@tempcnta by1}%
	\ifnum\pgfutil@tempcnta=1\relax%
		\edef\pgf@spec{color(0)=(#3);color(100)=(#3)}%
	\else%
		\pgfmathparse{50/(\pgfutil@tempcnta-1)}\let\pgf@step=\pgfmathresult%
		\pgfutil@tempcntb=1\relax%
		\pgfutil@for\pgf@tmp:={#3}\do{%
			\ifnum\pgfutil@tempcntb=1\relax%
				\edef\pgf@spec{color(0)=(\pgf@tmp);color(25)=(\pgf@tmp)}%
			\else%
	        \ifnum\pgfutil@tempcntb<\pgfutil@tempcnta\relax%
				\pgfmathparse{25+\pgf@step/4+(\pgfutil@tempcntb-1)*\pgf@step}%
				\edef\pgf@spec{\pgf@spec;color(\pgfmathresult)=(\pgf@tmp)}%
	        \else%
	        	\edef\pgf@spec{\pgf@spec;color(75)=(\pgf@tmp);color(100)=(\pgf@tmp)}%
	        \fi%
	    \fi%
	    \advance\pgfutil@tempcntb by1\relax%
	    }%
	\fi%
	\csname pgfdeclare#2shading\endcsname{#1}{100}\pgf@spec%
}

%%%%%%%% Math commands
%%%% Maths
\newcommand {\s}{{s}^{*}}
\newcommand {\sst}{\tilde{s}^{*}} % s* stable d'où le st
\newcommand {\N}{\tilde{N}}
\newcommand {\A}{\mathscr{A}}
\newcommand {\K}{\mathcal{K}}
\renewcommand{\S}{\mathscr{S}}
\newcommand{\R}{\mathds{R}}
\newcommand{\Prob}{\mathds{P}}
\newcommand{\F}{\mathcal{F}}

%%%% Theorem style
\newtheoremstyle{theo}{\topsep}{\topsep}{\itshape}{}{\bfseries}{.}{\newline}{\thmname{#1} \thmnumber{#2} \thmnote{~: \textit{#3}}}
\theoremstyle{theo}
\newtheorem{rem}{Remark}[section]
\newtheorem{defi}{Definition}[section]
\newtheorem{assum}{Assumption}[section]
\newtheorem{nota}{Notation}[section]

%%%%%%%%%%%%%%%%%%%%%   BEGIN DOCUMENT   %%%%%%%%%%%%%%%%%%%%%
\begin{document}
%%%% Starts by a blank page
\newpage
\thispagestyle{empty}
\mbox{}

%%%% Title page
\begin{titlepage}
	\begin{center}
		\textsc{Your title small caps} \\
	\vspace{2cm}
	par \\
	\vspace{2cm}
	Amaël Le Squin
	\vspace{1cm}
	Université de Sherbrooke \\
	\vspace{1cm}
	Thèse présentée au Département de biologie en vue \\
	de l'obtention du grade de docteur ès sciences (Ph.D.)
	\vfill
	\textsc{faculté des sciences} \\
	\textsc{université de sherbrooke} \\
	\vspace{2cm}
	Sherbrooke, Québec, Canada, \begin{french} \monthyeardate\today \end{french}
	\end{center}	
\end{titlepage}

%%%% Jury page (and not Jimmy page...)
\thispagestyle{empty}
\begin{center}
\begin{french}
	Le \today \\
	\vspace{1cm}
	\textit{Le jury a accepté la thèse d'Amaël Le Squin dans sa version finale} \\
	\vspace{1cm}
	Membre du jury \\
	\vspace{1cm}
	Professeur Dominique Gravel \\
	Directeur de recherche \\
	Département de biologie \\
	Université de Sherbrooke \\
	\vspace{1cm}
	Professeure Isabelle Boulangeat \\
	Codirectrice de recherche \\
	IRSTEA \\
	\vspace{1cm}
	Professeur Prénom et nom \\
	Évaluatrice ou Évaluateur externe \\
	Département nom \\
	Nom de l'institution \\
	\vspace{1cm}
	Professeure Prénom et nom \\
	Évaluatrice ou Évaluateur interne \\
	Département nom \\
	Nom de l'institution \\
	\vspace{1cm}
	Professeure Prénom et nom \\
	Président-rapporteur \\
	Département de biologie \\
	Université de Sherbrooke
\end{french}
\end{center}

%%%% From abstract (included) to Introduction (excluded)
%% Start roman numbering
\pagenumbering{roman}
\input{abstract_fr}

\input{abstract_en}

\input{acknowledgements}

%% Table of contents
\tableofcontents

%% Nomencalture
\printnomenclature

%% List of tables
\listoftables

%% List of figures
\listoffigures

\newpage
%%%% Main body
%% Start arabic numbering
\pagenumbering{arabic}

%% Input chapters
\input{./introduction}
\input{./article1/main}
% \input{./article3/main}

\nomenclature[a]{TPS}{Thermal protection System}
\nomenclature[L]{$v$}{Fluid velocity\nomunit{\metre\per\second}}
\nomenclature[G]{$\tau$}{Longitude \nomunit{\radian}}
\nomenclature[g]{$\tau$}{Dimensional time \nomunit{\second}}
\nomenclature[g]{$\delta$}{Geocentric Latitude \nomunit{\radian}}
\nomenclature[g]{$\delta$}{Differential Operator}
\nomenclature[g]{$\delta$}{Radius of Trust Region \nomunit{\meter}}
\nomenclature[g]{$\delta^*$}{Geodetic Latitude.  \nomunit{\radian}}

%%%% Ends by a blank page, congrats!
\newpage
\thispagestyle{empty}
\mbox{}

\end{document}

\nomenclature[a]{TPS}{Thermal protection System}
\nomenclature[L]{$v$}{Fluid velocity\nomunit{\metre\per\second}}
\nomenclature[G]{$\tau$}{Longitude \nomunit{\radian}}
\nomenclature[g]{$\tau$}{Dimensional time \nomunit{\second}}
\nomenclature[g]{$\delta$}{Geocentric Latitude \nomunit{\radian}}
\nomenclature[g]{$\delta$}{Differential Operator}
\nomenclature[g]{$\delta$}{Radius of Trust Region \nomunit{\meter}}
\nomenclature[g]{$\delta^*$}{Geodetic Latitude.  \nomunit{\radian}}

%%%% Ends by a blank page, congrats!
\newpage
\thispagestyle{empty}
\mbox{}

\end{document}
% 
\documentclass[titlepage,oneside,letterpaper,openright,12pt]{report}

%%%%%%%%%%%%%%%%%%%    COMMENTS    %%%%%%%%%%%%%%%%%%%
% 1. To stop numbering the lines, deactivate \linenumbers (activated after loading the package lineno)
% 
% 2. To make the index, you need to compile first with lualatex and then use makeindex. Example:
%		lualatex main.tex
%		makeindex main.nlo -s nomencl.ist -o main.nls
%		lualatex main.tex
% 
% 3. For the bibliography, use the command biber (as I am using the package biblatex, much more modern than bibtex).
%		To use multiple bibliography, use the refsection environment (3.13.3 Multiple Bibliographies in biblatex doc Version 3.13).
%		Do not use the chapterbib package with biblatex, there is refsection environment for that!
%		Each article will need to use
%		\begin{refsection}
%			Intro, Methods, Results, Discussion
%			\printbibliography[heading=subbibnumbered]
%		\end{refsection}
% 
% 4. I defined two languages in the document, French and British english (which is the main language), use \begin{french} ... \end{french} to switch to French in a paragraph
% 
% 5. To avoid multi-defined labels, I recommend to apply this command line in your shell for each tex file of the folders ./article1/, ./article2/, ...:
% 		sed -i.tmp 's/\\label{/\\label{WHATEVER::/g' *.tex
% 			This will change all the labels by adding WHATEVER before. Example (adding article1 for each tex file in ./article1/ folder):
% 			\label{sec::intro} in ./article1/chap1.tex becomes \label{article1::sec::intro}
%		sed -i.tmp 's/\\ref{/\\ref{WHATEVER::/g' *.tex
%			This will change all the refs by adding WHATEVER before. Example (adding article1 for each tex file in ./article1/ folder):
% 			\ref{sec::intro} in ./article1/chap1.tex becomes \ref{article1::sec::intro}
% 		sed -i.tmp 's/\\eqref{/\\eqref{WHATEVER::/g' *.tex
%			This will change all the eqrefs by adding WHATEVER before. Example (adding article1 for each tex file in ./article1/ folder):
% 			\eqref{eq::R0_equation} in ./article1/chap1.tex becomes \eqref{article1::eq::R0_equation}
% 
%	For safety reasons, the original file will be kept in a temporary file '.tmp'. Example:
%		sed -i.tmp 'Your instructions here' introduction.tex will create two files:
% 			introduction.tex which should be the one to keep (if no mistakes in the sed instructions)
% 			introduction.tex.tmp which is the original file (that you can dump if sed behaved the way you expected)
% 	For an introduction to the power of sed and regular expressions: https://www.grymoire.com/Unix/Sed.html#uh-1
% 
% 6. I defined some colours in this document. Except RdeepBlue, none of them is used in this template and can be deleted.
% 	If you delete RdeepBlue, do not forget to adapt all the commands using this colour

%%%%%%%%%%%%%%%%%%%    PACKAGES    %%%%%%%%%%%%%%%%%%%
%%%% font
\usepackage{fontspec}
\usepackage{marvosym}

%%%% Language
\usepackage{polyglossia}
	\setdefaultlanguage[variant=british]{english}
	\setotherlanguages{french}

%%%% Date
\usepackage{datetime}
	\newdateformat{monthyeardate}{\monthname[\THEMONTH] \THEYEAR}

%%%% marges, space, titles...
\usepackage[top=3.75cm, bottom=2.5cm, left=3cm, right=2.5cm]{geometry}
% \usepackage{setspace}
\setlength{\parindent}{0ex}   % indentation au début de chaque paragraphe
\setlength{\parskip}{3ex plus 0.5ex minus 0.2ex} % espace vertical entre paragraphes

\usepackage{titlesec}
\newcommand\chapterstring{CHAPITRE}
\titleformat{\chapter}[display]{\vspace{-8em}\center\bfseries}{\chapterstring~\thechapter}{12pt}{}[\vspace{1.5ex}]
\titleformat{\section}{\vspace{0mm plus 2cm}\addpenalty{-1000}\bfseries\sffamily}{\thesection}{1em}{}
\titleformat{\subsection}{\addpenalty{-500}\bfseries\sffamily}{\thesubsection}{1em}{}
\titleformat{\subsubsection}{\penalty-500\vspace{0pt plus 2pt}{}\bfseries\sffamily}{\thesubsubsection}{}{}[\vspace{-14pt}]

\usepackage{titletoc}
	\titlecontents{chapter}[0pt]{}{\bfseries\chapterstring~\thecontentslabel~}{\bfseries}{\hspace{1em plus 1fill}\contentspage}

%%%% Appendix
\usepackage{appendix}
%% Start of subappendices environment
\AtBeginEnvironment{subappendices}{%
	\chapter*{Appendix}
	\addcontentsline{toc}{chapter}{Appendices}
	\counterwithin{figure}{section}
	\counterwithin{table}{section}
}

%% End of subappendices environment
\AtEndEnvironment{subappendices}{%
	\counterwithout{figure}{section}
	\counterwithout{table}{section}
}

%%%% Graphics
%% To compile with lualatex
\usepackage[luatex]{graphicx}

%% Graphics' path, but I prefer to use absolute path for safety
% \graphicspath{{./article1/}{./article2/}{article3/}}

%% Numbering figures in sections rather than continuously
\usepackage{chngcntr}
\counterwithin{figure}{section}

%% Put figures within their declaration section 
\usepackage[section]{placeins}

%% To create multi-figures
\usepackage{caption}
\usepackage{subcaption}

%% Colours
\usepackage[dvipsnames, svgnames]{xcolor}
	\definecolor{Rblack}{RGB}{0,0,0}
	\definecolor{Rgrey}{RGB}{42,51,68}
	\definecolor{RdeepBlue}{RGB}{17,34,170}
	\definecolor{RmidBlue}{RGB}{32,88,220}
	\definecolor{RlightBlue}{RGB}{90,130,234}
	\definecolor{RuglyGreen}{RGB}{253,236,190}
	\definecolor{RyellowPea}{RGB}{253,219,91}
	\definecolor{RdeepYellow}{RGB}{250,185,53}
	\definecolor{Rsalmon}{RGB}{253,152,89}
	\definecolor{Rred}{RGB}{182,52,58}

	\definecolor{lightAvailability_grey}{RGB}{20,20,20}

%% Tikz
\usepackage{tikz}

%%%% Tab, list...
%% Best package for tables
\usepackage{booktabs}

%% List
\usepackage{enumerate}
\usepackage{enumitem}% http://ctan.org/pkg/enumitem

%% Nomenclature (containing three sections: Acronyms, greek symbols, roman symbols) https://tex.stackexchange.com/questions/452107/nomenclature-having-two-or-more-descriptions-and-units-for-a-symbol
\usepackage[intoc]{nomencl}

%% Change caption entries to whatever you want (also done in the ToC)
\addto\captionsenglish{ % Replace "english" with the language you use, if you use babel you might have to replace english by british or USenglish
	\renewcommand{\contentsname}{TABLE DES MATIÈRES}
	\renewcommand{\listtablename}{LISTE DES TABLEAUX}
	\renewcommand{\listfigurename}{LISTE DES FIGURES}
}

%%%% Links
\usepackage{url}
\usepackage[luatex, colorlinks=true, linkcolor=black, urlcolor=RdeepBlue, citecolor=RdeepBlue]{hyperref}

%%%% Bibliography and table of contents
%% Bibliography output using biblatex
\usepackage{csquotes}
\usepackage[style=apa, natbib=true, sorting=ynt]{biblatex}
% \usepackage[uniquelist=false, backend=biber, citestyle=authoryear-comp, sorting=nyt, sortcites=true, giveninits=true, hyperref, natbib, url=false, doi=false, eprint=false, isbn=false, maxbibnames=25, maxcitenames=2, mincitenames=1, uniquename=init, refsection=chapter]{biblatex}
\addbibresource{phd.bib}

%% Table of content (= ToC)
\usepackage[nottoc]{tocbibind}

%%%% Mathematics
\usepackage{amsthm}
\usepackage{amsmath}
	\allowdisplaybreaks % allows breaks between pages in formula
\usepackage{amssymb}
\usepackage{bbold}
\usepackage{dsfont}
\usepackage{mathrsfs}
\usepackage{bm}
\usepackage{xfrac}
\usepackage{etoolbox} % For renumbering
\usepackage{thmbox} % For "theorem" definitions
\usepackage{gensymb}
\usepackage{siunitx}
	\sisetup{
		inter-unit-product=\ensuremath{{}\cdot{}},
		per-mode=symbol
	}

\DeclareMathOperator{\logit}{logit}

%%%% Line numbers --conflict with amsthm-- pdf http://texblog.org/2012/02/08/adding-line-numbers-to-documents/
\usepackage{lineno}
	\linenumbers

%%%%%%%%%%%%%%%%%%%%%   OTHERS   %%%%%%%%%%%%%%%%%%%%%
% http://tex.stackexchange.com/questions/13304/which-package-version-am-i-using
\listfiles % Add \listfiles to your preamble and then look at the .log file. This will tell you the current version of all the packages loaded

%%%%%%%%%%%%%%%%%%%%%   NOMENCLATURE CODING   %%%%%%%%%%%%%%%%%%%%%
\makenomenclature
\renewcommand{\nomname}{LISTE DES ABBRÉVIATIONS}

\ExplSyntaxOn
\RenewDocumentCommand\nomgroup{m}
{
	\item[\textbf{\textcolor{RdeepBlue}{\choosenomgroup{#1}}}]
}
\NewDocumentCommand{\choosenomgroup}{m}
{
	\str_case:nn { #1 }
	{
		{A}{Acronymes}
		{L}{Symboles~latins}
		{G}{Symboles~grecs}
	}
}
\ExplSyntaxOff
\renewcommand*{\nompreamble}{\markboth{\nomname}{\nomname}}

\newcommand{\nomunit}[1]{%
	\renewcommand{\nomentryend}{\hspace*{\fill}\si{#1}}%
}

%%%%%%%%%%%%%%%%%%   HYPHENATION    %%%%%%%%%%%%%%%%%%
%%%% If latex has troubles to split a word, you can teach it here
% \hyphenation{cal-cu-lus}

%%%%%%%%%%%%%%%%%%%%%%%%%%%%%%%%%%%%%%%%%%%%%%%%%%%%%%%%%%%%%%%%%%%%
%%%%%%%%%%%%%%%%%   NEW COMMANDS USER'S SPECIFIC   %%%%%%%%%%%%%%%%%
%%%%%%%%%%%%%%%%%%%%%%%%%%%%%%%%%%%%%%%%%%%%%%%%%%%%%%%%%%%%%%%%%%%%
%%%%%%%% Commands that will be apply to the whole Ph.D. For chapter specific commands, declare them in the chapters (and use renewcommand if a name is used twice)
%%%% Text commands
\newcommand {\ie}{\textit{i.e., }}
\newcommand {\eg}{\textit{e.g., }}
\newcommand {\cf}{\textit{cf} }
\providecommand{\keywords}[1]{\textbf{Mots-clefs:} #1}


%%%% Tikz
%% Libraries (specific tikz commands are in article*/main.tex files)
\usetikzlibrary{arrows, plotmarks, decorations.markings}
\tikzstyle{arrow} = [->,>=stealth,thick,rounded corners=4pt,line width=1pt]
\usetikzlibrary{shadows}
\usetikzlibrary{shadings}
\usetikzlibrary{positioning} % relative coodinate
\usetikzlibrary{tikzmark, calc} % calc, to calculate coordinate
\usetikzlibrary{decorations.pathmorphing} % to snake an arrow
\usetikzlibrary{shapes.arrows}
\usetikzlibrary{patterns}

%% Gradient shadings command
\makeatletter
\def\createshadingfromlist#1#2#3{%
	\pgfutil@tempcnta=0\relax
	\pgfutil@for\pgf@tmp:={#3}\do{\advance\pgfutil@tempcnta by1}%
	\ifnum\pgfutil@tempcnta=1\relax%
		\edef\pgf@spec{color(0)=(#3);color(100)=(#3)}%
	\else%
		\pgfmathparse{50/(\pgfutil@tempcnta-1)}\let\pgf@step=\pgfmathresult%
		\pgfutil@tempcntb=1\relax%
		\pgfutil@for\pgf@tmp:={#3}\do{%
			\ifnum\pgfutil@tempcntb=1\relax%
				\edef\pgf@spec{color(0)=(\pgf@tmp);color(25)=(\pgf@tmp)}%
			\else%
	        \ifnum\pgfutil@tempcntb<\pgfutil@tempcnta\relax%
				\pgfmathparse{25+\pgf@step/4+(\pgfutil@tempcntb-1)*\pgf@step}%
				\edef\pgf@spec{\pgf@spec;color(\pgfmathresult)=(\pgf@tmp)}%
	        \else%
	        	\edef\pgf@spec{\pgf@spec;color(75)=(\pgf@tmp);color(100)=(\pgf@tmp)}%
	        \fi%
	    \fi%
	    \advance\pgfutil@tempcntb by1\relax%
	    }%
	\fi%
	\csname pgfdeclare#2shading\endcsname{#1}{100}\pgf@spec%
}

%%%%%%%% Math commands
%%%% Maths
\newcommand {\s}{{s}^{*}}
\newcommand {\sst}{\tilde{s}^{*}} % s* stable d'où le st
\newcommand {\N}{\tilde{N}}
\newcommand {\A}{\mathscr{A}}
\newcommand {\K}{\mathcal{K}}
\renewcommand{\S}{\mathscr{S}}
\newcommand{\R}{\mathds{R}}
\newcommand{\Prob}{\mathds{P}}
\newcommand{\F}{\mathcal{F}}

%%%% Theorem style
\newtheoremstyle{theo}{\topsep}{\topsep}{\itshape}{}{\bfseries}{.}{\newline}{\thmname{#1} \thmnumber{#2} \thmnote{~: \textit{#3}}}
\theoremstyle{theo}
\newtheorem{rem}{Remark}[section]
\newtheorem{defi}{Definition}[section]
\newtheorem{assum}{Assumption}[section]
\newtheorem{nota}{Notation}[section]

%%%%%%%%%%%%%%%%%%%%%   BEGIN DOCUMENT   %%%%%%%%%%%%%%%%%%%%%
\begin{document}
%%%% Starts by a blank page
\newpage
\thispagestyle{empty}
\mbox{}

%%%% Title page
\begin{titlepage}
	\begin{center}
		\textsc{Your title small caps} \\
	\vspace{2cm}
	par \\
	\vspace{2cm}
	Amaël Le Squin
	\vspace{1cm}
	Université de Sherbrooke \\
	\vspace{1cm}
	Thèse présentée au Département de biologie en vue \\
	de l'obtention du grade de docteur ès sciences (Ph.D.)
	\vfill
	\textsc{faculté des sciences} \\
	\textsc{université de sherbrooke} \\
	\vspace{2cm}
	Sherbrooke, Québec, Canada, \begin{french} \monthyeardate\today \end{french}
	\end{center}	
\end{titlepage}

%%%% Jury page (and not Jimmy page...)
\thispagestyle{empty}
\begin{center}
\begin{french}
	Le \today \\
	\vspace{1cm}
	\textit{Le jury a accepté la thèse d'Amaël Le Squin dans sa version finale} \\
	\vspace{1cm}
	Membre du jury \\
	\vspace{1cm}
	Professeur Dominique Gravel \\
	Directeur de recherche \\
	Département de biologie \\
	Université de Sherbrooke \\
	\vspace{1cm}
	Professeure Isabelle Boulangeat \\
	Codirectrice de recherche \\
	IRSTEA \\
	\vspace{1cm}
	Professeur Prénom et nom \\
	Évaluatrice ou Évaluateur externe \\
	Département nom \\
	Nom de l'institution \\
	\vspace{1cm}
	Professeure Prénom et nom \\
	Évaluatrice ou Évaluateur interne \\
	Département nom \\
	Nom de l'institution \\
	\vspace{1cm}
	Professeure Prénom et nom \\
	Président-rapporteur \\
	Département de biologie \\
	Université de Sherbrooke
\end{french}
\end{center}

%%%% From abstract (included) to Introduction (excluded)
%% Start roman numbering
\pagenumbering{roman}
\input{abstract_fr}

\input{abstract_en}

\input{acknowledgements}

%% Table of contents
\tableofcontents

%% Nomencalture
\printnomenclature

%% List of tables
\listoftables

%% List of figures
\listoffigures

\newpage
%%%% Main body
%% Start arabic numbering
\pagenumbering{arabic}

%% Input chapters
\input{./introduction}

\documentclass[titlepage,oneside,letterpaper,openright,12pt]{report}

%%%%%%%%%%%%%%%%%%%    COMMENTS    %%%%%%%%%%%%%%%%%%%
% 1. To stop numbering the lines, deactivate \linenumbers (activated after loading the package lineno)
% 
% 2. To make the index, you need to compile first with lualatex and then use makeindex. Example:
%		lualatex main.tex
%		makeindex main.nlo -s nomencl.ist -o main.nls
%		lualatex main.tex
% 
% 3. For the bibliography, use the command biber (as I am using the package biblatex, much more modern than bibtex).
%		To use multiple bibliography, use the refsection environment (3.13.3 Multiple Bibliographies in biblatex doc Version 3.13).
%		Do not use the chapterbib package with biblatex, there is refsection environment for that!
%		Each article will need to use
%		\begin{refsection}
%			Intro, Methods, Results, Discussion
%			\printbibliography[heading=subbibnumbered]
%		\end{refsection}
% 
% 4. I defined two languages in the document, French and British english (which is the main language), use \begin{french} ... \end{french} to switch to French in a paragraph
% 
% 5. To avoid multi-defined labels, I recommend to apply this command line in your shell for each tex file of the folders ./article1/, ./article2/, ...:
% 		sed -i.tmp 's/\\label{/\\label{WHATEVER::/g' *.tex
% 			This will change all the labels by adding WHATEVER before. Example (adding article1 for each tex file in ./article1/ folder):
% 			\label{sec::intro} in ./article1/chap1.tex becomes \label{article1::sec::intro}
%		sed -i.tmp 's/\\ref{/\\ref{WHATEVER::/g' *.tex
%			This will change all the refs by adding WHATEVER before. Example (adding article1 for each tex file in ./article1/ folder):
% 			\ref{sec::intro} in ./article1/chap1.tex becomes \ref{article1::sec::intro}
% 		sed -i.tmp 's/\\eqref{/\\eqref{WHATEVER::/g' *.tex
%			This will change all the eqrefs by adding WHATEVER before. Example (adding article1 for each tex file in ./article1/ folder):
% 			\eqref{eq::R0_equation} in ./article1/chap1.tex becomes \eqref{article1::eq::R0_equation}
% 
%	For safety reasons, the original file will be kept in a temporary file '.tmp'. Example:
%		sed -i.tmp 'Your instructions here' introduction.tex will create two files:
% 			introduction.tex which should be the one to keep (if no mistakes in the sed instructions)
% 			introduction.tex.tmp which is the original file (that you can dump if sed behaved the way you expected)
% 	For an introduction to the power of sed and regular expressions: https://www.grymoire.com/Unix/Sed.html#uh-1
% 
% 6. I defined some colours in this document. Except RdeepBlue, none of them is used in this template and can be deleted.
% 	If you delete RdeepBlue, do not forget to adapt all the commands using this colour

%%%%%%%%%%%%%%%%%%%    PACKAGES    %%%%%%%%%%%%%%%%%%%
%%%% font
\usepackage{fontspec}
\usepackage{marvosym}

%%%% Language
\usepackage{polyglossia}
	\setdefaultlanguage[variant=british]{english}
	\setotherlanguages{french}

%%%% Date
\usepackage{datetime}
	\newdateformat{monthyeardate}{\monthname[\THEMONTH] \THEYEAR}

%%%% marges, space, titles...
\usepackage[top=3.75cm, bottom=2.5cm, left=3cm, right=2.5cm]{geometry}
% \usepackage{setspace}
\setlength{\parindent}{0ex}   % indentation au début de chaque paragraphe
\setlength{\parskip}{3ex plus 0.5ex minus 0.2ex} % espace vertical entre paragraphes

\usepackage{titlesec}
\newcommand\chapterstring{CHAPITRE}
\titleformat{\chapter}[display]{\vspace{-8em}\center\bfseries}{\chapterstring~\thechapter}{12pt}{}[\vspace{1.5ex}]
\titleformat{\section}{\vspace{0mm plus 2cm}\addpenalty{-1000}\bfseries\sffamily}{\thesection}{1em}{}
\titleformat{\subsection}{\addpenalty{-500}\bfseries\sffamily}{\thesubsection}{1em}{}
\titleformat{\subsubsection}{\penalty-500\vspace{0pt plus 2pt}{}\bfseries\sffamily}{\thesubsubsection}{}{}[\vspace{-14pt}]

\usepackage{titletoc}
	\titlecontents{chapter}[0pt]{}{\bfseries\chapterstring~\thecontentslabel~}{\bfseries}{\hspace{1em plus 1fill}\contentspage}

%%%% Appendix
\usepackage{appendix}
%% Start of subappendices environment
\AtBeginEnvironment{subappendices}{%
	\chapter*{Appendix}
	\addcontentsline{toc}{chapter}{Appendices}
	\counterwithin{figure}{section}
	\counterwithin{table}{section}
}

%% End of subappendices environment
\AtEndEnvironment{subappendices}{%
	\counterwithout{figure}{section}
	\counterwithout{table}{section}
}

%%%% Graphics
%% To compile with lualatex
\usepackage[luatex]{graphicx}

%% Graphics' path, but I prefer to use absolute path for safety
% \graphicspath{{./article1/}{./article2/}{article3/}}

%% Numbering figures in sections rather than continuously
\usepackage{chngcntr}
\counterwithin{figure}{section}

%% Put figures within their declaration section 
\usepackage[section]{placeins}

%% To create multi-figures
\usepackage{caption}
\usepackage{subcaption}

%% Colours
\usepackage[dvipsnames, svgnames]{xcolor}
	\definecolor{Rblack}{RGB}{0,0,0}
	\definecolor{Rgrey}{RGB}{42,51,68}
	\definecolor{RdeepBlue}{RGB}{17,34,170}
	\definecolor{RmidBlue}{RGB}{32,88,220}
	\definecolor{RlightBlue}{RGB}{90,130,234}
	\definecolor{RuglyGreen}{RGB}{253,236,190}
	\definecolor{RyellowPea}{RGB}{253,219,91}
	\definecolor{RdeepYellow}{RGB}{250,185,53}
	\definecolor{Rsalmon}{RGB}{253,152,89}
	\definecolor{Rred}{RGB}{182,52,58}

	\definecolor{lightAvailability_grey}{RGB}{20,20,20}

%% Tikz
\usepackage{tikz}

%%%% Tab, list...
%% Best package for tables
\usepackage{booktabs}

%% List
\usepackage{enumerate}
\usepackage{enumitem}% http://ctan.org/pkg/enumitem

%% Nomenclature (containing three sections: Acronyms, greek symbols, roman symbols) https://tex.stackexchange.com/questions/452107/nomenclature-having-two-or-more-descriptions-and-units-for-a-symbol
\usepackage[intoc]{nomencl}

%% Change caption entries to whatever you want (also done in the ToC)
\addto\captionsenglish{ % Replace "english" with the language you use, if you use babel you might have to replace english by british or USenglish
	\renewcommand{\contentsname}{TABLE DES MATIÈRES}
	\renewcommand{\listtablename}{LISTE DES TABLEAUX}
	\renewcommand{\listfigurename}{LISTE DES FIGURES}
}

%%%% Links
\usepackage{url}
\usepackage[luatex, colorlinks=true, linkcolor=black, urlcolor=RdeepBlue, citecolor=RdeepBlue]{hyperref}

%%%% Bibliography and table of contents
%% Bibliography output using biblatex
\usepackage{csquotes}
\usepackage[style=apa, natbib=true, sorting=ynt]{biblatex}
% \usepackage[uniquelist=false, backend=biber, citestyle=authoryear-comp, sorting=nyt, sortcites=true, giveninits=true, hyperref, natbib, url=false, doi=false, eprint=false, isbn=false, maxbibnames=25, maxcitenames=2, mincitenames=1, uniquename=init, refsection=chapter]{biblatex}
\addbibresource{phd.bib}

%% Table of content (= ToC)
\usepackage[nottoc]{tocbibind}

%%%% Mathematics
\usepackage{amsthm}
\usepackage{amsmath}
	\allowdisplaybreaks % allows breaks between pages in formula
\usepackage{amssymb}
\usepackage{bbold}
\usepackage{dsfont}
\usepackage{mathrsfs}
\usepackage{bm}
\usepackage{xfrac}
\usepackage{etoolbox} % For renumbering
\usepackage{thmbox} % For "theorem" definitions
\usepackage{gensymb}
\usepackage{siunitx}
	\sisetup{
		inter-unit-product=\ensuremath{{}\cdot{}},
		per-mode=symbol
	}

\DeclareMathOperator{\logit}{logit}

%%%% Line numbers --conflict with amsthm-- pdf http://texblog.org/2012/02/08/adding-line-numbers-to-documents/
\usepackage{lineno}
	\linenumbers

%%%%%%%%%%%%%%%%%%%%%   OTHERS   %%%%%%%%%%%%%%%%%%%%%
% http://tex.stackexchange.com/questions/13304/which-package-version-am-i-using
\listfiles % Add \listfiles to your preamble and then look at the .log file. This will tell you the current version of all the packages loaded

%%%%%%%%%%%%%%%%%%%%%   NOMENCLATURE CODING   %%%%%%%%%%%%%%%%%%%%%
\makenomenclature
\renewcommand{\nomname}{LISTE DES ABBRÉVIATIONS}

\ExplSyntaxOn
\RenewDocumentCommand\nomgroup{m}
{
	\item[\textbf{\textcolor{RdeepBlue}{\choosenomgroup{#1}}}]
}
\NewDocumentCommand{\choosenomgroup}{m}
{
	\str_case:nn { #1 }
	{
		{A}{Acronymes}
		{L}{Symboles~latins}
		{G}{Symboles~grecs}
	}
}
\ExplSyntaxOff
\renewcommand*{\nompreamble}{\markboth{\nomname}{\nomname}}

\newcommand{\nomunit}[1]{%
	\renewcommand{\nomentryend}{\hspace*{\fill}\si{#1}}%
}

%%%%%%%%%%%%%%%%%%   HYPHENATION    %%%%%%%%%%%%%%%%%%
%%%% If latex has troubles to split a word, you can teach it here
% \hyphenation{cal-cu-lus}

%%%%%%%%%%%%%%%%%%%%%%%%%%%%%%%%%%%%%%%%%%%%%%%%%%%%%%%%%%%%%%%%%%%%
%%%%%%%%%%%%%%%%%   NEW COMMANDS USER'S SPECIFIC   %%%%%%%%%%%%%%%%%
%%%%%%%%%%%%%%%%%%%%%%%%%%%%%%%%%%%%%%%%%%%%%%%%%%%%%%%%%%%%%%%%%%%%
%%%%%%%% Commands that will be apply to the whole Ph.D. For chapter specific commands, declare them in the chapters (and use renewcommand if a name is used twice)
%%%% Text commands
\newcommand {\ie}{\textit{i.e., }}
\newcommand {\eg}{\textit{e.g., }}
\newcommand {\cf}{\textit{cf} }
\providecommand{\keywords}[1]{\textbf{Mots-clefs:} #1}


%%%% Tikz
%% Libraries (specific tikz commands are in article*/main.tex files)
\usetikzlibrary{arrows, plotmarks, decorations.markings}
\tikzstyle{arrow} = [->,>=stealth,thick,rounded corners=4pt,line width=1pt]
\usetikzlibrary{shadows}
\usetikzlibrary{shadings}
\usetikzlibrary{positioning} % relative coodinate
\usetikzlibrary{tikzmark, calc} % calc, to calculate coordinate
\usetikzlibrary{decorations.pathmorphing} % to snake an arrow
\usetikzlibrary{shapes.arrows}
\usetikzlibrary{patterns}

%% Gradient shadings command
\makeatletter
\def\createshadingfromlist#1#2#3{%
	\pgfutil@tempcnta=0\relax
	\pgfutil@for\pgf@tmp:={#3}\do{\advance\pgfutil@tempcnta by1}%
	\ifnum\pgfutil@tempcnta=1\relax%
		\edef\pgf@spec{color(0)=(#3);color(100)=(#3)}%
	\else%
		\pgfmathparse{50/(\pgfutil@tempcnta-1)}\let\pgf@step=\pgfmathresult%
		\pgfutil@tempcntb=1\relax%
		\pgfutil@for\pgf@tmp:={#3}\do{%
			\ifnum\pgfutil@tempcntb=1\relax%
				\edef\pgf@spec{color(0)=(\pgf@tmp);color(25)=(\pgf@tmp)}%
			\else%
	        \ifnum\pgfutil@tempcntb<\pgfutil@tempcnta\relax%
				\pgfmathparse{25+\pgf@step/4+(\pgfutil@tempcntb-1)*\pgf@step}%
				\edef\pgf@spec{\pgf@spec;color(\pgfmathresult)=(\pgf@tmp)}%
	        \else%
	        	\edef\pgf@spec{\pgf@spec;color(75)=(\pgf@tmp);color(100)=(\pgf@tmp)}%
	        \fi%
	    \fi%
	    \advance\pgfutil@tempcntb by1\relax%
	    }%
	\fi%
	\csname pgfdeclare#2shading\endcsname{#1}{100}\pgf@spec%
}

%%%%%%%% Math commands
%%%% Maths
\newcommand {\s}{{s}^{*}}
\newcommand {\sst}{\tilde{s}^{*}} % s* stable d'où le st
\newcommand {\N}{\tilde{N}}
\newcommand {\A}{\mathscr{A}}
\newcommand {\K}{\mathcal{K}}
\renewcommand{\S}{\mathscr{S}}
\newcommand{\R}{\mathds{R}}
\newcommand{\Prob}{\mathds{P}}
\newcommand{\F}{\mathcal{F}}

%%%% Theorem style
\newtheoremstyle{theo}{\topsep}{\topsep}{\itshape}{}{\bfseries}{.}{\newline}{\thmname{#1} \thmnumber{#2} \thmnote{~: \textit{#3}}}
\theoremstyle{theo}
\newtheorem{rem}{Remark}[section]
\newtheorem{defi}{Definition}[section]
\newtheorem{assum}{Assumption}[section]
\newtheorem{nota}{Notation}[section]

%%%%%%%%%%%%%%%%%%%%%   BEGIN DOCUMENT   %%%%%%%%%%%%%%%%%%%%%
\begin{document}
%%%% Starts by a blank page
\newpage
\thispagestyle{empty}
\mbox{}

%%%% Title page
\begin{titlepage}
	\begin{center}
		\textsc{Your title small caps} \\
	\vspace{2cm}
	par \\
	\vspace{2cm}
	Amaël Le Squin
	\vspace{1cm}
	Université de Sherbrooke \\
	\vspace{1cm}
	Thèse présentée au Département de biologie en vue \\
	de l'obtention du grade de docteur ès sciences (Ph.D.)
	\vfill
	\textsc{faculté des sciences} \\
	\textsc{université de sherbrooke} \\
	\vspace{2cm}
	Sherbrooke, Québec, Canada, \begin{french} \monthyeardate\today \end{french}
	\end{center}	
\end{titlepage}

%%%% Jury page (and not Jimmy page...)
\thispagestyle{empty}
\begin{center}
\begin{french}
	Le \today \\
	\vspace{1cm}
	\textit{Le jury a accepté la thèse d'Amaël Le Squin dans sa version finale} \\
	\vspace{1cm}
	Membre du jury \\
	\vspace{1cm}
	Professeur Dominique Gravel \\
	Directeur de recherche \\
	Département de biologie \\
	Université de Sherbrooke \\
	\vspace{1cm}
	Professeure Isabelle Boulangeat \\
	Codirectrice de recherche \\
	IRSTEA \\
	\vspace{1cm}
	Professeur Prénom et nom \\
	Évaluatrice ou Évaluateur externe \\
	Département nom \\
	Nom de l'institution \\
	\vspace{1cm}
	Professeure Prénom et nom \\
	Évaluatrice ou Évaluateur interne \\
	Département nom \\
	Nom de l'institution \\
	\vspace{1cm}
	Professeure Prénom et nom \\
	Président-rapporteur \\
	Département de biologie \\
	Université de Sherbrooke
\end{french}
\end{center}

%%%% From abstract (included) to Introduction (excluded)
%% Start roman numbering
\pagenumbering{roman}
\input{abstract_fr}

\input{abstract_en}

\input{acknowledgements}

%% Table of contents
\tableofcontents

%% Nomencalture
\printnomenclature

%% List of tables
\listoftables

%% List of figures
\listoffigures

\newpage
%%%% Main body
%% Start arabic numbering
\pagenumbering{arabic}

%% Input chapters
\input{./introduction}
\input{./article1/main}
% \input{./article3/main}

\nomenclature[a]{TPS}{Thermal protection System}
\nomenclature[L]{$v$}{Fluid velocity\nomunit{\metre\per\second}}
\nomenclature[G]{$\tau$}{Longitude \nomunit{\radian}}
\nomenclature[g]{$\tau$}{Dimensional time \nomunit{\second}}
\nomenclature[g]{$\delta$}{Geocentric Latitude \nomunit{\radian}}
\nomenclature[g]{$\delta$}{Differential Operator}
\nomenclature[g]{$\delta$}{Radius of Trust Region \nomunit{\meter}}
\nomenclature[g]{$\delta^*$}{Geodetic Latitude.  \nomunit{\radian}}

%%%% Ends by a blank page, congrats!
\newpage
\thispagestyle{empty}
\mbox{}

\end{document}
% 
\documentclass[titlepage,oneside,letterpaper,openright,12pt]{report}

%%%%%%%%%%%%%%%%%%%    COMMENTS    %%%%%%%%%%%%%%%%%%%
% 1. To stop numbering the lines, deactivate \linenumbers (activated after loading the package lineno)
% 
% 2. To make the index, you need to compile first with lualatex and then use makeindex. Example:
%		lualatex main.tex
%		makeindex main.nlo -s nomencl.ist -o main.nls
%		lualatex main.tex
% 
% 3. For the bibliography, use the command biber (as I am using the package biblatex, much more modern than bibtex).
%		To use multiple bibliography, use the refsection environment (3.13.3 Multiple Bibliographies in biblatex doc Version 3.13).
%		Do not use the chapterbib package with biblatex, there is refsection environment for that!
%		Each article will need to use
%		\begin{refsection}
%			Intro, Methods, Results, Discussion
%			\printbibliography[heading=subbibnumbered]
%		\end{refsection}
% 
% 4. I defined two languages in the document, French and British english (which is the main language), use \begin{french} ... \end{french} to switch to French in a paragraph
% 
% 5. To avoid multi-defined labels, I recommend to apply this command line in your shell for each tex file of the folders ./article1/, ./article2/, ...:
% 		sed -i.tmp 's/\\label{/\\label{WHATEVER::/g' *.tex
% 			This will change all the labels by adding WHATEVER before. Example (adding article1 for each tex file in ./article1/ folder):
% 			\label{sec::intro} in ./article1/chap1.tex becomes \label{article1::sec::intro}
%		sed -i.tmp 's/\\ref{/\\ref{WHATEVER::/g' *.tex
%			This will change all the refs by adding WHATEVER before. Example (adding article1 for each tex file in ./article1/ folder):
% 			\ref{sec::intro} in ./article1/chap1.tex becomes \ref{article1::sec::intro}
% 		sed -i.tmp 's/\\eqref{/\\eqref{WHATEVER::/g' *.tex
%			This will change all the eqrefs by adding WHATEVER before. Example (adding article1 for each tex file in ./article1/ folder):
% 			\eqref{eq::R0_equation} in ./article1/chap1.tex becomes \eqref{article1::eq::R0_equation}
% 
%	For safety reasons, the original file will be kept in a temporary file '.tmp'. Example:
%		sed -i.tmp 'Your instructions here' introduction.tex will create two files:
% 			introduction.tex which should be the one to keep (if no mistakes in the sed instructions)
% 			introduction.tex.tmp which is the original file (that you can dump if sed behaved the way you expected)
% 	For an introduction to the power of sed and regular expressions: https://www.grymoire.com/Unix/Sed.html#uh-1
% 
% 6. I defined some colours in this document. Except RdeepBlue, none of them is used in this template and can be deleted.
% 	If you delete RdeepBlue, do not forget to adapt all the commands using this colour

%%%%%%%%%%%%%%%%%%%    PACKAGES    %%%%%%%%%%%%%%%%%%%
%%%% font
\usepackage{fontspec}
\usepackage{marvosym}

%%%% Language
\usepackage{polyglossia}
	\setdefaultlanguage[variant=british]{english}
	\setotherlanguages{french}

%%%% Date
\usepackage{datetime}
	\newdateformat{monthyeardate}{\monthname[\THEMONTH] \THEYEAR}

%%%% marges, space, titles...
\usepackage[top=3.75cm, bottom=2.5cm, left=3cm, right=2.5cm]{geometry}
% \usepackage{setspace}
\setlength{\parindent}{0ex}   % indentation au début de chaque paragraphe
\setlength{\parskip}{3ex plus 0.5ex minus 0.2ex} % espace vertical entre paragraphes

\usepackage{titlesec}
\newcommand\chapterstring{CHAPITRE}
\titleformat{\chapter}[display]{\vspace{-8em}\center\bfseries}{\chapterstring~\thechapter}{12pt}{}[\vspace{1.5ex}]
\titleformat{\section}{\vspace{0mm plus 2cm}\addpenalty{-1000}\bfseries\sffamily}{\thesection}{1em}{}
\titleformat{\subsection}{\addpenalty{-500}\bfseries\sffamily}{\thesubsection}{1em}{}
\titleformat{\subsubsection}{\penalty-500\vspace{0pt plus 2pt}{}\bfseries\sffamily}{\thesubsubsection}{}{}[\vspace{-14pt}]

\usepackage{titletoc}
	\titlecontents{chapter}[0pt]{}{\bfseries\chapterstring~\thecontentslabel~}{\bfseries}{\hspace{1em plus 1fill}\contentspage}

%%%% Appendix
\usepackage{appendix}
%% Start of subappendices environment
\AtBeginEnvironment{subappendices}{%
	\chapter*{Appendix}
	\addcontentsline{toc}{chapter}{Appendices}
	\counterwithin{figure}{section}
	\counterwithin{table}{section}
}

%% End of subappendices environment
\AtEndEnvironment{subappendices}{%
	\counterwithout{figure}{section}
	\counterwithout{table}{section}
}

%%%% Graphics
%% To compile with lualatex
\usepackage[luatex]{graphicx}

%% Graphics' path, but I prefer to use absolute path for safety
% \graphicspath{{./article1/}{./article2/}{article3/}}

%% Numbering figures in sections rather than continuously
\usepackage{chngcntr}
\counterwithin{figure}{section}

%% Put figures within their declaration section 
\usepackage[section]{placeins}

%% To create multi-figures
\usepackage{caption}
\usepackage{subcaption}

%% Colours
\usepackage[dvipsnames, svgnames]{xcolor}
	\definecolor{Rblack}{RGB}{0,0,0}
	\definecolor{Rgrey}{RGB}{42,51,68}
	\definecolor{RdeepBlue}{RGB}{17,34,170}
	\definecolor{RmidBlue}{RGB}{32,88,220}
	\definecolor{RlightBlue}{RGB}{90,130,234}
	\definecolor{RuglyGreen}{RGB}{253,236,190}
	\definecolor{RyellowPea}{RGB}{253,219,91}
	\definecolor{RdeepYellow}{RGB}{250,185,53}
	\definecolor{Rsalmon}{RGB}{253,152,89}
	\definecolor{Rred}{RGB}{182,52,58}

	\definecolor{lightAvailability_grey}{RGB}{20,20,20}

%% Tikz
\usepackage{tikz}

%%%% Tab, list...
%% Best package for tables
\usepackage{booktabs}

%% List
\usepackage{enumerate}
\usepackage{enumitem}% http://ctan.org/pkg/enumitem

%% Nomenclature (containing three sections: Acronyms, greek symbols, roman symbols) https://tex.stackexchange.com/questions/452107/nomenclature-having-two-or-more-descriptions-and-units-for-a-symbol
\usepackage[intoc]{nomencl}

%% Change caption entries to whatever you want (also done in the ToC)
\addto\captionsenglish{ % Replace "english" with the language you use, if you use babel you might have to replace english by british or USenglish
	\renewcommand{\contentsname}{TABLE DES MATIÈRES}
	\renewcommand{\listtablename}{LISTE DES TABLEAUX}
	\renewcommand{\listfigurename}{LISTE DES FIGURES}
}

%%%% Links
\usepackage{url}
\usepackage[luatex, colorlinks=true, linkcolor=black, urlcolor=RdeepBlue, citecolor=RdeepBlue]{hyperref}

%%%% Bibliography and table of contents
%% Bibliography output using biblatex
\usepackage{csquotes}
\usepackage[style=apa, natbib=true, sorting=ynt]{biblatex}
% \usepackage[uniquelist=false, backend=biber, citestyle=authoryear-comp, sorting=nyt, sortcites=true, giveninits=true, hyperref, natbib, url=false, doi=false, eprint=false, isbn=false, maxbibnames=25, maxcitenames=2, mincitenames=1, uniquename=init, refsection=chapter]{biblatex}
\addbibresource{phd.bib}

%% Table of content (= ToC)
\usepackage[nottoc]{tocbibind}

%%%% Mathematics
\usepackage{amsthm}
\usepackage{amsmath}
	\allowdisplaybreaks % allows breaks between pages in formula
\usepackage{amssymb}
\usepackage{bbold}
\usepackage{dsfont}
\usepackage{mathrsfs}
\usepackage{bm}
\usepackage{xfrac}
\usepackage{etoolbox} % For renumbering
\usepackage{thmbox} % For "theorem" definitions
\usepackage{gensymb}
\usepackage{siunitx}
	\sisetup{
		inter-unit-product=\ensuremath{{}\cdot{}},
		per-mode=symbol
	}

\DeclareMathOperator{\logit}{logit}

%%%% Line numbers --conflict with amsthm-- pdf http://texblog.org/2012/02/08/adding-line-numbers-to-documents/
\usepackage{lineno}
	\linenumbers

%%%%%%%%%%%%%%%%%%%%%   OTHERS   %%%%%%%%%%%%%%%%%%%%%
% http://tex.stackexchange.com/questions/13304/which-package-version-am-i-using
\listfiles % Add \listfiles to your preamble and then look at the .log file. This will tell you the current version of all the packages loaded

%%%%%%%%%%%%%%%%%%%%%   NOMENCLATURE CODING   %%%%%%%%%%%%%%%%%%%%%
\makenomenclature
\renewcommand{\nomname}{LISTE DES ABBRÉVIATIONS}

\ExplSyntaxOn
\RenewDocumentCommand\nomgroup{m}
{
	\item[\textbf{\textcolor{RdeepBlue}{\choosenomgroup{#1}}}]
}
\NewDocumentCommand{\choosenomgroup}{m}
{
	\str_case:nn { #1 }
	{
		{A}{Acronymes}
		{L}{Symboles~latins}
		{G}{Symboles~grecs}
	}
}
\ExplSyntaxOff
\renewcommand*{\nompreamble}{\markboth{\nomname}{\nomname}}

\newcommand{\nomunit}[1]{%
	\renewcommand{\nomentryend}{\hspace*{\fill}\si{#1}}%
}

%%%%%%%%%%%%%%%%%%   HYPHENATION    %%%%%%%%%%%%%%%%%%
%%%% If latex has troubles to split a word, you can teach it here
% \hyphenation{cal-cu-lus}

%%%%%%%%%%%%%%%%%%%%%%%%%%%%%%%%%%%%%%%%%%%%%%%%%%%%%%%%%%%%%%%%%%%%
%%%%%%%%%%%%%%%%%   NEW COMMANDS USER'S SPECIFIC   %%%%%%%%%%%%%%%%%
%%%%%%%%%%%%%%%%%%%%%%%%%%%%%%%%%%%%%%%%%%%%%%%%%%%%%%%%%%%%%%%%%%%%
%%%%%%%% Commands that will be apply to the whole Ph.D. For chapter specific commands, declare them in the chapters (and use renewcommand if a name is used twice)
%%%% Text commands
\newcommand {\ie}{\textit{i.e., }}
\newcommand {\eg}{\textit{e.g., }}
\newcommand {\cf}{\textit{cf} }
\providecommand{\keywords}[1]{\textbf{Mots-clefs:} #1}


%%%% Tikz
%% Libraries (specific tikz commands are in article*/main.tex files)
\usetikzlibrary{arrows, plotmarks, decorations.markings}
\tikzstyle{arrow} = [->,>=stealth,thick,rounded corners=4pt,line width=1pt]
\usetikzlibrary{shadows}
\usetikzlibrary{shadings}
\usetikzlibrary{positioning} % relative coodinate
\usetikzlibrary{tikzmark, calc} % calc, to calculate coordinate
\usetikzlibrary{decorations.pathmorphing} % to snake an arrow
\usetikzlibrary{shapes.arrows}
\usetikzlibrary{patterns}

%% Gradient shadings command
\makeatletter
\def\createshadingfromlist#1#2#3{%
	\pgfutil@tempcnta=0\relax
	\pgfutil@for\pgf@tmp:={#3}\do{\advance\pgfutil@tempcnta by1}%
	\ifnum\pgfutil@tempcnta=1\relax%
		\edef\pgf@spec{color(0)=(#3);color(100)=(#3)}%
	\else%
		\pgfmathparse{50/(\pgfutil@tempcnta-1)}\let\pgf@step=\pgfmathresult%
		\pgfutil@tempcntb=1\relax%
		\pgfutil@for\pgf@tmp:={#3}\do{%
			\ifnum\pgfutil@tempcntb=1\relax%
				\edef\pgf@spec{color(0)=(\pgf@tmp);color(25)=(\pgf@tmp)}%
			\else%
	        \ifnum\pgfutil@tempcntb<\pgfutil@tempcnta\relax%
				\pgfmathparse{25+\pgf@step/4+(\pgfutil@tempcntb-1)*\pgf@step}%
				\edef\pgf@spec{\pgf@spec;color(\pgfmathresult)=(\pgf@tmp)}%
	        \else%
	        	\edef\pgf@spec{\pgf@spec;color(75)=(\pgf@tmp);color(100)=(\pgf@tmp)}%
	        \fi%
	    \fi%
	    \advance\pgfutil@tempcntb by1\relax%
	    }%
	\fi%
	\csname pgfdeclare#2shading\endcsname{#1}{100}\pgf@spec%
}

%%%%%%%% Math commands
%%%% Maths
\newcommand {\s}{{s}^{*}}
\newcommand {\sst}{\tilde{s}^{*}} % s* stable d'où le st
\newcommand {\N}{\tilde{N}}
\newcommand {\A}{\mathscr{A}}
\newcommand {\K}{\mathcal{K}}
\renewcommand{\S}{\mathscr{S}}
\newcommand{\R}{\mathds{R}}
\newcommand{\Prob}{\mathds{P}}
\newcommand{\F}{\mathcal{F}}

%%%% Theorem style
\newtheoremstyle{theo}{\topsep}{\topsep}{\itshape}{}{\bfseries}{.}{\newline}{\thmname{#1} \thmnumber{#2} \thmnote{~: \textit{#3}}}
\theoremstyle{theo}
\newtheorem{rem}{Remark}[section]
\newtheorem{defi}{Definition}[section]
\newtheorem{assum}{Assumption}[section]
\newtheorem{nota}{Notation}[section]

%%%%%%%%%%%%%%%%%%%%%   BEGIN DOCUMENT   %%%%%%%%%%%%%%%%%%%%%
\begin{document}
%%%% Starts by a blank page
\newpage
\thispagestyle{empty}
\mbox{}

%%%% Title page
\begin{titlepage}
	\begin{center}
		\textsc{Your title small caps} \\
	\vspace{2cm}
	par \\
	\vspace{2cm}
	Amaël Le Squin
	\vspace{1cm}
	Université de Sherbrooke \\
	\vspace{1cm}
	Thèse présentée au Département de biologie en vue \\
	de l'obtention du grade de docteur ès sciences (Ph.D.)
	\vfill
	\textsc{faculté des sciences} \\
	\textsc{université de sherbrooke} \\
	\vspace{2cm}
	Sherbrooke, Québec, Canada, \begin{french} \monthyeardate\today \end{french}
	\end{center}	
\end{titlepage}

%%%% Jury page (and not Jimmy page...)
\thispagestyle{empty}
\begin{center}
\begin{french}
	Le \today \\
	\vspace{1cm}
	\textit{Le jury a accepté la thèse d'Amaël Le Squin dans sa version finale} \\
	\vspace{1cm}
	Membre du jury \\
	\vspace{1cm}
	Professeur Dominique Gravel \\
	Directeur de recherche \\
	Département de biologie \\
	Université de Sherbrooke \\
	\vspace{1cm}
	Professeure Isabelle Boulangeat \\
	Codirectrice de recherche \\
	IRSTEA \\
	\vspace{1cm}
	Professeur Prénom et nom \\
	Évaluatrice ou Évaluateur externe \\
	Département nom \\
	Nom de l'institution \\
	\vspace{1cm}
	Professeure Prénom et nom \\
	Évaluatrice ou Évaluateur interne \\
	Département nom \\
	Nom de l'institution \\
	\vspace{1cm}
	Professeure Prénom et nom \\
	Président-rapporteur \\
	Département de biologie \\
	Université de Sherbrooke
\end{french}
\end{center}

%%%% From abstract (included) to Introduction (excluded)
%% Start roman numbering
\pagenumbering{roman}
\input{abstract_fr}

\input{abstract_en}

\input{acknowledgements}

%% Table of contents
\tableofcontents

%% Nomencalture
\printnomenclature

%% List of tables
\listoftables

%% List of figures
\listoffigures

\newpage
%%%% Main body
%% Start arabic numbering
\pagenumbering{arabic}

%% Input chapters
\input{./introduction}
\input{./article1/main}
% \input{./article3/main}

\nomenclature[a]{TPS}{Thermal protection System}
\nomenclature[L]{$v$}{Fluid velocity\nomunit{\metre\per\second}}
\nomenclature[G]{$\tau$}{Longitude \nomunit{\radian}}
\nomenclature[g]{$\tau$}{Dimensional time \nomunit{\second}}
\nomenclature[g]{$\delta$}{Geocentric Latitude \nomunit{\radian}}
\nomenclature[g]{$\delta$}{Differential Operator}
\nomenclature[g]{$\delta$}{Radius of Trust Region \nomunit{\meter}}
\nomenclature[g]{$\delta^*$}{Geodetic Latitude.  \nomunit{\radian}}

%%%% Ends by a blank page, congrats!
\newpage
\thispagestyle{empty}
\mbox{}

\end{document}

\nomenclature[a]{TPS}{Thermal protection System}
\nomenclature[L]{$v$}{Fluid velocity\nomunit{\metre\per\second}}
\nomenclature[G]{$\tau$}{Longitude \nomunit{\radian}}
\nomenclature[g]{$\tau$}{Dimensional time \nomunit{\second}}
\nomenclature[g]{$\delta$}{Geocentric Latitude \nomunit{\radian}}
\nomenclature[g]{$\delta$}{Differential Operator}
\nomenclature[g]{$\delta$}{Radius of Trust Region \nomunit{\meter}}
\nomenclature[g]{$\delta^*$}{Geodetic Latitude.  \nomunit{\radian}}

%%%% Ends by a blank page, congrats!
\newpage
\thispagestyle{empty}
\mbox{}

\end{document}

\nomenclature[a]{TPS}{Thermal protection System}
\nomenclature[L]{$v$}{Fluid velocity\nomunit{\metre\per\second}}
\nomenclature[G]{$\tau$}{Longitude \nomunit{\radian}}
\nomenclature[g]{$\tau$}{Dimensional time \nomunit{\second}}
\nomenclature[g]{$\delta$}{Geocentric Latitude \nomunit{\radian}}
\nomenclature[g]{$\delta$}{Differential Operator}
\nomenclature[g]{$\delta$}{Radius of Trust Region \nomunit{\meter}}
\nomenclature[g]{$\delta^*$}{Geodetic Latitude.  \nomunit{\radian}}

%%%% Ends by a blank page, congrats!
\newpage
\thispagestyle{empty}
\mbox{}

\end{document}
% 
\documentclass[titlepage,oneside,letterpaper,openright,12pt]{report}

%%%%%%%%%%%%%%%%%%%    COMMENTS    %%%%%%%%%%%%%%%%%%%
% 1. To stop numbering the lines, deactivate \linenumbers (activated after loading the package lineno)
% 
% 2. To make the index, you need to compile first with lualatex and then use makeindex. Example:
%		lualatex main.tex
%		makeindex main.nlo -s nomencl.ist -o main.nls
%		lualatex main.tex
% 
% 3. For the bibliography, use the command biber (as I am using the package biblatex, much more modern than bibtex).
%		To use multiple bibliography, use the refsection environment (3.13.3 Multiple Bibliographies in biblatex doc Version 3.13).
%		Do not use the chapterbib package with biblatex, there is refsection environment for that!
%		Each article will need to use
%		\begin{refsection}
%			Intro, Methods, Results, Discussion
%			\printbibliography[heading=subbibnumbered]
%		\end{refsection}
% 
% 4. I defined two languages in the document, French and British english (which is the main language), use \begin{french} ... \end{french} to switch to French in a paragraph
% 
% 5. To avoid multi-defined labels, I recommend to apply this command line in your shell for each tex file of the folders ./article1/, ./article2/, ...:
% 		sed -i.tmp 's/\\label{/\\label{WHATEVER::/g' *.tex
% 			This will change all the labels by adding WHATEVER before. Example (adding article1 for each tex file in ./article1/ folder):
% 			\label{sec::intro} in ./article1/chap1.tex becomes \label{article1::sec::intro}
%		sed -i.tmp 's/\\ref{/\\ref{WHATEVER::/g' *.tex
%			This will change all the refs by adding WHATEVER before. Example (adding article1 for each tex file in ./article1/ folder):
% 			\ref{sec::intro} in ./article1/chap1.tex becomes \ref{article1::sec::intro}
% 		sed -i.tmp 's/\\eqref{/\\eqref{WHATEVER::/g' *.tex
%			This will change all the eqrefs by adding WHATEVER before. Example (adding article1 for each tex file in ./article1/ folder):
% 			\eqref{eq::R0_equation} in ./article1/chap1.tex becomes \eqref{article1::eq::R0_equation}
% 
%	For safety reasons, the original file will be kept in a temporary file '.tmp'. Example:
%		sed -i.tmp 'Your instructions here' introduction.tex will create two files:
% 			introduction.tex which should be the one to keep (if no mistakes in the sed instructions)
% 			introduction.tex.tmp which is the original file (that you can dump if sed behaved the way you expected)
% 	For an introduction to the power of sed and regular expressions: https://www.grymoire.com/Unix/Sed.html#uh-1
% 
% 6. I defined some colours in this document. Except RdeepBlue, none of them is used in this template and can be deleted.
% 	If you delete RdeepBlue, do not forget to adapt all the commands using this colour

%%%%%%%%%%%%%%%%%%%    PACKAGES    %%%%%%%%%%%%%%%%%%%
%%%% font
\usepackage{fontspec}
\usepackage{marvosym}

%%%% Language
\usepackage{polyglossia}
	\setdefaultlanguage[variant=british]{english}
	\setotherlanguages{french}

%%%% Date
\usepackage{datetime}
	\newdateformat{monthyeardate}{\monthname[\THEMONTH] \THEYEAR}

%%%% marges, space, titles...
\usepackage[top=3.75cm, bottom=2.5cm, left=3cm, right=2.5cm]{geometry}
% \usepackage{setspace}
\setlength{\parindent}{0ex}   % indentation au début de chaque paragraphe
\setlength{\parskip}{3ex plus 0.5ex minus 0.2ex} % espace vertical entre paragraphes

\usepackage{titlesec}
\newcommand\chapterstring{CHAPITRE}
\titleformat{\chapter}[display]{\vspace{-8em}\center\bfseries}{\chapterstring~\thechapter}{12pt}{}[\vspace{1.5ex}]
\titleformat{\section}{\vspace{0mm plus 2cm}\addpenalty{-1000}\bfseries\sffamily}{\thesection}{1em}{}
\titleformat{\subsection}{\addpenalty{-500}\bfseries\sffamily}{\thesubsection}{1em}{}
\titleformat{\subsubsection}{\penalty-500\vspace{0pt plus 2pt}{}\bfseries\sffamily}{\thesubsubsection}{}{}[\vspace{-14pt}]

\usepackage{titletoc}
	\titlecontents{chapter}[0pt]{}{\bfseries\chapterstring~\thecontentslabel~}{\bfseries}{\hspace{1em plus 1fill}\contentspage}

%%%% Appendix
\usepackage{appendix}
%% Start of subappendices environment
\AtBeginEnvironment{subappendices}{%
	\chapter*{Appendix}
	\addcontentsline{toc}{chapter}{Appendices}
	\counterwithin{figure}{section}
	\counterwithin{table}{section}
}

%% End of subappendices environment
\AtEndEnvironment{subappendices}{%
	\counterwithout{figure}{section}
	\counterwithout{table}{section}
}

%%%% Graphics
%% To compile with lualatex
\usepackage[luatex]{graphicx}

%% Graphics' path, but I prefer to use absolute path for safety
% \graphicspath{{./article1/}{./article2/}{article3/}}

%% Numbering figures in sections rather than continuously
\usepackage{chngcntr}
\counterwithin{figure}{section}

%% Put figures within their declaration section 
\usepackage[section]{placeins}

%% To create multi-figures
\usepackage{caption}
\usepackage{subcaption}

%% Colours
\usepackage[dvipsnames, svgnames]{xcolor}
	\definecolor{Rblack}{RGB}{0,0,0}
	\definecolor{Rgrey}{RGB}{42,51,68}
	\definecolor{RdeepBlue}{RGB}{17,34,170}
	\definecolor{RmidBlue}{RGB}{32,88,220}
	\definecolor{RlightBlue}{RGB}{90,130,234}
	\definecolor{RuglyGreen}{RGB}{253,236,190}
	\definecolor{RyellowPea}{RGB}{253,219,91}
	\definecolor{RdeepYellow}{RGB}{250,185,53}
	\definecolor{Rsalmon}{RGB}{253,152,89}
	\definecolor{Rred}{RGB}{182,52,58}

	\definecolor{lightAvailability_grey}{RGB}{20,20,20}

%% Tikz
\usepackage{tikz}

%%%% Tab, list...
%% Best package for tables
\usepackage{booktabs}

%% List
\usepackage{enumerate}
\usepackage{enumitem}% http://ctan.org/pkg/enumitem

%% Nomenclature (containing three sections: Acronyms, greek symbols, roman symbols) https://tex.stackexchange.com/questions/452107/nomenclature-having-two-or-more-descriptions-and-units-for-a-symbol
\usepackage[intoc]{nomencl}

%% Change caption entries to whatever you want (also done in the ToC)
\addto\captionsenglish{ % Replace "english" with the language you use, if you use babel you might have to replace english by british or USenglish
	\renewcommand{\contentsname}{TABLE DES MATIÈRES}
	\renewcommand{\listtablename}{LISTE DES TABLEAUX}
	\renewcommand{\listfigurename}{LISTE DES FIGURES}
}

%%%% Links
\usepackage{url}
\usepackage[luatex, colorlinks=true, linkcolor=black, urlcolor=RdeepBlue, citecolor=RdeepBlue]{hyperref}

%%%% Bibliography and table of contents
%% Bibliography output using biblatex
\usepackage{csquotes}
\usepackage[style=apa, natbib=true, sorting=ynt]{biblatex}
% \usepackage[uniquelist=false, backend=biber, citestyle=authoryear-comp, sorting=nyt, sortcites=true, giveninits=true, hyperref, natbib, url=false, doi=false, eprint=false, isbn=false, maxbibnames=25, maxcitenames=2, mincitenames=1, uniquename=init, refsection=chapter]{biblatex}
\addbibresource{phd.bib}

%% Table of content (= ToC)
\usepackage[nottoc]{tocbibind}

%%%% Mathematics
\usepackage{amsthm}
\usepackage{amsmath}
	\allowdisplaybreaks % allows breaks between pages in formula
\usepackage{amssymb}
\usepackage{bbold}
\usepackage{dsfont}
\usepackage{mathrsfs}
\usepackage{bm}
\usepackage{xfrac}
\usepackage{etoolbox} % For renumbering
\usepackage{thmbox} % For "theorem" definitions
\usepackage{gensymb}
\usepackage{siunitx}
	\sisetup{
		inter-unit-product=\ensuremath{{}\cdot{}},
		per-mode=symbol
	}

\DeclareMathOperator{\logit}{logit}

%%%% Line numbers --conflict with amsthm-- pdf http://texblog.org/2012/02/08/adding-line-numbers-to-documents/
\usepackage{lineno}
	\linenumbers

%%%%%%%%%%%%%%%%%%%%%   OTHERS   %%%%%%%%%%%%%%%%%%%%%
% http://tex.stackexchange.com/questions/13304/which-package-version-am-i-using
\listfiles % Add \listfiles to your preamble and then look at the .log file. This will tell you the current version of all the packages loaded

%%%%%%%%%%%%%%%%%%%%%   NOMENCLATURE CODING   %%%%%%%%%%%%%%%%%%%%%
\makenomenclature
\renewcommand{\nomname}{LISTE DES ABBRÉVIATIONS}

\ExplSyntaxOn
\RenewDocumentCommand\nomgroup{m}
{
	\item[\textbf{\textcolor{RdeepBlue}{\choosenomgroup{#1}}}]
}
\NewDocumentCommand{\choosenomgroup}{m}
{
	\str_case:nn { #1 }
	{
		{A}{Acronymes}
		{L}{Symboles~latins}
		{G}{Symboles~grecs}
	}
}
\ExplSyntaxOff
\renewcommand*{\nompreamble}{\markboth{\nomname}{\nomname}}

\newcommand{\nomunit}[1]{%
	\renewcommand{\nomentryend}{\hspace*{\fill}\si{#1}}%
}

%%%%%%%%%%%%%%%%%%   HYPHENATION    %%%%%%%%%%%%%%%%%%
%%%% If latex has troubles to split a word, you can teach it here
% \hyphenation{cal-cu-lus}

%%%%%%%%%%%%%%%%%%%%%%%%%%%%%%%%%%%%%%%%%%%%%%%%%%%%%%%%%%%%%%%%%%%%
%%%%%%%%%%%%%%%%%   NEW COMMANDS USER'S SPECIFIC   %%%%%%%%%%%%%%%%%
%%%%%%%%%%%%%%%%%%%%%%%%%%%%%%%%%%%%%%%%%%%%%%%%%%%%%%%%%%%%%%%%%%%%
%%%%%%%% Commands that will be apply to the whole Ph.D. For chapter specific commands, declare them in the chapters (and use renewcommand if a name is used twice)
%%%% Text commands
\newcommand {\ie}{\textit{i.e., }}
\newcommand {\eg}{\textit{e.g., }}
\newcommand {\cf}{\textit{cf} }
\providecommand{\keywords}[1]{\textbf{Mots-clefs:} #1}


%%%% Tikz
%% Libraries (specific tikz commands are in article*/main.tex files)
\usetikzlibrary{arrows, plotmarks, decorations.markings}
\tikzstyle{arrow} = [->,>=stealth,thick,rounded corners=4pt,line width=1pt]
\usetikzlibrary{shadows}
\usetikzlibrary{shadings}
\usetikzlibrary{positioning} % relative coodinate
\usetikzlibrary{tikzmark, calc} % calc, to calculate coordinate
\usetikzlibrary{decorations.pathmorphing} % to snake an arrow
\usetikzlibrary{shapes.arrows}
\usetikzlibrary{patterns}

%% Gradient shadings command
\makeatletter
\def\createshadingfromlist#1#2#3{%
	\pgfutil@tempcnta=0\relax
	\pgfutil@for\pgf@tmp:={#3}\do{\advance\pgfutil@tempcnta by1}%
	\ifnum\pgfutil@tempcnta=1\relax%
		\edef\pgf@spec{color(0)=(#3);color(100)=(#3)}%
	\else%
		\pgfmathparse{50/(\pgfutil@tempcnta-1)}\let\pgf@step=\pgfmathresult%
		\pgfutil@tempcntb=1\relax%
		\pgfutil@for\pgf@tmp:={#3}\do{%
			\ifnum\pgfutil@tempcntb=1\relax%
				\edef\pgf@spec{color(0)=(\pgf@tmp);color(25)=(\pgf@tmp)}%
			\else%
	        \ifnum\pgfutil@tempcntb<\pgfutil@tempcnta\relax%
				\pgfmathparse{25+\pgf@step/4+(\pgfutil@tempcntb-1)*\pgf@step}%
				\edef\pgf@spec{\pgf@spec;color(\pgfmathresult)=(\pgf@tmp)}%
	        \else%
	        	\edef\pgf@spec{\pgf@spec;color(75)=(\pgf@tmp);color(100)=(\pgf@tmp)}%
	        \fi%
	    \fi%
	    \advance\pgfutil@tempcntb by1\relax%
	    }%
	\fi%
	\csname pgfdeclare#2shading\endcsname{#1}{100}\pgf@spec%
}

%%%%%%%% Math commands
%%%% Maths
\newcommand {\s}{{s}^{*}}
\newcommand {\sst}{\tilde{s}^{*}} % s* stable d'où le st
\newcommand {\N}{\tilde{N}}
\newcommand {\A}{\mathscr{A}}
\newcommand {\K}{\mathcal{K}}
\renewcommand{\S}{\mathscr{S}}
\newcommand{\R}{\mathds{R}}
\newcommand{\Prob}{\mathds{P}}
\newcommand{\F}{\mathcal{F}}

%%%% Theorem style
\newtheoremstyle{theo}{\topsep}{\topsep}{\itshape}{}{\bfseries}{.}{\newline}{\thmname{#1} \thmnumber{#2} \thmnote{~: \textit{#3}}}
\theoremstyle{theo}
\newtheorem{rem}{Remark}[section]
\newtheorem{defi}{Definition}[section]
\newtheorem{assum}{Assumption}[section]
\newtheorem{nota}{Notation}[section]

%%%%%%%%%%%%%%%%%%%%%   BEGIN DOCUMENT   %%%%%%%%%%%%%%%%%%%%%
\begin{document}
%%%% Starts by a blank page
\newpage
\thispagestyle{empty}
\mbox{}

%%%% Title page
\begin{titlepage}
	\begin{center}
		\textsc{Your title small caps} \\
	\vspace{2cm}
	par \\
	\vspace{2cm}
	Amaël Le Squin
	\vspace{1cm}
	Université de Sherbrooke \\
	\vspace{1cm}
	Thèse présentée au Département de biologie en vue \\
	de l'obtention du grade de docteur ès sciences (Ph.D.)
	\vfill
	\textsc{faculté des sciences} \\
	\textsc{université de sherbrooke} \\
	\vspace{2cm}
	Sherbrooke, Québec, Canada, \begin{french} \monthyeardate\today \end{french}
	\end{center}	
\end{titlepage}

%%%% Jury page (and not Jimmy page...)
\thispagestyle{empty}
\begin{center}
\begin{french}
	Le \today \\
	\vspace{1cm}
	\textit{Le jury a accepté la thèse d'Amaël Le Squin dans sa version finale} \\
	\vspace{1cm}
	Membre du jury \\
	\vspace{1cm}
	Professeur Dominique Gravel \\
	Directeur de recherche \\
	Département de biologie \\
	Université de Sherbrooke \\
	\vspace{1cm}
	Professeure Isabelle Boulangeat \\
	Codirectrice de recherche \\
	IRSTEA \\
	\vspace{1cm}
	Professeur Prénom et nom \\
	Évaluatrice ou Évaluateur externe \\
	Département nom \\
	Nom de l'institution \\
	\vspace{1cm}
	Professeure Prénom et nom \\
	Évaluatrice ou Évaluateur interne \\
	Département nom \\
	Nom de l'institution \\
	\vspace{1cm}
	Professeure Prénom et nom \\
	Président-rapporteur \\
	Département de biologie \\
	Université de Sherbrooke
\end{french}
\end{center}

%%%% From abstract (included) to Introduction (excluded)
%% Start roman numbering
\pagenumbering{roman}
\input{abstract_fr}

\input{abstract_en}

\input{acknowledgements}

%% Table of contents
\tableofcontents

%% Nomencalture
\printnomenclature

%% List of tables
\listoftables

%% List of figures
\listoffigures

\newpage
%%%% Main body
%% Start arabic numbering
\pagenumbering{arabic}

%% Input chapters
\input{./introduction}

\documentclass[titlepage,oneside,letterpaper,openright,12pt]{report}

%%%%%%%%%%%%%%%%%%%    COMMENTS    %%%%%%%%%%%%%%%%%%%
% 1. To stop numbering the lines, deactivate \linenumbers (activated after loading the package lineno)
% 
% 2. To make the index, you need to compile first with lualatex and then use makeindex. Example:
%		lualatex main.tex
%		makeindex main.nlo -s nomencl.ist -o main.nls
%		lualatex main.tex
% 
% 3. For the bibliography, use the command biber (as I am using the package biblatex, much more modern than bibtex).
%		To use multiple bibliography, use the refsection environment (3.13.3 Multiple Bibliographies in biblatex doc Version 3.13).
%		Do not use the chapterbib package with biblatex, there is refsection environment for that!
%		Each article will need to use
%		\begin{refsection}
%			Intro, Methods, Results, Discussion
%			\printbibliography[heading=subbibnumbered]
%		\end{refsection}
% 
% 4. I defined two languages in the document, French and British english (which is the main language), use \begin{french} ... \end{french} to switch to French in a paragraph
% 
% 5. To avoid multi-defined labels, I recommend to apply this command line in your shell for each tex file of the folders ./article1/, ./article2/, ...:
% 		sed -i.tmp 's/\\label{/\\label{WHATEVER::/g' *.tex
% 			This will change all the labels by adding WHATEVER before. Example (adding article1 for each tex file in ./article1/ folder):
% 			\label{sec::intro} in ./article1/chap1.tex becomes \label{article1::sec::intro}
%		sed -i.tmp 's/\\ref{/\\ref{WHATEVER::/g' *.tex
%			This will change all the refs by adding WHATEVER before. Example (adding article1 for each tex file in ./article1/ folder):
% 			\ref{sec::intro} in ./article1/chap1.tex becomes \ref{article1::sec::intro}
% 		sed -i.tmp 's/\\eqref{/\\eqref{WHATEVER::/g' *.tex
%			This will change all the eqrefs by adding WHATEVER before. Example (adding article1 for each tex file in ./article1/ folder):
% 			\eqref{eq::R0_equation} in ./article1/chap1.tex becomes \eqref{article1::eq::R0_equation}
% 
%	For safety reasons, the original file will be kept in a temporary file '.tmp'. Example:
%		sed -i.tmp 'Your instructions here' introduction.tex will create two files:
% 			introduction.tex which should be the one to keep (if no mistakes in the sed instructions)
% 			introduction.tex.tmp which is the original file (that you can dump if sed behaved the way you expected)
% 	For an introduction to the power of sed and regular expressions: https://www.grymoire.com/Unix/Sed.html#uh-1
% 
% 6. I defined some colours in this document. Except RdeepBlue, none of them is used in this template and can be deleted.
% 	If you delete RdeepBlue, do not forget to adapt all the commands using this colour

%%%%%%%%%%%%%%%%%%%    PACKAGES    %%%%%%%%%%%%%%%%%%%
%%%% font
\usepackage{fontspec}
\usepackage{marvosym}

%%%% Language
\usepackage{polyglossia}
	\setdefaultlanguage[variant=british]{english}
	\setotherlanguages{french}

%%%% Date
\usepackage{datetime}
	\newdateformat{monthyeardate}{\monthname[\THEMONTH] \THEYEAR}

%%%% marges, space, titles...
\usepackage[top=3.75cm, bottom=2.5cm, left=3cm, right=2.5cm]{geometry}
% \usepackage{setspace}
\setlength{\parindent}{0ex}   % indentation au début de chaque paragraphe
\setlength{\parskip}{3ex plus 0.5ex minus 0.2ex} % espace vertical entre paragraphes

\usepackage{titlesec}
\newcommand\chapterstring{CHAPITRE}
\titleformat{\chapter}[display]{\vspace{-8em}\center\bfseries}{\chapterstring~\thechapter}{12pt}{}[\vspace{1.5ex}]
\titleformat{\section}{\vspace{0mm plus 2cm}\addpenalty{-1000}\bfseries\sffamily}{\thesection}{1em}{}
\titleformat{\subsection}{\addpenalty{-500}\bfseries\sffamily}{\thesubsection}{1em}{}
\titleformat{\subsubsection}{\penalty-500\vspace{0pt plus 2pt}{}\bfseries\sffamily}{\thesubsubsection}{}{}[\vspace{-14pt}]

\usepackage{titletoc}
	\titlecontents{chapter}[0pt]{}{\bfseries\chapterstring~\thecontentslabel~}{\bfseries}{\hspace{1em plus 1fill}\contentspage}

%%%% Appendix
\usepackage{appendix}
%% Start of subappendices environment
\AtBeginEnvironment{subappendices}{%
	\chapter*{Appendix}
	\addcontentsline{toc}{chapter}{Appendices}
	\counterwithin{figure}{section}
	\counterwithin{table}{section}
}

%% End of subappendices environment
\AtEndEnvironment{subappendices}{%
	\counterwithout{figure}{section}
	\counterwithout{table}{section}
}

%%%% Graphics
%% To compile with lualatex
\usepackage[luatex]{graphicx}

%% Graphics' path, but I prefer to use absolute path for safety
% \graphicspath{{./article1/}{./article2/}{article3/}}

%% Numbering figures in sections rather than continuously
\usepackage{chngcntr}
\counterwithin{figure}{section}

%% Put figures within their declaration section 
\usepackage[section]{placeins}

%% To create multi-figures
\usepackage{caption}
\usepackage{subcaption}

%% Colours
\usepackage[dvipsnames, svgnames]{xcolor}
	\definecolor{Rblack}{RGB}{0,0,0}
	\definecolor{Rgrey}{RGB}{42,51,68}
	\definecolor{RdeepBlue}{RGB}{17,34,170}
	\definecolor{RmidBlue}{RGB}{32,88,220}
	\definecolor{RlightBlue}{RGB}{90,130,234}
	\definecolor{RuglyGreen}{RGB}{253,236,190}
	\definecolor{RyellowPea}{RGB}{253,219,91}
	\definecolor{RdeepYellow}{RGB}{250,185,53}
	\definecolor{Rsalmon}{RGB}{253,152,89}
	\definecolor{Rred}{RGB}{182,52,58}

	\definecolor{lightAvailability_grey}{RGB}{20,20,20}

%% Tikz
\usepackage{tikz}

%%%% Tab, list...
%% Best package for tables
\usepackage{booktabs}

%% List
\usepackage{enumerate}
\usepackage{enumitem}% http://ctan.org/pkg/enumitem

%% Nomenclature (containing three sections: Acronyms, greek symbols, roman symbols) https://tex.stackexchange.com/questions/452107/nomenclature-having-two-or-more-descriptions-and-units-for-a-symbol
\usepackage[intoc]{nomencl}

%% Change caption entries to whatever you want (also done in the ToC)
\addto\captionsenglish{ % Replace "english" with the language you use, if you use babel you might have to replace english by british or USenglish
	\renewcommand{\contentsname}{TABLE DES MATIÈRES}
	\renewcommand{\listtablename}{LISTE DES TABLEAUX}
	\renewcommand{\listfigurename}{LISTE DES FIGURES}
}

%%%% Links
\usepackage{url}
\usepackage[luatex, colorlinks=true, linkcolor=black, urlcolor=RdeepBlue, citecolor=RdeepBlue]{hyperref}

%%%% Bibliography and table of contents
%% Bibliography output using biblatex
\usepackage{csquotes}
\usepackage[style=apa, natbib=true, sorting=ynt]{biblatex}
% \usepackage[uniquelist=false, backend=biber, citestyle=authoryear-comp, sorting=nyt, sortcites=true, giveninits=true, hyperref, natbib, url=false, doi=false, eprint=false, isbn=false, maxbibnames=25, maxcitenames=2, mincitenames=1, uniquename=init, refsection=chapter]{biblatex}
\addbibresource{phd.bib}

%% Table of content (= ToC)
\usepackage[nottoc]{tocbibind}

%%%% Mathematics
\usepackage{amsthm}
\usepackage{amsmath}
	\allowdisplaybreaks % allows breaks between pages in formula
\usepackage{amssymb}
\usepackage{bbold}
\usepackage{dsfont}
\usepackage{mathrsfs}
\usepackage{bm}
\usepackage{xfrac}
\usepackage{etoolbox} % For renumbering
\usepackage{thmbox} % For "theorem" definitions
\usepackage{gensymb}
\usepackage{siunitx}
	\sisetup{
		inter-unit-product=\ensuremath{{}\cdot{}},
		per-mode=symbol
	}

\DeclareMathOperator{\logit}{logit}

%%%% Line numbers --conflict with amsthm-- pdf http://texblog.org/2012/02/08/adding-line-numbers-to-documents/
\usepackage{lineno}
	\linenumbers

%%%%%%%%%%%%%%%%%%%%%   OTHERS   %%%%%%%%%%%%%%%%%%%%%
% http://tex.stackexchange.com/questions/13304/which-package-version-am-i-using
\listfiles % Add \listfiles to your preamble and then look at the .log file. This will tell you the current version of all the packages loaded

%%%%%%%%%%%%%%%%%%%%%   NOMENCLATURE CODING   %%%%%%%%%%%%%%%%%%%%%
\makenomenclature
\renewcommand{\nomname}{LISTE DES ABBRÉVIATIONS}

\ExplSyntaxOn
\RenewDocumentCommand\nomgroup{m}
{
	\item[\textbf{\textcolor{RdeepBlue}{\choosenomgroup{#1}}}]
}
\NewDocumentCommand{\choosenomgroup}{m}
{
	\str_case:nn { #1 }
	{
		{A}{Acronymes}
		{L}{Symboles~latins}
		{G}{Symboles~grecs}
	}
}
\ExplSyntaxOff
\renewcommand*{\nompreamble}{\markboth{\nomname}{\nomname}}

\newcommand{\nomunit}[1]{%
	\renewcommand{\nomentryend}{\hspace*{\fill}\si{#1}}%
}

%%%%%%%%%%%%%%%%%%   HYPHENATION    %%%%%%%%%%%%%%%%%%
%%%% If latex has troubles to split a word, you can teach it here
% \hyphenation{cal-cu-lus}

%%%%%%%%%%%%%%%%%%%%%%%%%%%%%%%%%%%%%%%%%%%%%%%%%%%%%%%%%%%%%%%%%%%%
%%%%%%%%%%%%%%%%%   NEW COMMANDS USER'S SPECIFIC   %%%%%%%%%%%%%%%%%
%%%%%%%%%%%%%%%%%%%%%%%%%%%%%%%%%%%%%%%%%%%%%%%%%%%%%%%%%%%%%%%%%%%%
%%%%%%%% Commands that will be apply to the whole Ph.D. For chapter specific commands, declare them in the chapters (and use renewcommand if a name is used twice)
%%%% Text commands
\newcommand {\ie}{\textit{i.e., }}
\newcommand {\eg}{\textit{e.g., }}
\newcommand {\cf}{\textit{cf} }
\providecommand{\keywords}[1]{\textbf{Mots-clefs:} #1}


%%%% Tikz
%% Libraries (specific tikz commands are in article*/main.tex files)
\usetikzlibrary{arrows, plotmarks, decorations.markings}
\tikzstyle{arrow} = [->,>=stealth,thick,rounded corners=4pt,line width=1pt]
\usetikzlibrary{shadows}
\usetikzlibrary{shadings}
\usetikzlibrary{positioning} % relative coodinate
\usetikzlibrary{tikzmark, calc} % calc, to calculate coordinate
\usetikzlibrary{decorations.pathmorphing} % to snake an arrow
\usetikzlibrary{shapes.arrows}
\usetikzlibrary{patterns}

%% Gradient shadings command
\makeatletter
\def\createshadingfromlist#1#2#3{%
	\pgfutil@tempcnta=0\relax
	\pgfutil@for\pgf@tmp:={#3}\do{\advance\pgfutil@tempcnta by1}%
	\ifnum\pgfutil@tempcnta=1\relax%
		\edef\pgf@spec{color(0)=(#3);color(100)=(#3)}%
	\else%
		\pgfmathparse{50/(\pgfutil@tempcnta-1)}\let\pgf@step=\pgfmathresult%
		\pgfutil@tempcntb=1\relax%
		\pgfutil@for\pgf@tmp:={#3}\do{%
			\ifnum\pgfutil@tempcntb=1\relax%
				\edef\pgf@spec{color(0)=(\pgf@tmp);color(25)=(\pgf@tmp)}%
			\else%
	        \ifnum\pgfutil@tempcntb<\pgfutil@tempcnta\relax%
				\pgfmathparse{25+\pgf@step/4+(\pgfutil@tempcntb-1)*\pgf@step}%
				\edef\pgf@spec{\pgf@spec;color(\pgfmathresult)=(\pgf@tmp)}%
	        \else%
	        	\edef\pgf@spec{\pgf@spec;color(75)=(\pgf@tmp);color(100)=(\pgf@tmp)}%
	        \fi%
	    \fi%
	    \advance\pgfutil@tempcntb by1\relax%
	    }%
	\fi%
	\csname pgfdeclare#2shading\endcsname{#1}{100}\pgf@spec%
}

%%%%%%%% Math commands
%%%% Maths
\newcommand {\s}{{s}^{*}}
\newcommand {\sst}{\tilde{s}^{*}} % s* stable d'où le st
\newcommand {\N}{\tilde{N}}
\newcommand {\A}{\mathscr{A}}
\newcommand {\K}{\mathcal{K}}
\renewcommand{\S}{\mathscr{S}}
\newcommand{\R}{\mathds{R}}
\newcommand{\Prob}{\mathds{P}}
\newcommand{\F}{\mathcal{F}}

%%%% Theorem style
\newtheoremstyle{theo}{\topsep}{\topsep}{\itshape}{}{\bfseries}{.}{\newline}{\thmname{#1} \thmnumber{#2} \thmnote{~: \textit{#3}}}
\theoremstyle{theo}
\newtheorem{rem}{Remark}[section]
\newtheorem{defi}{Definition}[section]
\newtheorem{assum}{Assumption}[section]
\newtheorem{nota}{Notation}[section]

%%%%%%%%%%%%%%%%%%%%%   BEGIN DOCUMENT   %%%%%%%%%%%%%%%%%%%%%
\begin{document}
%%%% Starts by a blank page
\newpage
\thispagestyle{empty}
\mbox{}

%%%% Title page
\begin{titlepage}
	\begin{center}
		\textsc{Your title small caps} \\
	\vspace{2cm}
	par \\
	\vspace{2cm}
	Amaël Le Squin
	\vspace{1cm}
	Université de Sherbrooke \\
	\vspace{1cm}
	Thèse présentée au Département de biologie en vue \\
	de l'obtention du grade de docteur ès sciences (Ph.D.)
	\vfill
	\textsc{faculté des sciences} \\
	\textsc{université de sherbrooke} \\
	\vspace{2cm}
	Sherbrooke, Québec, Canada, \begin{french} \monthyeardate\today \end{french}
	\end{center}	
\end{titlepage}

%%%% Jury page (and not Jimmy page...)
\thispagestyle{empty}
\begin{center}
\begin{french}
	Le \today \\
	\vspace{1cm}
	\textit{Le jury a accepté la thèse d'Amaël Le Squin dans sa version finale} \\
	\vspace{1cm}
	Membre du jury \\
	\vspace{1cm}
	Professeur Dominique Gravel \\
	Directeur de recherche \\
	Département de biologie \\
	Université de Sherbrooke \\
	\vspace{1cm}
	Professeure Isabelle Boulangeat \\
	Codirectrice de recherche \\
	IRSTEA \\
	\vspace{1cm}
	Professeur Prénom et nom \\
	Évaluatrice ou Évaluateur externe \\
	Département nom \\
	Nom de l'institution \\
	\vspace{1cm}
	Professeure Prénom et nom \\
	Évaluatrice ou Évaluateur interne \\
	Département nom \\
	Nom de l'institution \\
	\vspace{1cm}
	Professeure Prénom et nom \\
	Président-rapporteur \\
	Département de biologie \\
	Université de Sherbrooke
\end{french}
\end{center}

%%%% From abstract (included) to Introduction (excluded)
%% Start roman numbering
\pagenumbering{roman}
\input{abstract_fr}

\input{abstract_en}

\input{acknowledgements}

%% Table of contents
\tableofcontents

%% Nomencalture
\printnomenclature

%% List of tables
\listoftables

%% List of figures
\listoffigures

\newpage
%%%% Main body
%% Start arabic numbering
\pagenumbering{arabic}

%% Input chapters
\input{./introduction}

\documentclass[titlepage,oneside,letterpaper,openright,12pt]{report}

%%%%%%%%%%%%%%%%%%%    COMMENTS    %%%%%%%%%%%%%%%%%%%
% 1. To stop numbering the lines, deactivate \linenumbers (activated after loading the package lineno)
% 
% 2. To make the index, you need to compile first with lualatex and then use makeindex. Example:
%		lualatex main.tex
%		makeindex main.nlo -s nomencl.ist -o main.nls
%		lualatex main.tex
% 
% 3. For the bibliography, use the command biber (as I am using the package biblatex, much more modern than bibtex).
%		To use multiple bibliography, use the refsection environment (3.13.3 Multiple Bibliographies in biblatex doc Version 3.13).
%		Do not use the chapterbib package with biblatex, there is refsection environment for that!
%		Each article will need to use
%		\begin{refsection}
%			Intro, Methods, Results, Discussion
%			\printbibliography[heading=subbibnumbered]
%		\end{refsection}
% 
% 4. I defined two languages in the document, French and British english (which is the main language), use \begin{french} ... \end{french} to switch to French in a paragraph
% 
% 5. To avoid multi-defined labels, I recommend to apply this command line in your shell for each tex file of the folders ./article1/, ./article2/, ...:
% 		sed -i.tmp 's/\\label{/\\label{WHATEVER::/g' *.tex
% 			This will change all the labels by adding WHATEVER before. Example (adding article1 for each tex file in ./article1/ folder):
% 			\label{sec::intro} in ./article1/chap1.tex becomes \label{article1::sec::intro}
%		sed -i.tmp 's/\\ref{/\\ref{WHATEVER::/g' *.tex
%			This will change all the refs by adding WHATEVER before. Example (adding article1 for each tex file in ./article1/ folder):
% 			\ref{sec::intro} in ./article1/chap1.tex becomes \ref{article1::sec::intro}
% 		sed -i.tmp 's/\\eqref{/\\eqref{WHATEVER::/g' *.tex
%			This will change all the eqrefs by adding WHATEVER before. Example (adding article1 for each tex file in ./article1/ folder):
% 			\eqref{eq::R0_equation} in ./article1/chap1.tex becomes \eqref{article1::eq::R0_equation}
% 
%	For safety reasons, the original file will be kept in a temporary file '.tmp'. Example:
%		sed -i.tmp 'Your instructions here' introduction.tex will create two files:
% 			introduction.tex which should be the one to keep (if no mistakes in the sed instructions)
% 			introduction.tex.tmp which is the original file (that you can dump if sed behaved the way you expected)
% 	For an introduction to the power of sed and regular expressions: https://www.grymoire.com/Unix/Sed.html#uh-1
% 
% 6. I defined some colours in this document. Except RdeepBlue, none of them is used in this template and can be deleted.
% 	If you delete RdeepBlue, do not forget to adapt all the commands using this colour

%%%%%%%%%%%%%%%%%%%    PACKAGES    %%%%%%%%%%%%%%%%%%%
%%%% font
\usepackage{fontspec}
\usepackage{marvosym}

%%%% Language
\usepackage{polyglossia}
	\setdefaultlanguage[variant=british]{english}
	\setotherlanguages{french}

%%%% Date
\usepackage{datetime}
	\newdateformat{monthyeardate}{\monthname[\THEMONTH] \THEYEAR}

%%%% marges, space, titles...
\usepackage[top=3.75cm, bottom=2.5cm, left=3cm, right=2.5cm]{geometry}
% \usepackage{setspace}
\setlength{\parindent}{0ex}   % indentation au début de chaque paragraphe
\setlength{\parskip}{3ex plus 0.5ex minus 0.2ex} % espace vertical entre paragraphes

\usepackage{titlesec}
\newcommand\chapterstring{CHAPITRE}
\titleformat{\chapter}[display]{\vspace{-8em}\center\bfseries}{\chapterstring~\thechapter}{12pt}{}[\vspace{1.5ex}]
\titleformat{\section}{\vspace{0mm plus 2cm}\addpenalty{-1000}\bfseries\sffamily}{\thesection}{1em}{}
\titleformat{\subsection}{\addpenalty{-500}\bfseries\sffamily}{\thesubsection}{1em}{}
\titleformat{\subsubsection}{\penalty-500\vspace{0pt plus 2pt}{}\bfseries\sffamily}{\thesubsubsection}{}{}[\vspace{-14pt}]

\usepackage{titletoc}
	\titlecontents{chapter}[0pt]{}{\bfseries\chapterstring~\thecontentslabel~}{\bfseries}{\hspace{1em plus 1fill}\contentspage}

%%%% Appendix
\usepackage{appendix}
%% Start of subappendices environment
\AtBeginEnvironment{subappendices}{%
	\chapter*{Appendix}
	\addcontentsline{toc}{chapter}{Appendices}
	\counterwithin{figure}{section}
	\counterwithin{table}{section}
}

%% End of subappendices environment
\AtEndEnvironment{subappendices}{%
	\counterwithout{figure}{section}
	\counterwithout{table}{section}
}

%%%% Graphics
%% To compile with lualatex
\usepackage[luatex]{graphicx}

%% Graphics' path, but I prefer to use absolute path for safety
% \graphicspath{{./article1/}{./article2/}{article3/}}

%% Numbering figures in sections rather than continuously
\usepackage{chngcntr}
\counterwithin{figure}{section}

%% Put figures within their declaration section 
\usepackage[section]{placeins}

%% To create multi-figures
\usepackage{caption}
\usepackage{subcaption}

%% Colours
\usepackage[dvipsnames, svgnames]{xcolor}
	\definecolor{Rblack}{RGB}{0,0,0}
	\definecolor{Rgrey}{RGB}{42,51,68}
	\definecolor{RdeepBlue}{RGB}{17,34,170}
	\definecolor{RmidBlue}{RGB}{32,88,220}
	\definecolor{RlightBlue}{RGB}{90,130,234}
	\definecolor{RuglyGreen}{RGB}{253,236,190}
	\definecolor{RyellowPea}{RGB}{253,219,91}
	\definecolor{RdeepYellow}{RGB}{250,185,53}
	\definecolor{Rsalmon}{RGB}{253,152,89}
	\definecolor{Rred}{RGB}{182,52,58}

	\definecolor{lightAvailability_grey}{RGB}{20,20,20}

%% Tikz
\usepackage{tikz}

%%%% Tab, list...
%% Best package for tables
\usepackage{booktabs}

%% List
\usepackage{enumerate}
\usepackage{enumitem}% http://ctan.org/pkg/enumitem

%% Nomenclature (containing three sections: Acronyms, greek symbols, roman symbols) https://tex.stackexchange.com/questions/452107/nomenclature-having-two-or-more-descriptions-and-units-for-a-symbol
\usepackage[intoc]{nomencl}

%% Change caption entries to whatever you want (also done in the ToC)
\addto\captionsenglish{ % Replace "english" with the language you use, if you use babel you might have to replace english by british or USenglish
	\renewcommand{\contentsname}{TABLE DES MATIÈRES}
	\renewcommand{\listtablename}{LISTE DES TABLEAUX}
	\renewcommand{\listfigurename}{LISTE DES FIGURES}
}

%%%% Links
\usepackage{url}
\usepackage[luatex, colorlinks=true, linkcolor=black, urlcolor=RdeepBlue, citecolor=RdeepBlue]{hyperref}

%%%% Bibliography and table of contents
%% Bibliography output using biblatex
\usepackage{csquotes}
\usepackage[style=apa, natbib=true, sorting=ynt]{biblatex}
% \usepackage[uniquelist=false, backend=biber, citestyle=authoryear-comp, sorting=nyt, sortcites=true, giveninits=true, hyperref, natbib, url=false, doi=false, eprint=false, isbn=false, maxbibnames=25, maxcitenames=2, mincitenames=1, uniquename=init, refsection=chapter]{biblatex}
\addbibresource{phd.bib}

%% Table of content (= ToC)
\usepackage[nottoc]{tocbibind}

%%%% Mathematics
\usepackage{amsthm}
\usepackage{amsmath}
	\allowdisplaybreaks % allows breaks between pages in formula
\usepackage{amssymb}
\usepackage{bbold}
\usepackage{dsfont}
\usepackage{mathrsfs}
\usepackage{bm}
\usepackage{xfrac}
\usepackage{etoolbox} % For renumbering
\usepackage{thmbox} % For "theorem" definitions
\usepackage{gensymb}
\usepackage{siunitx}
	\sisetup{
		inter-unit-product=\ensuremath{{}\cdot{}},
		per-mode=symbol
	}

\DeclareMathOperator{\logit}{logit}

%%%% Line numbers --conflict with amsthm-- pdf http://texblog.org/2012/02/08/adding-line-numbers-to-documents/
\usepackage{lineno}
	\linenumbers

%%%%%%%%%%%%%%%%%%%%%   OTHERS   %%%%%%%%%%%%%%%%%%%%%
% http://tex.stackexchange.com/questions/13304/which-package-version-am-i-using
\listfiles % Add \listfiles to your preamble and then look at the .log file. This will tell you the current version of all the packages loaded

%%%%%%%%%%%%%%%%%%%%%   NOMENCLATURE CODING   %%%%%%%%%%%%%%%%%%%%%
\makenomenclature
\renewcommand{\nomname}{LISTE DES ABBRÉVIATIONS}

\ExplSyntaxOn
\RenewDocumentCommand\nomgroup{m}
{
	\item[\textbf{\textcolor{RdeepBlue}{\choosenomgroup{#1}}}]
}
\NewDocumentCommand{\choosenomgroup}{m}
{
	\str_case:nn { #1 }
	{
		{A}{Acronymes}
		{L}{Symboles~latins}
		{G}{Symboles~grecs}
	}
}
\ExplSyntaxOff
\renewcommand*{\nompreamble}{\markboth{\nomname}{\nomname}}

\newcommand{\nomunit}[1]{%
	\renewcommand{\nomentryend}{\hspace*{\fill}\si{#1}}%
}

%%%%%%%%%%%%%%%%%%   HYPHENATION    %%%%%%%%%%%%%%%%%%
%%%% If latex has troubles to split a word, you can teach it here
% \hyphenation{cal-cu-lus}

%%%%%%%%%%%%%%%%%%%%%%%%%%%%%%%%%%%%%%%%%%%%%%%%%%%%%%%%%%%%%%%%%%%%
%%%%%%%%%%%%%%%%%   NEW COMMANDS USER'S SPECIFIC   %%%%%%%%%%%%%%%%%
%%%%%%%%%%%%%%%%%%%%%%%%%%%%%%%%%%%%%%%%%%%%%%%%%%%%%%%%%%%%%%%%%%%%
%%%%%%%% Commands that will be apply to the whole Ph.D. For chapter specific commands, declare them in the chapters (and use renewcommand if a name is used twice)
%%%% Text commands
\newcommand {\ie}{\textit{i.e., }}
\newcommand {\eg}{\textit{e.g., }}
\newcommand {\cf}{\textit{cf} }
\providecommand{\keywords}[1]{\textbf{Mots-clefs:} #1}


%%%% Tikz
%% Libraries (specific tikz commands are in article*/main.tex files)
\usetikzlibrary{arrows, plotmarks, decorations.markings}
\tikzstyle{arrow} = [->,>=stealth,thick,rounded corners=4pt,line width=1pt]
\usetikzlibrary{shadows}
\usetikzlibrary{shadings}
\usetikzlibrary{positioning} % relative coodinate
\usetikzlibrary{tikzmark, calc} % calc, to calculate coordinate
\usetikzlibrary{decorations.pathmorphing} % to snake an arrow
\usetikzlibrary{shapes.arrows}
\usetikzlibrary{patterns}

%% Gradient shadings command
\makeatletter
\def\createshadingfromlist#1#2#3{%
	\pgfutil@tempcnta=0\relax
	\pgfutil@for\pgf@tmp:={#3}\do{\advance\pgfutil@tempcnta by1}%
	\ifnum\pgfutil@tempcnta=1\relax%
		\edef\pgf@spec{color(0)=(#3);color(100)=(#3)}%
	\else%
		\pgfmathparse{50/(\pgfutil@tempcnta-1)}\let\pgf@step=\pgfmathresult%
		\pgfutil@tempcntb=1\relax%
		\pgfutil@for\pgf@tmp:={#3}\do{%
			\ifnum\pgfutil@tempcntb=1\relax%
				\edef\pgf@spec{color(0)=(\pgf@tmp);color(25)=(\pgf@tmp)}%
			\else%
	        \ifnum\pgfutil@tempcntb<\pgfutil@tempcnta\relax%
				\pgfmathparse{25+\pgf@step/4+(\pgfutil@tempcntb-1)*\pgf@step}%
				\edef\pgf@spec{\pgf@spec;color(\pgfmathresult)=(\pgf@tmp)}%
	        \else%
	        	\edef\pgf@spec{\pgf@spec;color(75)=(\pgf@tmp);color(100)=(\pgf@tmp)}%
	        \fi%
	    \fi%
	    \advance\pgfutil@tempcntb by1\relax%
	    }%
	\fi%
	\csname pgfdeclare#2shading\endcsname{#1}{100}\pgf@spec%
}

%%%%%%%% Math commands
%%%% Maths
\newcommand {\s}{{s}^{*}}
\newcommand {\sst}{\tilde{s}^{*}} % s* stable d'où le st
\newcommand {\N}{\tilde{N}}
\newcommand {\A}{\mathscr{A}}
\newcommand {\K}{\mathcal{K}}
\renewcommand{\S}{\mathscr{S}}
\newcommand{\R}{\mathds{R}}
\newcommand{\Prob}{\mathds{P}}
\newcommand{\F}{\mathcal{F}}

%%%% Theorem style
\newtheoremstyle{theo}{\topsep}{\topsep}{\itshape}{}{\bfseries}{.}{\newline}{\thmname{#1} \thmnumber{#2} \thmnote{~: \textit{#3}}}
\theoremstyle{theo}
\newtheorem{rem}{Remark}[section]
\newtheorem{defi}{Definition}[section]
\newtheorem{assum}{Assumption}[section]
\newtheorem{nota}{Notation}[section]

%%%%%%%%%%%%%%%%%%%%%   BEGIN DOCUMENT   %%%%%%%%%%%%%%%%%%%%%
\begin{document}
%%%% Starts by a blank page
\newpage
\thispagestyle{empty}
\mbox{}

%%%% Title page
\begin{titlepage}
	\begin{center}
		\textsc{Your title small caps} \\
	\vspace{2cm}
	par \\
	\vspace{2cm}
	Amaël Le Squin
	\vspace{1cm}
	Université de Sherbrooke \\
	\vspace{1cm}
	Thèse présentée au Département de biologie en vue \\
	de l'obtention du grade de docteur ès sciences (Ph.D.)
	\vfill
	\textsc{faculté des sciences} \\
	\textsc{université de sherbrooke} \\
	\vspace{2cm}
	Sherbrooke, Québec, Canada, \begin{french} \monthyeardate\today \end{french}
	\end{center}	
\end{titlepage}

%%%% Jury page (and not Jimmy page...)
\thispagestyle{empty}
\begin{center}
\begin{french}
	Le \today \\
	\vspace{1cm}
	\textit{Le jury a accepté la thèse d'Amaël Le Squin dans sa version finale} \\
	\vspace{1cm}
	Membre du jury \\
	\vspace{1cm}
	Professeur Dominique Gravel \\
	Directeur de recherche \\
	Département de biologie \\
	Université de Sherbrooke \\
	\vspace{1cm}
	Professeure Isabelle Boulangeat \\
	Codirectrice de recherche \\
	IRSTEA \\
	\vspace{1cm}
	Professeur Prénom et nom \\
	Évaluatrice ou Évaluateur externe \\
	Département nom \\
	Nom de l'institution \\
	\vspace{1cm}
	Professeure Prénom et nom \\
	Évaluatrice ou Évaluateur interne \\
	Département nom \\
	Nom de l'institution \\
	\vspace{1cm}
	Professeure Prénom et nom \\
	Président-rapporteur \\
	Département de biologie \\
	Université de Sherbrooke
\end{french}
\end{center}

%%%% From abstract (included) to Introduction (excluded)
%% Start roman numbering
\pagenumbering{roman}
\input{abstract_fr}

\input{abstract_en}

\input{acknowledgements}

%% Table of contents
\tableofcontents

%% Nomencalture
\printnomenclature

%% List of tables
\listoftables

%% List of figures
\listoffigures

\newpage
%%%% Main body
%% Start arabic numbering
\pagenumbering{arabic}

%% Input chapters
\input{./introduction}
\input{./article1/main}
% \input{./article3/main}

\nomenclature[a]{TPS}{Thermal protection System}
\nomenclature[L]{$v$}{Fluid velocity\nomunit{\metre\per\second}}
\nomenclature[G]{$\tau$}{Longitude \nomunit{\radian}}
\nomenclature[g]{$\tau$}{Dimensional time \nomunit{\second}}
\nomenclature[g]{$\delta$}{Geocentric Latitude \nomunit{\radian}}
\nomenclature[g]{$\delta$}{Differential Operator}
\nomenclature[g]{$\delta$}{Radius of Trust Region \nomunit{\meter}}
\nomenclature[g]{$\delta^*$}{Geodetic Latitude.  \nomunit{\radian}}

%%%% Ends by a blank page, congrats!
\newpage
\thispagestyle{empty}
\mbox{}

\end{document}
% 
\documentclass[titlepage,oneside,letterpaper,openright,12pt]{report}

%%%%%%%%%%%%%%%%%%%    COMMENTS    %%%%%%%%%%%%%%%%%%%
% 1. To stop numbering the lines, deactivate \linenumbers (activated after loading the package lineno)
% 
% 2. To make the index, you need to compile first with lualatex and then use makeindex. Example:
%		lualatex main.tex
%		makeindex main.nlo -s nomencl.ist -o main.nls
%		lualatex main.tex
% 
% 3. For the bibliography, use the command biber (as I am using the package biblatex, much more modern than bibtex).
%		To use multiple bibliography, use the refsection environment (3.13.3 Multiple Bibliographies in biblatex doc Version 3.13).
%		Do not use the chapterbib package with biblatex, there is refsection environment for that!
%		Each article will need to use
%		\begin{refsection}
%			Intro, Methods, Results, Discussion
%			\printbibliography[heading=subbibnumbered]
%		\end{refsection}
% 
% 4. I defined two languages in the document, French and British english (which is the main language), use \begin{french} ... \end{french} to switch to French in a paragraph
% 
% 5. To avoid multi-defined labels, I recommend to apply this command line in your shell for each tex file of the folders ./article1/, ./article2/, ...:
% 		sed -i.tmp 's/\\label{/\\label{WHATEVER::/g' *.tex
% 			This will change all the labels by adding WHATEVER before. Example (adding article1 for each tex file in ./article1/ folder):
% 			\label{sec::intro} in ./article1/chap1.tex becomes \label{article1::sec::intro}
%		sed -i.tmp 's/\\ref{/\\ref{WHATEVER::/g' *.tex
%			This will change all the refs by adding WHATEVER before. Example (adding article1 for each tex file in ./article1/ folder):
% 			\ref{sec::intro} in ./article1/chap1.tex becomes \ref{article1::sec::intro}
% 		sed -i.tmp 's/\\eqref{/\\eqref{WHATEVER::/g' *.tex
%			This will change all the eqrefs by adding WHATEVER before. Example (adding article1 for each tex file in ./article1/ folder):
% 			\eqref{eq::R0_equation} in ./article1/chap1.tex becomes \eqref{article1::eq::R0_equation}
% 
%	For safety reasons, the original file will be kept in a temporary file '.tmp'. Example:
%		sed -i.tmp 'Your instructions here' introduction.tex will create two files:
% 			introduction.tex which should be the one to keep (if no mistakes in the sed instructions)
% 			introduction.tex.tmp which is the original file (that you can dump if sed behaved the way you expected)
% 	For an introduction to the power of sed and regular expressions: https://www.grymoire.com/Unix/Sed.html#uh-1
% 
% 6. I defined some colours in this document. Except RdeepBlue, none of them is used in this template and can be deleted.
% 	If you delete RdeepBlue, do not forget to adapt all the commands using this colour

%%%%%%%%%%%%%%%%%%%    PACKAGES    %%%%%%%%%%%%%%%%%%%
%%%% font
\usepackage{fontspec}
\usepackage{marvosym}

%%%% Language
\usepackage{polyglossia}
	\setdefaultlanguage[variant=british]{english}
	\setotherlanguages{french}

%%%% Date
\usepackage{datetime}
	\newdateformat{monthyeardate}{\monthname[\THEMONTH] \THEYEAR}

%%%% marges, space, titles...
\usepackage[top=3.75cm, bottom=2.5cm, left=3cm, right=2.5cm]{geometry}
% \usepackage{setspace}
\setlength{\parindent}{0ex}   % indentation au début de chaque paragraphe
\setlength{\parskip}{3ex plus 0.5ex minus 0.2ex} % espace vertical entre paragraphes

\usepackage{titlesec}
\newcommand\chapterstring{CHAPITRE}
\titleformat{\chapter}[display]{\vspace{-8em}\center\bfseries}{\chapterstring~\thechapter}{12pt}{}[\vspace{1.5ex}]
\titleformat{\section}{\vspace{0mm plus 2cm}\addpenalty{-1000}\bfseries\sffamily}{\thesection}{1em}{}
\titleformat{\subsection}{\addpenalty{-500}\bfseries\sffamily}{\thesubsection}{1em}{}
\titleformat{\subsubsection}{\penalty-500\vspace{0pt plus 2pt}{}\bfseries\sffamily}{\thesubsubsection}{}{}[\vspace{-14pt}]

\usepackage{titletoc}
	\titlecontents{chapter}[0pt]{}{\bfseries\chapterstring~\thecontentslabel~}{\bfseries}{\hspace{1em plus 1fill}\contentspage}

%%%% Appendix
\usepackage{appendix}
%% Start of subappendices environment
\AtBeginEnvironment{subappendices}{%
	\chapter*{Appendix}
	\addcontentsline{toc}{chapter}{Appendices}
	\counterwithin{figure}{section}
	\counterwithin{table}{section}
}

%% End of subappendices environment
\AtEndEnvironment{subappendices}{%
	\counterwithout{figure}{section}
	\counterwithout{table}{section}
}

%%%% Graphics
%% To compile with lualatex
\usepackage[luatex]{graphicx}

%% Graphics' path, but I prefer to use absolute path for safety
% \graphicspath{{./article1/}{./article2/}{article3/}}

%% Numbering figures in sections rather than continuously
\usepackage{chngcntr}
\counterwithin{figure}{section}

%% Put figures within their declaration section 
\usepackage[section]{placeins}

%% To create multi-figures
\usepackage{caption}
\usepackage{subcaption}

%% Colours
\usepackage[dvipsnames, svgnames]{xcolor}
	\definecolor{Rblack}{RGB}{0,0,0}
	\definecolor{Rgrey}{RGB}{42,51,68}
	\definecolor{RdeepBlue}{RGB}{17,34,170}
	\definecolor{RmidBlue}{RGB}{32,88,220}
	\definecolor{RlightBlue}{RGB}{90,130,234}
	\definecolor{RuglyGreen}{RGB}{253,236,190}
	\definecolor{RyellowPea}{RGB}{253,219,91}
	\definecolor{RdeepYellow}{RGB}{250,185,53}
	\definecolor{Rsalmon}{RGB}{253,152,89}
	\definecolor{Rred}{RGB}{182,52,58}

	\definecolor{lightAvailability_grey}{RGB}{20,20,20}

%% Tikz
\usepackage{tikz}

%%%% Tab, list...
%% Best package for tables
\usepackage{booktabs}

%% List
\usepackage{enumerate}
\usepackage{enumitem}% http://ctan.org/pkg/enumitem

%% Nomenclature (containing three sections: Acronyms, greek symbols, roman symbols) https://tex.stackexchange.com/questions/452107/nomenclature-having-two-or-more-descriptions-and-units-for-a-symbol
\usepackage[intoc]{nomencl}

%% Change caption entries to whatever you want (also done in the ToC)
\addto\captionsenglish{ % Replace "english" with the language you use, if you use babel you might have to replace english by british or USenglish
	\renewcommand{\contentsname}{TABLE DES MATIÈRES}
	\renewcommand{\listtablename}{LISTE DES TABLEAUX}
	\renewcommand{\listfigurename}{LISTE DES FIGURES}
}

%%%% Links
\usepackage{url}
\usepackage[luatex, colorlinks=true, linkcolor=black, urlcolor=RdeepBlue, citecolor=RdeepBlue]{hyperref}

%%%% Bibliography and table of contents
%% Bibliography output using biblatex
\usepackage{csquotes}
\usepackage[style=apa, natbib=true, sorting=ynt]{biblatex}
% \usepackage[uniquelist=false, backend=biber, citestyle=authoryear-comp, sorting=nyt, sortcites=true, giveninits=true, hyperref, natbib, url=false, doi=false, eprint=false, isbn=false, maxbibnames=25, maxcitenames=2, mincitenames=1, uniquename=init, refsection=chapter]{biblatex}
\addbibresource{phd.bib}

%% Table of content (= ToC)
\usepackage[nottoc]{tocbibind}

%%%% Mathematics
\usepackage{amsthm}
\usepackage{amsmath}
	\allowdisplaybreaks % allows breaks between pages in formula
\usepackage{amssymb}
\usepackage{bbold}
\usepackage{dsfont}
\usepackage{mathrsfs}
\usepackage{bm}
\usepackage{xfrac}
\usepackage{etoolbox} % For renumbering
\usepackage{thmbox} % For "theorem" definitions
\usepackage{gensymb}
\usepackage{siunitx}
	\sisetup{
		inter-unit-product=\ensuremath{{}\cdot{}},
		per-mode=symbol
	}

\DeclareMathOperator{\logit}{logit}

%%%% Line numbers --conflict with amsthm-- pdf http://texblog.org/2012/02/08/adding-line-numbers-to-documents/
\usepackage{lineno}
	\linenumbers

%%%%%%%%%%%%%%%%%%%%%   OTHERS   %%%%%%%%%%%%%%%%%%%%%
% http://tex.stackexchange.com/questions/13304/which-package-version-am-i-using
\listfiles % Add \listfiles to your preamble and then look at the .log file. This will tell you the current version of all the packages loaded

%%%%%%%%%%%%%%%%%%%%%   NOMENCLATURE CODING   %%%%%%%%%%%%%%%%%%%%%
\makenomenclature
\renewcommand{\nomname}{LISTE DES ABBRÉVIATIONS}

\ExplSyntaxOn
\RenewDocumentCommand\nomgroup{m}
{
	\item[\textbf{\textcolor{RdeepBlue}{\choosenomgroup{#1}}}]
}
\NewDocumentCommand{\choosenomgroup}{m}
{
	\str_case:nn { #1 }
	{
		{A}{Acronymes}
		{L}{Symboles~latins}
		{G}{Symboles~grecs}
	}
}
\ExplSyntaxOff
\renewcommand*{\nompreamble}{\markboth{\nomname}{\nomname}}

\newcommand{\nomunit}[1]{%
	\renewcommand{\nomentryend}{\hspace*{\fill}\si{#1}}%
}

%%%%%%%%%%%%%%%%%%   HYPHENATION    %%%%%%%%%%%%%%%%%%
%%%% If latex has troubles to split a word, you can teach it here
% \hyphenation{cal-cu-lus}

%%%%%%%%%%%%%%%%%%%%%%%%%%%%%%%%%%%%%%%%%%%%%%%%%%%%%%%%%%%%%%%%%%%%
%%%%%%%%%%%%%%%%%   NEW COMMANDS USER'S SPECIFIC   %%%%%%%%%%%%%%%%%
%%%%%%%%%%%%%%%%%%%%%%%%%%%%%%%%%%%%%%%%%%%%%%%%%%%%%%%%%%%%%%%%%%%%
%%%%%%%% Commands that will be apply to the whole Ph.D. For chapter specific commands, declare them in the chapters (and use renewcommand if a name is used twice)
%%%% Text commands
\newcommand {\ie}{\textit{i.e., }}
\newcommand {\eg}{\textit{e.g., }}
\newcommand {\cf}{\textit{cf} }
\providecommand{\keywords}[1]{\textbf{Mots-clefs:} #1}


%%%% Tikz
%% Libraries (specific tikz commands are in article*/main.tex files)
\usetikzlibrary{arrows, plotmarks, decorations.markings}
\tikzstyle{arrow} = [->,>=stealth,thick,rounded corners=4pt,line width=1pt]
\usetikzlibrary{shadows}
\usetikzlibrary{shadings}
\usetikzlibrary{positioning} % relative coodinate
\usetikzlibrary{tikzmark, calc} % calc, to calculate coordinate
\usetikzlibrary{decorations.pathmorphing} % to snake an arrow
\usetikzlibrary{shapes.arrows}
\usetikzlibrary{patterns}

%% Gradient shadings command
\makeatletter
\def\createshadingfromlist#1#2#3{%
	\pgfutil@tempcnta=0\relax
	\pgfutil@for\pgf@tmp:={#3}\do{\advance\pgfutil@tempcnta by1}%
	\ifnum\pgfutil@tempcnta=1\relax%
		\edef\pgf@spec{color(0)=(#3);color(100)=(#3)}%
	\else%
		\pgfmathparse{50/(\pgfutil@tempcnta-1)}\let\pgf@step=\pgfmathresult%
		\pgfutil@tempcntb=1\relax%
		\pgfutil@for\pgf@tmp:={#3}\do{%
			\ifnum\pgfutil@tempcntb=1\relax%
				\edef\pgf@spec{color(0)=(\pgf@tmp);color(25)=(\pgf@tmp)}%
			\else%
	        \ifnum\pgfutil@tempcntb<\pgfutil@tempcnta\relax%
				\pgfmathparse{25+\pgf@step/4+(\pgfutil@tempcntb-1)*\pgf@step}%
				\edef\pgf@spec{\pgf@spec;color(\pgfmathresult)=(\pgf@tmp)}%
	        \else%
	        	\edef\pgf@spec{\pgf@spec;color(75)=(\pgf@tmp);color(100)=(\pgf@tmp)}%
	        \fi%
	    \fi%
	    \advance\pgfutil@tempcntb by1\relax%
	    }%
	\fi%
	\csname pgfdeclare#2shading\endcsname{#1}{100}\pgf@spec%
}

%%%%%%%% Math commands
%%%% Maths
\newcommand {\s}{{s}^{*}}
\newcommand {\sst}{\tilde{s}^{*}} % s* stable d'où le st
\newcommand {\N}{\tilde{N}}
\newcommand {\A}{\mathscr{A}}
\newcommand {\K}{\mathcal{K}}
\renewcommand{\S}{\mathscr{S}}
\newcommand{\R}{\mathds{R}}
\newcommand{\Prob}{\mathds{P}}
\newcommand{\F}{\mathcal{F}}

%%%% Theorem style
\newtheoremstyle{theo}{\topsep}{\topsep}{\itshape}{}{\bfseries}{.}{\newline}{\thmname{#1} \thmnumber{#2} \thmnote{~: \textit{#3}}}
\theoremstyle{theo}
\newtheorem{rem}{Remark}[section]
\newtheorem{defi}{Definition}[section]
\newtheorem{assum}{Assumption}[section]
\newtheorem{nota}{Notation}[section]

%%%%%%%%%%%%%%%%%%%%%   BEGIN DOCUMENT   %%%%%%%%%%%%%%%%%%%%%
\begin{document}
%%%% Starts by a blank page
\newpage
\thispagestyle{empty}
\mbox{}

%%%% Title page
\begin{titlepage}
	\begin{center}
		\textsc{Your title small caps} \\
	\vspace{2cm}
	par \\
	\vspace{2cm}
	Amaël Le Squin
	\vspace{1cm}
	Université de Sherbrooke \\
	\vspace{1cm}
	Thèse présentée au Département de biologie en vue \\
	de l'obtention du grade de docteur ès sciences (Ph.D.)
	\vfill
	\textsc{faculté des sciences} \\
	\textsc{université de sherbrooke} \\
	\vspace{2cm}
	Sherbrooke, Québec, Canada, \begin{french} \monthyeardate\today \end{french}
	\end{center}	
\end{titlepage}

%%%% Jury page (and not Jimmy page...)
\thispagestyle{empty}
\begin{center}
\begin{french}
	Le \today \\
	\vspace{1cm}
	\textit{Le jury a accepté la thèse d'Amaël Le Squin dans sa version finale} \\
	\vspace{1cm}
	Membre du jury \\
	\vspace{1cm}
	Professeur Dominique Gravel \\
	Directeur de recherche \\
	Département de biologie \\
	Université de Sherbrooke \\
	\vspace{1cm}
	Professeure Isabelle Boulangeat \\
	Codirectrice de recherche \\
	IRSTEA \\
	\vspace{1cm}
	Professeur Prénom et nom \\
	Évaluatrice ou Évaluateur externe \\
	Département nom \\
	Nom de l'institution \\
	\vspace{1cm}
	Professeure Prénom et nom \\
	Évaluatrice ou Évaluateur interne \\
	Département nom \\
	Nom de l'institution \\
	\vspace{1cm}
	Professeure Prénom et nom \\
	Président-rapporteur \\
	Département de biologie \\
	Université de Sherbrooke
\end{french}
\end{center}

%%%% From abstract (included) to Introduction (excluded)
%% Start roman numbering
\pagenumbering{roman}
\input{abstract_fr}

\input{abstract_en}

\input{acknowledgements}

%% Table of contents
\tableofcontents

%% Nomencalture
\printnomenclature

%% List of tables
\listoftables

%% List of figures
\listoffigures

\newpage
%%%% Main body
%% Start arabic numbering
\pagenumbering{arabic}

%% Input chapters
\input{./introduction}
\input{./article1/main}
% \input{./article3/main}

\nomenclature[a]{TPS}{Thermal protection System}
\nomenclature[L]{$v$}{Fluid velocity\nomunit{\metre\per\second}}
\nomenclature[G]{$\tau$}{Longitude \nomunit{\radian}}
\nomenclature[g]{$\tau$}{Dimensional time \nomunit{\second}}
\nomenclature[g]{$\delta$}{Geocentric Latitude \nomunit{\radian}}
\nomenclature[g]{$\delta$}{Differential Operator}
\nomenclature[g]{$\delta$}{Radius of Trust Region \nomunit{\meter}}
\nomenclature[g]{$\delta^*$}{Geodetic Latitude.  \nomunit{\radian}}

%%%% Ends by a blank page, congrats!
\newpage
\thispagestyle{empty}
\mbox{}

\end{document}

\nomenclature[a]{TPS}{Thermal protection System}
\nomenclature[L]{$v$}{Fluid velocity\nomunit{\metre\per\second}}
\nomenclature[G]{$\tau$}{Longitude \nomunit{\radian}}
\nomenclature[g]{$\tau$}{Dimensional time \nomunit{\second}}
\nomenclature[g]{$\delta$}{Geocentric Latitude \nomunit{\radian}}
\nomenclature[g]{$\delta$}{Differential Operator}
\nomenclature[g]{$\delta$}{Radius of Trust Region \nomunit{\meter}}
\nomenclature[g]{$\delta^*$}{Geodetic Latitude.  \nomunit{\radian}}

%%%% Ends by a blank page, congrats!
\newpage
\thispagestyle{empty}
\mbox{}

\end{document}
% 
\documentclass[titlepage,oneside,letterpaper,openright,12pt]{report}

%%%%%%%%%%%%%%%%%%%    COMMENTS    %%%%%%%%%%%%%%%%%%%
% 1. To stop numbering the lines, deactivate \linenumbers (activated after loading the package lineno)
% 
% 2. To make the index, you need to compile first with lualatex and then use makeindex. Example:
%		lualatex main.tex
%		makeindex main.nlo -s nomencl.ist -o main.nls
%		lualatex main.tex
% 
% 3. For the bibliography, use the command biber (as I am using the package biblatex, much more modern than bibtex).
%		To use multiple bibliography, use the refsection environment (3.13.3 Multiple Bibliographies in biblatex doc Version 3.13).
%		Do not use the chapterbib package with biblatex, there is refsection environment for that!
%		Each article will need to use
%		\begin{refsection}
%			Intro, Methods, Results, Discussion
%			\printbibliography[heading=subbibnumbered]
%		\end{refsection}
% 
% 4. I defined two languages in the document, French and British english (which is the main language), use \begin{french} ... \end{french} to switch to French in a paragraph
% 
% 5. To avoid multi-defined labels, I recommend to apply this command line in your shell for each tex file of the folders ./article1/, ./article2/, ...:
% 		sed -i.tmp 's/\\label{/\\label{WHATEVER::/g' *.tex
% 			This will change all the labels by adding WHATEVER before. Example (adding article1 for each tex file in ./article1/ folder):
% 			\label{sec::intro} in ./article1/chap1.tex becomes \label{article1::sec::intro}
%		sed -i.tmp 's/\\ref{/\\ref{WHATEVER::/g' *.tex
%			This will change all the refs by adding WHATEVER before. Example (adding article1 for each tex file in ./article1/ folder):
% 			\ref{sec::intro} in ./article1/chap1.tex becomes \ref{article1::sec::intro}
% 		sed -i.tmp 's/\\eqref{/\\eqref{WHATEVER::/g' *.tex
%			This will change all the eqrefs by adding WHATEVER before. Example (adding article1 for each tex file in ./article1/ folder):
% 			\eqref{eq::R0_equation} in ./article1/chap1.tex becomes \eqref{article1::eq::R0_equation}
% 
%	For safety reasons, the original file will be kept in a temporary file '.tmp'. Example:
%		sed -i.tmp 'Your instructions here' introduction.tex will create two files:
% 			introduction.tex which should be the one to keep (if no mistakes in the sed instructions)
% 			introduction.tex.tmp which is the original file (that you can dump if sed behaved the way you expected)
% 	For an introduction to the power of sed and regular expressions: https://www.grymoire.com/Unix/Sed.html#uh-1
% 
% 6. I defined some colours in this document. Except RdeepBlue, none of them is used in this template and can be deleted.
% 	If you delete RdeepBlue, do not forget to adapt all the commands using this colour

%%%%%%%%%%%%%%%%%%%    PACKAGES    %%%%%%%%%%%%%%%%%%%
%%%% font
\usepackage{fontspec}
\usepackage{marvosym}

%%%% Language
\usepackage{polyglossia}
	\setdefaultlanguage[variant=british]{english}
	\setotherlanguages{french}

%%%% Date
\usepackage{datetime}
	\newdateformat{monthyeardate}{\monthname[\THEMONTH] \THEYEAR}

%%%% marges, space, titles...
\usepackage[top=3.75cm, bottom=2.5cm, left=3cm, right=2.5cm]{geometry}
% \usepackage{setspace}
\setlength{\parindent}{0ex}   % indentation au début de chaque paragraphe
\setlength{\parskip}{3ex plus 0.5ex minus 0.2ex} % espace vertical entre paragraphes

\usepackage{titlesec}
\newcommand\chapterstring{CHAPITRE}
\titleformat{\chapter}[display]{\vspace{-8em}\center\bfseries}{\chapterstring~\thechapter}{12pt}{}[\vspace{1.5ex}]
\titleformat{\section}{\vspace{0mm plus 2cm}\addpenalty{-1000}\bfseries\sffamily}{\thesection}{1em}{}
\titleformat{\subsection}{\addpenalty{-500}\bfseries\sffamily}{\thesubsection}{1em}{}
\titleformat{\subsubsection}{\penalty-500\vspace{0pt plus 2pt}{}\bfseries\sffamily}{\thesubsubsection}{}{}[\vspace{-14pt}]

\usepackage{titletoc}
	\titlecontents{chapter}[0pt]{}{\bfseries\chapterstring~\thecontentslabel~}{\bfseries}{\hspace{1em plus 1fill}\contentspage}

%%%% Appendix
\usepackage{appendix}
%% Start of subappendices environment
\AtBeginEnvironment{subappendices}{%
	\chapter*{Appendix}
	\addcontentsline{toc}{chapter}{Appendices}
	\counterwithin{figure}{section}
	\counterwithin{table}{section}
}

%% End of subappendices environment
\AtEndEnvironment{subappendices}{%
	\counterwithout{figure}{section}
	\counterwithout{table}{section}
}

%%%% Graphics
%% To compile with lualatex
\usepackage[luatex]{graphicx}

%% Graphics' path, but I prefer to use absolute path for safety
% \graphicspath{{./article1/}{./article2/}{article3/}}

%% Numbering figures in sections rather than continuously
\usepackage{chngcntr}
\counterwithin{figure}{section}

%% Put figures within their declaration section 
\usepackage[section]{placeins}

%% To create multi-figures
\usepackage{caption}
\usepackage{subcaption}

%% Colours
\usepackage[dvipsnames, svgnames]{xcolor}
	\definecolor{Rblack}{RGB}{0,0,0}
	\definecolor{Rgrey}{RGB}{42,51,68}
	\definecolor{RdeepBlue}{RGB}{17,34,170}
	\definecolor{RmidBlue}{RGB}{32,88,220}
	\definecolor{RlightBlue}{RGB}{90,130,234}
	\definecolor{RuglyGreen}{RGB}{253,236,190}
	\definecolor{RyellowPea}{RGB}{253,219,91}
	\definecolor{RdeepYellow}{RGB}{250,185,53}
	\definecolor{Rsalmon}{RGB}{253,152,89}
	\definecolor{Rred}{RGB}{182,52,58}

	\definecolor{lightAvailability_grey}{RGB}{20,20,20}

%% Tikz
\usepackage{tikz}

%%%% Tab, list...
%% Best package for tables
\usepackage{booktabs}

%% List
\usepackage{enumerate}
\usepackage{enumitem}% http://ctan.org/pkg/enumitem

%% Nomenclature (containing three sections: Acronyms, greek symbols, roman symbols) https://tex.stackexchange.com/questions/452107/nomenclature-having-two-or-more-descriptions-and-units-for-a-symbol
\usepackage[intoc]{nomencl}

%% Change caption entries to whatever you want (also done in the ToC)
\addto\captionsenglish{ % Replace "english" with the language you use, if you use babel you might have to replace english by british or USenglish
	\renewcommand{\contentsname}{TABLE DES MATIÈRES}
	\renewcommand{\listtablename}{LISTE DES TABLEAUX}
	\renewcommand{\listfigurename}{LISTE DES FIGURES}
}

%%%% Links
\usepackage{url}
\usepackage[luatex, colorlinks=true, linkcolor=black, urlcolor=RdeepBlue, citecolor=RdeepBlue]{hyperref}

%%%% Bibliography and table of contents
%% Bibliography output using biblatex
\usepackage{csquotes}
\usepackage[style=apa, natbib=true, sorting=ynt]{biblatex}
% \usepackage[uniquelist=false, backend=biber, citestyle=authoryear-comp, sorting=nyt, sortcites=true, giveninits=true, hyperref, natbib, url=false, doi=false, eprint=false, isbn=false, maxbibnames=25, maxcitenames=2, mincitenames=1, uniquename=init, refsection=chapter]{biblatex}
\addbibresource{phd.bib}

%% Table of content (= ToC)
\usepackage[nottoc]{tocbibind}

%%%% Mathematics
\usepackage{amsthm}
\usepackage{amsmath}
	\allowdisplaybreaks % allows breaks between pages in formula
\usepackage{amssymb}
\usepackage{bbold}
\usepackage{dsfont}
\usepackage{mathrsfs}
\usepackage{bm}
\usepackage{xfrac}
\usepackage{etoolbox} % For renumbering
\usepackage{thmbox} % For "theorem" definitions
\usepackage{gensymb}
\usepackage{siunitx}
	\sisetup{
		inter-unit-product=\ensuremath{{}\cdot{}},
		per-mode=symbol
	}

\DeclareMathOperator{\logit}{logit}

%%%% Line numbers --conflict with amsthm-- pdf http://texblog.org/2012/02/08/adding-line-numbers-to-documents/
\usepackage{lineno}
	\linenumbers

%%%%%%%%%%%%%%%%%%%%%   OTHERS   %%%%%%%%%%%%%%%%%%%%%
% http://tex.stackexchange.com/questions/13304/which-package-version-am-i-using
\listfiles % Add \listfiles to your preamble and then look at the .log file. This will tell you the current version of all the packages loaded

%%%%%%%%%%%%%%%%%%%%%   NOMENCLATURE CODING   %%%%%%%%%%%%%%%%%%%%%
\makenomenclature
\renewcommand{\nomname}{LISTE DES ABBRÉVIATIONS}

\ExplSyntaxOn
\RenewDocumentCommand\nomgroup{m}
{
	\item[\textbf{\textcolor{RdeepBlue}{\choosenomgroup{#1}}}]
}
\NewDocumentCommand{\choosenomgroup}{m}
{
	\str_case:nn { #1 }
	{
		{A}{Acronymes}
		{L}{Symboles~latins}
		{G}{Symboles~grecs}
	}
}
\ExplSyntaxOff
\renewcommand*{\nompreamble}{\markboth{\nomname}{\nomname}}

\newcommand{\nomunit}[1]{%
	\renewcommand{\nomentryend}{\hspace*{\fill}\si{#1}}%
}

%%%%%%%%%%%%%%%%%%   HYPHENATION    %%%%%%%%%%%%%%%%%%
%%%% If latex has troubles to split a word, you can teach it here
% \hyphenation{cal-cu-lus}

%%%%%%%%%%%%%%%%%%%%%%%%%%%%%%%%%%%%%%%%%%%%%%%%%%%%%%%%%%%%%%%%%%%%
%%%%%%%%%%%%%%%%%   NEW COMMANDS USER'S SPECIFIC   %%%%%%%%%%%%%%%%%
%%%%%%%%%%%%%%%%%%%%%%%%%%%%%%%%%%%%%%%%%%%%%%%%%%%%%%%%%%%%%%%%%%%%
%%%%%%%% Commands that will be apply to the whole Ph.D. For chapter specific commands, declare them in the chapters (and use renewcommand if a name is used twice)
%%%% Text commands
\newcommand {\ie}{\textit{i.e., }}
\newcommand {\eg}{\textit{e.g., }}
\newcommand {\cf}{\textit{cf} }
\providecommand{\keywords}[1]{\textbf{Mots-clefs:} #1}


%%%% Tikz
%% Libraries (specific tikz commands are in article*/main.tex files)
\usetikzlibrary{arrows, plotmarks, decorations.markings}
\tikzstyle{arrow} = [->,>=stealth,thick,rounded corners=4pt,line width=1pt]
\usetikzlibrary{shadows}
\usetikzlibrary{shadings}
\usetikzlibrary{positioning} % relative coodinate
\usetikzlibrary{tikzmark, calc} % calc, to calculate coordinate
\usetikzlibrary{decorations.pathmorphing} % to snake an arrow
\usetikzlibrary{shapes.arrows}
\usetikzlibrary{patterns}

%% Gradient shadings command
\makeatletter
\def\createshadingfromlist#1#2#3{%
	\pgfutil@tempcnta=0\relax
	\pgfutil@for\pgf@tmp:={#3}\do{\advance\pgfutil@tempcnta by1}%
	\ifnum\pgfutil@tempcnta=1\relax%
		\edef\pgf@spec{color(0)=(#3);color(100)=(#3)}%
	\else%
		\pgfmathparse{50/(\pgfutil@tempcnta-1)}\let\pgf@step=\pgfmathresult%
		\pgfutil@tempcntb=1\relax%
		\pgfutil@for\pgf@tmp:={#3}\do{%
			\ifnum\pgfutil@tempcntb=1\relax%
				\edef\pgf@spec{color(0)=(\pgf@tmp);color(25)=(\pgf@tmp)}%
			\else%
	        \ifnum\pgfutil@tempcntb<\pgfutil@tempcnta\relax%
				\pgfmathparse{25+\pgf@step/4+(\pgfutil@tempcntb-1)*\pgf@step}%
				\edef\pgf@spec{\pgf@spec;color(\pgfmathresult)=(\pgf@tmp)}%
	        \else%
	        	\edef\pgf@spec{\pgf@spec;color(75)=(\pgf@tmp);color(100)=(\pgf@tmp)}%
	        \fi%
	    \fi%
	    \advance\pgfutil@tempcntb by1\relax%
	    }%
	\fi%
	\csname pgfdeclare#2shading\endcsname{#1}{100}\pgf@spec%
}

%%%%%%%% Math commands
%%%% Maths
\newcommand {\s}{{s}^{*}}
\newcommand {\sst}{\tilde{s}^{*}} % s* stable d'où le st
\newcommand {\N}{\tilde{N}}
\newcommand {\A}{\mathscr{A}}
\newcommand {\K}{\mathcal{K}}
\renewcommand{\S}{\mathscr{S}}
\newcommand{\R}{\mathds{R}}
\newcommand{\Prob}{\mathds{P}}
\newcommand{\F}{\mathcal{F}}

%%%% Theorem style
\newtheoremstyle{theo}{\topsep}{\topsep}{\itshape}{}{\bfseries}{.}{\newline}{\thmname{#1} \thmnumber{#2} \thmnote{~: \textit{#3}}}
\theoremstyle{theo}
\newtheorem{rem}{Remark}[section]
\newtheorem{defi}{Definition}[section]
\newtheorem{assum}{Assumption}[section]
\newtheorem{nota}{Notation}[section]

%%%%%%%%%%%%%%%%%%%%%   BEGIN DOCUMENT   %%%%%%%%%%%%%%%%%%%%%
\begin{document}
%%%% Starts by a blank page
\newpage
\thispagestyle{empty}
\mbox{}

%%%% Title page
\begin{titlepage}
	\begin{center}
		\textsc{Your title small caps} \\
	\vspace{2cm}
	par \\
	\vspace{2cm}
	Amaël Le Squin
	\vspace{1cm}
	Université de Sherbrooke \\
	\vspace{1cm}
	Thèse présentée au Département de biologie en vue \\
	de l'obtention du grade de docteur ès sciences (Ph.D.)
	\vfill
	\textsc{faculté des sciences} \\
	\textsc{université de sherbrooke} \\
	\vspace{2cm}
	Sherbrooke, Québec, Canada, \begin{french} \monthyeardate\today \end{french}
	\end{center}	
\end{titlepage}

%%%% Jury page (and not Jimmy page...)
\thispagestyle{empty}
\begin{center}
\begin{french}
	Le \today \\
	\vspace{1cm}
	\textit{Le jury a accepté la thèse d'Amaël Le Squin dans sa version finale} \\
	\vspace{1cm}
	Membre du jury \\
	\vspace{1cm}
	Professeur Dominique Gravel \\
	Directeur de recherche \\
	Département de biologie \\
	Université de Sherbrooke \\
	\vspace{1cm}
	Professeure Isabelle Boulangeat \\
	Codirectrice de recherche \\
	IRSTEA \\
	\vspace{1cm}
	Professeur Prénom et nom \\
	Évaluatrice ou Évaluateur externe \\
	Département nom \\
	Nom de l'institution \\
	\vspace{1cm}
	Professeure Prénom et nom \\
	Évaluatrice ou Évaluateur interne \\
	Département nom \\
	Nom de l'institution \\
	\vspace{1cm}
	Professeure Prénom et nom \\
	Président-rapporteur \\
	Département de biologie \\
	Université de Sherbrooke
\end{french}
\end{center}

%%%% From abstract (included) to Introduction (excluded)
%% Start roman numbering
\pagenumbering{roman}
\input{abstract_fr}

\input{abstract_en}

\input{acknowledgements}

%% Table of contents
\tableofcontents

%% Nomencalture
\printnomenclature

%% List of tables
\listoftables

%% List of figures
\listoffigures

\newpage
%%%% Main body
%% Start arabic numbering
\pagenumbering{arabic}

%% Input chapters
\input{./introduction}

\documentclass[titlepage,oneside,letterpaper,openright,12pt]{report}

%%%%%%%%%%%%%%%%%%%    COMMENTS    %%%%%%%%%%%%%%%%%%%
% 1. To stop numbering the lines, deactivate \linenumbers (activated after loading the package lineno)
% 
% 2. To make the index, you need to compile first with lualatex and then use makeindex. Example:
%		lualatex main.tex
%		makeindex main.nlo -s nomencl.ist -o main.nls
%		lualatex main.tex
% 
% 3. For the bibliography, use the command biber (as I am using the package biblatex, much more modern than bibtex).
%		To use multiple bibliography, use the refsection environment (3.13.3 Multiple Bibliographies in biblatex doc Version 3.13).
%		Do not use the chapterbib package with biblatex, there is refsection environment for that!
%		Each article will need to use
%		\begin{refsection}
%			Intro, Methods, Results, Discussion
%			\printbibliography[heading=subbibnumbered]
%		\end{refsection}
% 
% 4. I defined two languages in the document, French and British english (which is the main language), use \begin{french} ... \end{french} to switch to French in a paragraph
% 
% 5. To avoid multi-defined labels, I recommend to apply this command line in your shell for each tex file of the folders ./article1/, ./article2/, ...:
% 		sed -i.tmp 's/\\label{/\\label{WHATEVER::/g' *.tex
% 			This will change all the labels by adding WHATEVER before. Example (adding article1 for each tex file in ./article1/ folder):
% 			\label{sec::intro} in ./article1/chap1.tex becomes \label{article1::sec::intro}
%		sed -i.tmp 's/\\ref{/\\ref{WHATEVER::/g' *.tex
%			This will change all the refs by adding WHATEVER before. Example (adding article1 for each tex file in ./article1/ folder):
% 			\ref{sec::intro} in ./article1/chap1.tex becomes \ref{article1::sec::intro}
% 		sed -i.tmp 's/\\eqref{/\\eqref{WHATEVER::/g' *.tex
%			This will change all the eqrefs by adding WHATEVER before. Example (adding article1 for each tex file in ./article1/ folder):
% 			\eqref{eq::R0_equation} in ./article1/chap1.tex becomes \eqref{article1::eq::R0_equation}
% 
%	For safety reasons, the original file will be kept in a temporary file '.tmp'. Example:
%		sed -i.tmp 'Your instructions here' introduction.tex will create two files:
% 			introduction.tex which should be the one to keep (if no mistakes in the sed instructions)
% 			introduction.tex.tmp which is the original file (that you can dump if sed behaved the way you expected)
% 	For an introduction to the power of sed and regular expressions: https://www.grymoire.com/Unix/Sed.html#uh-1
% 
% 6. I defined some colours in this document. Except RdeepBlue, none of them is used in this template and can be deleted.
% 	If you delete RdeepBlue, do not forget to adapt all the commands using this colour

%%%%%%%%%%%%%%%%%%%    PACKAGES    %%%%%%%%%%%%%%%%%%%
%%%% font
\usepackage{fontspec}
\usepackage{marvosym}

%%%% Language
\usepackage{polyglossia}
	\setdefaultlanguage[variant=british]{english}
	\setotherlanguages{french}

%%%% Date
\usepackage{datetime}
	\newdateformat{monthyeardate}{\monthname[\THEMONTH] \THEYEAR}

%%%% marges, space, titles...
\usepackage[top=3.75cm, bottom=2.5cm, left=3cm, right=2.5cm]{geometry}
% \usepackage{setspace}
\setlength{\parindent}{0ex}   % indentation au début de chaque paragraphe
\setlength{\parskip}{3ex plus 0.5ex minus 0.2ex} % espace vertical entre paragraphes

\usepackage{titlesec}
\newcommand\chapterstring{CHAPITRE}
\titleformat{\chapter}[display]{\vspace{-8em}\center\bfseries}{\chapterstring~\thechapter}{12pt}{}[\vspace{1.5ex}]
\titleformat{\section}{\vspace{0mm plus 2cm}\addpenalty{-1000}\bfseries\sffamily}{\thesection}{1em}{}
\titleformat{\subsection}{\addpenalty{-500}\bfseries\sffamily}{\thesubsection}{1em}{}
\titleformat{\subsubsection}{\penalty-500\vspace{0pt plus 2pt}{}\bfseries\sffamily}{\thesubsubsection}{}{}[\vspace{-14pt}]

\usepackage{titletoc}
	\titlecontents{chapter}[0pt]{}{\bfseries\chapterstring~\thecontentslabel~}{\bfseries}{\hspace{1em plus 1fill}\contentspage}

%%%% Appendix
\usepackage{appendix}
%% Start of subappendices environment
\AtBeginEnvironment{subappendices}{%
	\chapter*{Appendix}
	\addcontentsline{toc}{chapter}{Appendices}
	\counterwithin{figure}{section}
	\counterwithin{table}{section}
}

%% End of subappendices environment
\AtEndEnvironment{subappendices}{%
	\counterwithout{figure}{section}
	\counterwithout{table}{section}
}

%%%% Graphics
%% To compile with lualatex
\usepackage[luatex]{graphicx}

%% Graphics' path, but I prefer to use absolute path for safety
% \graphicspath{{./article1/}{./article2/}{article3/}}

%% Numbering figures in sections rather than continuously
\usepackage{chngcntr}
\counterwithin{figure}{section}

%% Put figures within their declaration section 
\usepackage[section]{placeins}

%% To create multi-figures
\usepackage{caption}
\usepackage{subcaption}

%% Colours
\usepackage[dvipsnames, svgnames]{xcolor}
	\definecolor{Rblack}{RGB}{0,0,0}
	\definecolor{Rgrey}{RGB}{42,51,68}
	\definecolor{RdeepBlue}{RGB}{17,34,170}
	\definecolor{RmidBlue}{RGB}{32,88,220}
	\definecolor{RlightBlue}{RGB}{90,130,234}
	\definecolor{RuglyGreen}{RGB}{253,236,190}
	\definecolor{RyellowPea}{RGB}{253,219,91}
	\definecolor{RdeepYellow}{RGB}{250,185,53}
	\definecolor{Rsalmon}{RGB}{253,152,89}
	\definecolor{Rred}{RGB}{182,52,58}

	\definecolor{lightAvailability_grey}{RGB}{20,20,20}

%% Tikz
\usepackage{tikz}

%%%% Tab, list...
%% Best package for tables
\usepackage{booktabs}

%% List
\usepackage{enumerate}
\usepackage{enumitem}% http://ctan.org/pkg/enumitem

%% Nomenclature (containing three sections: Acronyms, greek symbols, roman symbols) https://tex.stackexchange.com/questions/452107/nomenclature-having-two-or-more-descriptions-and-units-for-a-symbol
\usepackage[intoc]{nomencl}

%% Change caption entries to whatever you want (also done in the ToC)
\addto\captionsenglish{ % Replace "english" with the language you use, if you use babel you might have to replace english by british or USenglish
	\renewcommand{\contentsname}{TABLE DES MATIÈRES}
	\renewcommand{\listtablename}{LISTE DES TABLEAUX}
	\renewcommand{\listfigurename}{LISTE DES FIGURES}
}

%%%% Links
\usepackage{url}
\usepackage[luatex, colorlinks=true, linkcolor=black, urlcolor=RdeepBlue, citecolor=RdeepBlue]{hyperref}

%%%% Bibliography and table of contents
%% Bibliography output using biblatex
\usepackage{csquotes}
\usepackage[style=apa, natbib=true, sorting=ynt]{biblatex}
% \usepackage[uniquelist=false, backend=biber, citestyle=authoryear-comp, sorting=nyt, sortcites=true, giveninits=true, hyperref, natbib, url=false, doi=false, eprint=false, isbn=false, maxbibnames=25, maxcitenames=2, mincitenames=1, uniquename=init, refsection=chapter]{biblatex}
\addbibresource{phd.bib}

%% Table of content (= ToC)
\usepackage[nottoc]{tocbibind}

%%%% Mathematics
\usepackage{amsthm}
\usepackage{amsmath}
	\allowdisplaybreaks % allows breaks between pages in formula
\usepackage{amssymb}
\usepackage{bbold}
\usepackage{dsfont}
\usepackage{mathrsfs}
\usepackage{bm}
\usepackage{xfrac}
\usepackage{etoolbox} % For renumbering
\usepackage{thmbox} % For "theorem" definitions
\usepackage{gensymb}
\usepackage{siunitx}
	\sisetup{
		inter-unit-product=\ensuremath{{}\cdot{}},
		per-mode=symbol
	}

\DeclareMathOperator{\logit}{logit}

%%%% Line numbers --conflict with amsthm-- pdf http://texblog.org/2012/02/08/adding-line-numbers-to-documents/
\usepackage{lineno}
	\linenumbers

%%%%%%%%%%%%%%%%%%%%%   OTHERS   %%%%%%%%%%%%%%%%%%%%%
% http://tex.stackexchange.com/questions/13304/which-package-version-am-i-using
\listfiles % Add \listfiles to your preamble and then look at the .log file. This will tell you the current version of all the packages loaded

%%%%%%%%%%%%%%%%%%%%%   NOMENCLATURE CODING   %%%%%%%%%%%%%%%%%%%%%
\makenomenclature
\renewcommand{\nomname}{LISTE DES ABBRÉVIATIONS}

\ExplSyntaxOn
\RenewDocumentCommand\nomgroup{m}
{
	\item[\textbf{\textcolor{RdeepBlue}{\choosenomgroup{#1}}}]
}
\NewDocumentCommand{\choosenomgroup}{m}
{
	\str_case:nn { #1 }
	{
		{A}{Acronymes}
		{L}{Symboles~latins}
		{G}{Symboles~grecs}
	}
}
\ExplSyntaxOff
\renewcommand*{\nompreamble}{\markboth{\nomname}{\nomname}}

\newcommand{\nomunit}[1]{%
	\renewcommand{\nomentryend}{\hspace*{\fill}\si{#1}}%
}

%%%%%%%%%%%%%%%%%%   HYPHENATION    %%%%%%%%%%%%%%%%%%
%%%% If latex has troubles to split a word, you can teach it here
% \hyphenation{cal-cu-lus}

%%%%%%%%%%%%%%%%%%%%%%%%%%%%%%%%%%%%%%%%%%%%%%%%%%%%%%%%%%%%%%%%%%%%
%%%%%%%%%%%%%%%%%   NEW COMMANDS USER'S SPECIFIC   %%%%%%%%%%%%%%%%%
%%%%%%%%%%%%%%%%%%%%%%%%%%%%%%%%%%%%%%%%%%%%%%%%%%%%%%%%%%%%%%%%%%%%
%%%%%%%% Commands that will be apply to the whole Ph.D. For chapter specific commands, declare them in the chapters (and use renewcommand if a name is used twice)
%%%% Text commands
\newcommand {\ie}{\textit{i.e., }}
\newcommand {\eg}{\textit{e.g., }}
\newcommand {\cf}{\textit{cf} }
\providecommand{\keywords}[1]{\textbf{Mots-clefs:} #1}


%%%% Tikz
%% Libraries (specific tikz commands are in article*/main.tex files)
\usetikzlibrary{arrows, plotmarks, decorations.markings}
\tikzstyle{arrow} = [->,>=stealth,thick,rounded corners=4pt,line width=1pt]
\usetikzlibrary{shadows}
\usetikzlibrary{shadings}
\usetikzlibrary{positioning} % relative coodinate
\usetikzlibrary{tikzmark, calc} % calc, to calculate coordinate
\usetikzlibrary{decorations.pathmorphing} % to snake an arrow
\usetikzlibrary{shapes.arrows}
\usetikzlibrary{patterns}

%% Gradient shadings command
\makeatletter
\def\createshadingfromlist#1#2#3{%
	\pgfutil@tempcnta=0\relax
	\pgfutil@for\pgf@tmp:={#3}\do{\advance\pgfutil@tempcnta by1}%
	\ifnum\pgfutil@tempcnta=1\relax%
		\edef\pgf@spec{color(0)=(#3);color(100)=(#3)}%
	\else%
		\pgfmathparse{50/(\pgfutil@tempcnta-1)}\let\pgf@step=\pgfmathresult%
		\pgfutil@tempcntb=1\relax%
		\pgfutil@for\pgf@tmp:={#3}\do{%
			\ifnum\pgfutil@tempcntb=1\relax%
				\edef\pgf@spec{color(0)=(\pgf@tmp);color(25)=(\pgf@tmp)}%
			\else%
	        \ifnum\pgfutil@tempcntb<\pgfutil@tempcnta\relax%
				\pgfmathparse{25+\pgf@step/4+(\pgfutil@tempcntb-1)*\pgf@step}%
				\edef\pgf@spec{\pgf@spec;color(\pgfmathresult)=(\pgf@tmp)}%
	        \else%
	        	\edef\pgf@spec{\pgf@spec;color(75)=(\pgf@tmp);color(100)=(\pgf@tmp)}%
	        \fi%
	    \fi%
	    \advance\pgfutil@tempcntb by1\relax%
	    }%
	\fi%
	\csname pgfdeclare#2shading\endcsname{#1}{100}\pgf@spec%
}

%%%%%%%% Math commands
%%%% Maths
\newcommand {\s}{{s}^{*}}
\newcommand {\sst}{\tilde{s}^{*}} % s* stable d'où le st
\newcommand {\N}{\tilde{N}}
\newcommand {\A}{\mathscr{A}}
\newcommand {\K}{\mathcal{K}}
\renewcommand{\S}{\mathscr{S}}
\newcommand{\R}{\mathds{R}}
\newcommand{\Prob}{\mathds{P}}
\newcommand{\F}{\mathcal{F}}

%%%% Theorem style
\newtheoremstyle{theo}{\topsep}{\topsep}{\itshape}{}{\bfseries}{.}{\newline}{\thmname{#1} \thmnumber{#2} \thmnote{~: \textit{#3}}}
\theoremstyle{theo}
\newtheorem{rem}{Remark}[section]
\newtheorem{defi}{Definition}[section]
\newtheorem{assum}{Assumption}[section]
\newtheorem{nota}{Notation}[section]

%%%%%%%%%%%%%%%%%%%%%   BEGIN DOCUMENT   %%%%%%%%%%%%%%%%%%%%%
\begin{document}
%%%% Starts by a blank page
\newpage
\thispagestyle{empty}
\mbox{}

%%%% Title page
\begin{titlepage}
	\begin{center}
		\textsc{Your title small caps} \\
	\vspace{2cm}
	par \\
	\vspace{2cm}
	Amaël Le Squin
	\vspace{1cm}
	Université de Sherbrooke \\
	\vspace{1cm}
	Thèse présentée au Département de biologie en vue \\
	de l'obtention du grade de docteur ès sciences (Ph.D.)
	\vfill
	\textsc{faculté des sciences} \\
	\textsc{université de sherbrooke} \\
	\vspace{2cm}
	Sherbrooke, Québec, Canada, \begin{french} \monthyeardate\today \end{french}
	\end{center}	
\end{titlepage}

%%%% Jury page (and not Jimmy page...)
\thispagestyle{empty}
\begin{center}
\begin{french}
	Le \today \\
	\vspace{1cm}
	\textit{Le jury a accepté la thèse d'Amaël Le Squin dans sa version finale} \\
	\vspace{1cm}
	Membre du jury \\
	\vspace{1cm}
	Professeur Dominique Gravel \\
	Directeur de recherche \\
	Département de biologie \\
	Université de Sherbrooke \\
	\vspace{1cm}
	Professeure Isabelle Boulangeat \\
	Codirectrice de recherche \\
	IRSTEA \\
	\vspace{1cm}
	Professeur Prénom et nom \\
	Évaluatrice ou Évaluateur externe \\
	Département nom \\
	Nom de l'institution \\
	\vspace{1cm}
	Professeure Prénom et nom \\
	Évaluatrice ou Évaluateur interne \\
	Département nom \\
	Nom de l'institution \\
	\vspace{1cm}
	Professeure Prénom et nom \\
	Président-rapporteur \\
	Département de biologie \\
	Université de Sherbrooke
\end{french}
\end{center}

%%%% From abstract (included) to Introduction (excluded)
%% Start roman numbering
\pagenumbering{roman}
\input{abstract_fr}

\input{abstract_en}

\input{acknowledgements}

%% Table of contents
\tableofcontents

%% Nomencalture
\printnomenclature

%% List of tables
\listoftables

%% List of figures
\listoffigures

\newpage
%%%% Main body
%% Start arabic numbering
\pagenumbering{arabic}

%% Input chapters
\input{./introduction}
\input{./article1/main}
% \input{./article3/main}

\nomenclature[a]{TPS}{Thermal protection System}
\nomenclature[L]{$v$}{Fluid velocity\nomunit{\metre\per\second}}
\nomenclature[G]{$\tau$}{Longitude \nomunit{\radian}}
\nomenclature[g]{$\tau$}{Dimensional time \nomunit{\second}}
\nomenclature[g]{$\delta$}{Geocentric Latitude \nomunit{\radian}}
\nomenclature[g]{$\delta$}{Differential Operator}
\nomenclature[g]{$\delta$}{Radius of Trust Region \nomunit{\meter}}
\nomenclature[g]{$\delta^*$}{Geodetic Latitude.  \nomunit{\radian}}

%%%% Ends by a blank page, congrats!
\newpage
\thispagestyle{empty}
\mbox{}

\end{document}
% 
\documentclass[titlepage,oneside,letterpaper,openright,12pt]{report}

%%%%%%%%%%%%%%%%%%%    COMMENTS    %%%%%%%%%%%%%%%%%%%
% 1. To stop numbering the lines, deactivate \linenumbers (activated after loading the package lineno)
% 
% 2. To make the index, you need to compile first with lualatex and then use makeindex. Example:
%		lualatex main.tex
%		makeindex main.nlo -s nomencl.ist -o main.nls
%		lualatex main.tex
% 
% 3. For the bibliography, use the command biber (as I am using the package biblatex, much more modern than bibtex).
%		To use multiple bibliography, use the refsection environment (3.13.3 Multiple Bibliographies in biblatex doc Version 3.13).
%		Do not use the chapterbib package with biblatex, there is refsection environment for that!
%		Each article will need to use
%		\begin{refsection}
%			Intro, Methods, Results, Discussion
%			\printbibliography[heading=subbibnumbered]
%		\end{refsection}
% 
% 4. I defined two languages in the document, French and British english (which is the main language), use \begin{french} ... \end{french} to switch to French in a paragraph
% 
% 5. To avoid multi-defined labels, I recommend to apply this command line in your shell for each tex file of the folders ./article1/, ./article2/, ...:
% 		sed -i.tmp 's/\\label{/\\label{WHATEVER::/g' *.tex
% 			This will change all the labels by adding WHATEVER before. Example (adding article1 for each tex file in ./article1/ folder):
% 			\label{sec::intro} in ./article1/chap1.tex becomes \label{article1::sec::intro}
%		sed -i.tmp 's/\\ref{/\\ref{WHATEVER::/g' *.tex
%			This will change all the refs by adding WHATEVER before. Example (adding article1 for each tex file in ./article1/ folder):
% 			\ref{sec::intro} in ./article1/chap1.tex becomes \ref{article1::sec::intro}
% 		sed -i.tmp 's/\\eqref{/\\eqref{WHATEVER::/g' *.tex
%			This will change all the eqrefs by adding WHATEVER before. Example (adding article1 for each tex file in ./article1/ folder):
% 			\eqref{eq::R0_equation} in ./article1/chap1.tex becomes \eqref{article1::eq::R0_equation}
% 
%	For safety reasons, the original file will be kept in a temporary file '.tmp'. Example:
%		sed -i.tmp 'Your instructions here' introduction.tex will create two files:
% 			introduction.tex which should be the one to keep (if no mistakes in the sed instructions)
% 			introduction.tex.tmp which is the original file (that you can dump if sed behaved the way you expected)
% 	For an introduction to the power of sed and regular expressions: https://www.grymoire.com/Unix/Sed.html#uh-1
% 
% 6. I defined some colours in this document. Except RdeepBlue, none of them is used in this template and can be deleted.
% 	If you delete RdeepBlue, do not forget to adapt all the commands using this colour

%%%%%%%%%%%%%%%%%%%    PACKAGES    %%%%%%%%%%%%%%%%%%%
%%%% font
\usepackage{fontspec}
\usepackage{marvosym}

%%%% Language
\usepackage{polyglossia}
	\setdefaultlanguage[variant=british]{english}
	\setotherlanguages{french}

%%%% Date
\usepackage{datetime}
	\newdateformat{monthyeardate}{\monthname[\THEMONTH] \THEYEAR}

%%%% marges, space, titles...
\usepackage[top=3.75cm, bottom=2.5cm, left=3cm, right=2.5cm]{geometry}
% \usepackage{setspace}
\setlength{\parindent}{0ex}   % indentation au début de chaque paragraphe
\setlength{\parskip}{3ex plus 0.5ex minus 0.2ex} % espace vertical entre paragraphes

\usepackage{titlesec}
\newcommand\chapterstring{CHAPITRE}
\titleformat{\chapter}[display]{\vspace{-8em}\center\bfseries}{\chapterstring~\thechapter}{12pt}{}[\vspace{1.5ex}]
\titleformat{\section}{\vspace{0mm plus 2cm}\addpenalty{-1000}\bfseries\sffamily}{\thesection}{1em}{}
\titleformat{\subsection}{\addpenalty{-500}\bfseries\sffamily}{\thesubsection}{1em}{}
\titleformat{\subsubsection}{\penalty-500\vspace{0pt plus 2pt}{}\bfseries\sffamily}{\thesubsubsection}{}{}[\vspace{-14pt}]

\usepackage{titletoc}
	\titlecontents{chapter}[0pt]{}{\bfseries\chapterstring~\thecontentslabel~}{\bfseries}{\hspace{1em plus 1fill}\contentspage}

%%%% Appendix
\usepackage{appendix}
%% Start of subappendices environment
\AtBeginEnvironment{subappendices}{%
	\chapter*{Appendix}
	\addcontentsline{toc}{chapter}{Appendices}
	\counterwithin{figure}{section}
	\counterwithin{table}{section}
}

%% End of subappendices environment
\AtEndEnvironment{subappendices}{%
	\counterwithout{figure}{section}
	\counterwithout{table}{section}
}

%%%% Graphics
%% To compile with lualatex
\usepackage[luatex]{graphicx}

%% Graphics' path, but I prefer to use absolute path for safety
% \graphicspath{{./article1/}{./article2/}{article3/}}

%% Numbering figures in sections rather than continuously
\usepackage{chngcntr}
\counterwithin{figure}{section}

%% Put figures within their declaration section 
\usepackage[section]{placeins}

%% To create multi-figures
\usepackage{caption}
\usepackage{subcaption}

%% Colours
\usepackage[dvipsnames, svgnames]{xcolor}
	\definecolor{Rblack}{RGB}{0,0,0}
	\definecolor{Rgrey}{RGB}{42,51,68}
	\definecolor{RdeepBlue}{RGB}{17,34,170}
	\definecolor{RmidBlue}{RGB}{32,88,220}
	\definecolor{RlightBlue}{RGB}{90,130,234}
	\definecolor{RuglyGreen}{RGB}{253,236,190}
	\definecolor{RyellowPea}{RGB}{253,219,91}
	\definecolor{RdeepYellow}{RGB}{250,185,53}
	\definecolor{Rsalmon}{RGB}{253,152,89}
	\definecolor{Rred}{RGB}{182,52,58}

	\definecolor{lightAvailability_grey}{RGB}{20,20,20}

%% Tikz
\usepackage{tikz}

%%%% Tab, list...
%% Best package for tables
\usepackage{booktabs}

%% List
\usepackage{enumerate}
\usepackage{enumitem}% http://ctan.org/pkg/enumitem

%% Nomenclature (containing three sections: Acronyms, greek symbols, roman symbols) https://tex.stackexchange.com/questions/452107/nomenclature-having-two-or-more-descriptions-and-units-for-a-symbol
\usepackage[intoc]{nomencl}

%% Change caption entries to whatever you want (also done in the ToC)
\addto\captionsenglish{ % Replace "english" with the language you use, if you use babel you might have to replace english by british or USenglish
	\renewcommand{\contentsname}{TABLE DES MATIÈRES}
	\renewcommand{\listtablename}{LISTE DES TABLEAUX}
	\renewcommand{\listfigurename}{LISTE DES FIGURES}
}

%%%% Links
\usepackage{url}
\usepackage[luatex, colorlinks=true, linkcolor=black, urlcolor=RdeepBlue, citecolor=RdeepBlue]{hyperref}

%%%% Bibliography and table of contents
%% Bibliography output using biblatex
\usepackage{csquotes}
\usepackage[style=apa, natbib=true, sorting=ynt]{biblatex}
% \usepackage[uniquelist=false, backend=biber, citestyle=authoryear-comp, sorting=nyt, sortcites=true, giveninits=true, hyperref, natbib, url=false, doi=false, eprint=false, isbn=false, maxbibnames=25, maxcitenames=2, mincitenames=1, uniquename=init, refsection=chapter]{biblatex}
\addbibresource{phd.bib}

%% Table of content (= ToC)
\usepackage[nottoc]{tocbibind}

%%%% Mathematics
\usepackage{amsthm}
\usepackage{amsmath}
	\allowdisplaybreaks % allows breaks between pages in formula
\usepackage{amssymb}
\usepackage{bbold}
\usepackage{dsfont}
\usepackage{mathrsfs}
\usepackage{bm}
\usepackage{xfrac}
\usepackage{etoolbox} % For renumbering
\usepackage{thmbox} % For "theorem" definitions
\usepackage{gensymb}
\usepackage{siunitx}
	\sisetup{
		inter-unit-product=\ensuremath{{}\cdot{}},
		per-mode=symbol
	}

\DeclareMathOperator{\logit}{logit}

%%%% Line numbers --conflict with amsthm-- pdf http://texblog.org/2012/02/08/adding-line-numbers-to-documents/
\usepackage{lineno}
	\linenumbers

%%%%%%%%%%%%%%%%%%%%%   OTHERS   %%%%%%%%%%%%%%%%%%%%%
% http://tex.stackexchange.com/questions/13304/which-package-version-am-i-using
\listfiles % Add \listfiles to your preamble and then look at the .log file. This will tell you the current version of all the packages loaded

%%%%%%%%%%%%%%%%%%%%%   NOMENCLATURE CODING   %%%%%%%%%%%%%%%%%%%%%
\makenomenclature
\renewcommand{\nomname}{LISTE DES ABBRÉVIATIONS}

\ExplSyntaxOn
\RenewDocumentCommand\nomgroup{m}
{
	\item[\textbf{\textcolor{RdeepBlue}{\choosenomgroup{#1}}}]
}
\NewDocumentCommand{\choosenomgroup}{m}
{
	\str_case:nn { #1 }
	{
		{A}{Acronymes}
		{L}{Symboles~latins}
		{G}{Symboles~grecs}
	}
}
\ExplSyntaxOff
\renewcommand*{\nompreamble}{\markboth{\nomname}{\nomname}}

\newcommand{\nomunit}[1]{%
	\renewcommand{\nomentryend}{\hspace*{\fill}\si{#1}}%
}

%%%%%%%%%%%%%%%%%%   HYPHENATION    %%%%%%%%%%%%%%%%%%
%%%% If latex has troubles to split a word, you can teach it here
% \hyphenation{cal-cu-lus}

%%%%%%%%%%%%%%%%%%%%%%%%%%%%%%%%%%%%%%%%%%%%%%%%%%%%%%%%%%%%%%%%%%%%
%%%%%%%%%%%%%%%%%   NEW COMMANDS USER'S SPECIFIC   %%%%%%%%%%%%%%%%%
%%%%%%%%%%%%%%%%%%%%%%%%%%%%%%%%%%%%%%%%%%%%%%%%%%%%%%%%%%%%%%%%%%%%
%%%%%%%% Commands that will be apply to the whole Ph.D. For chapter specific commands, declare them in the chapters (and use renewcommand if a name is used twice)
%%%% Text commands
\newcommand {\ie}{\textit{i.e., }}
\newcommand {\eg}{\textit{e.g., }}
\newcommand {\cf}{\textit{cf} }
\providecommand{\keywords}[1]{\textbf{Mots-clefs:} #1}


%%%% Tikz
%% Libraries (specific tikz commands are in article*/main.tex files)
\usetikzlibrary{arrows, plotmarks, decorations.markings}
\tikzstyle{arrow} = [->,>=stealth,thick,rounded corners=4pt,line width=1pt]
\usetikzlibrary{shadows}
\usetikzlibrary{shadings}
\usetikzlibrary{positioning} % relative coodinate
\usetikzlibrary{tikzmark, calc} % calc, to calculate coordinate
\usetikzlibrary{decorations.pathmorphing} % to snake an arrow
\usetikzlibrary{shapes.arrows}
\usetikzlibrary{patterns}

%% Gradient shadings command
\makeatletter
\def\createshadingfromlist#1#2#3{%
	\pgfutil@tempcnta=0\relax
	\pgfutil@for\pgf@tmp:={#3}\do{\advance\pgfutil@tempcnta by1}%
	\ifnum\pgfutil@tempcnta=1\relax%
		\edef\pgf@spec{color(0)=(#3);color(100)=(#3)}%
	\else%
		\pgfmathparse{50/(\pgfutil@tempcnta-1)}\let\pgf@step=\pgfmathresult%
		\pgfutil@tempcntb=1\relax%
		\pgfutil@for\pgf@tmp:={#3}\do{%
			\ifnum\pgfutil@tempcntb=1\relax%
				\edef\pgf@spec{color(0)=(\pgf@tmp);color(25)=(\pgf@tmp)}%
			\else%
	        \ifnum\pgfutil@tempcntb<\pgfutil@tempcnta\relax%
				\pgfmathparse{25+\pgf@step/4+(\pgfutil@tempcntb-1)*\pgf@step}%
				\edef\pgf@spec{\pgf@spec;color(\pgfmathresult)=(\pgf@tmp)}%
	        \else%
	        	\edef\pgf@spec{\pgf@spec;color(75)=(\pgf@tmp);color(100)=(\pgf@tmp)}%
	        \fi%
	    \fi%
	    \advance\pgfutil@tempcntb by1\relax%
	    }%
	\fi%
	\csname pgfdeclare#2shading\endcsname{#1}{100}\pgf@spec%
}

%%%%%%%% Math commands
%%%% Maths
\newcommand {\s}{{s}^{*}}
\newcommand {\sst}{\tilde{s}^{*}} % s* stable d'où le st
\newcommand {\N}{\tilde{N}}
\newcommand {\A}{\mathscr{A}}
\newcommand {\K}{\mathcal{K}}
\renewcommand{\S}{\mathscr{S}}
\newcommand{\R}{\mathds{R}}
\newcommand{\Prob}{\mathds{P}}
\newcommand{\F}{\mathcal{F}}

%%%% Theorem style
\newtheoremstyle{theo}{\topsep}{\topsep}{\itshape}{}{\bfseries}{.}{\newline}{\thmname{#1} \thmnumber{#2} \thmnote{~: \textit{#3}}}
\theoremstyle{theo}
\newtheorem{rem}{Remark}[section]
\newtheorem{defi}{Definition}[section]
\newtheorem{assum}{Assumption}[section]
\newtheorem{nota}{Notation}[section]

%%%%%%%%%%%%%%%%%%%%%   BEGIN DOCUMENT   %%%%%%%%%%%%%%%%%%%%%
\begin{document}
%%%% Starts by a blank page
\newpage
\thispagestyle{empty}
\mbox{}

%%%% Title page
\begin{titlepage}
	\begin{center}
		\textsc{Your title small caps} \\
	\vspace{2cm}
	par \\
	\vspace{2cm}
	Amaël Le Squin
	\vspace{1cm}
	Université de Sherbrooke \\
	\vspace{1cm}
	Thèse présentée au Département de biologie en vue \\
	de l'obtention du grade de docteur ès sciences (Ph.D.)
	\vfill
	\textsc{faculté des sciences} \\
	\textsc{université de sherbrooke} \\
	\vspace{2cm}
	Sherbrooke, Québec, Canada, \begin{french} \monthyeardate\today \end{french}
	\end{center}	
\end{titlepage}

%%%% Jury page (and not Jimmy page...)
\thispagestyle{empty}
\begin{center}
\begin{french}
	Le \today \\
	\vspace{1cm}
	\textit{Le jury a accepté la thèse d'Amaël Le Squin dans sa version finale} \\
	\vspace{1cm}
	Membre du jury \\
	\vspace{1cm}
	Professeur Dominique Gravel \\
	Directeur de recherche \\
	Département de biologie \\
	Université de Sherbrooke \\
	\vspace{1cm}
	Professeure Isabelle Boulangeat \\
	Codirectrice de recherche \\
	IRSTEA \\
	\vspace{1cm}
	Professeur Prénom et nom \\
	Évaluatrice ou Évaluateur externe \\
	Département nom \\
	Nom de l'institution \\
	\vspace{1cm}
	Professeure Prénom et nom \\
	Évaluatrice ou Évaluateur interne \\
	Département nom \\
	Nom de l'institution \\
	\vspace{1cm}
	Professeure Prénom et nom \\
	Président-rapporteur \\
	Département de biologie \\
	Université de Sherbrooke
\end{french}
\end{center}

%%%% From abstract (included) to Introduction (excluded)
%% Start roman numbering
\pagenumbering{roman}
\input{abstract_fr}

\input{abstract_en}

\input{acknowledgements}

%% Table of contents
\tableofcontents

%% Nomencalture
\printnomenclature

%% List of tables
\listoftables

%% List of figures
\listoffigures

\newpage
%%%% Main body
%% Start arabic numbering
\pagenumbering{arabic}

%% Input chapters
\input{./introduction}
\input{./article1/main}
% \input{./article3/main}

\nomenclature[a]{TPS}{Thermal protection System}
\nomenclature[L]{$v$}{Fluid velocity\nomunit{\metre\per\second}}
\nomenclature[G]{$\tau$}{Longitude \nomunit{\radian}}
\nomenclature[g]{$\tau$}{Dimensional time \nomunit{\second}}
\nomenclature[g]{$\delta$}{Geocentric Latitude \nomunit{\radian}}
\nomenclature[g]{$\delta$}{Differential Operator}
\nomenclature[g]{$\delta$}{Radius of Trust Region \nomunit{\meter}}
\nomenclature[g]{$\delta^*$}{Geodetic Latitude.  \nomunit{\radian}}

%%%% Ends by a blank page, congrats!
\newpage
\thispagestyle{empty}
\mbox{}

\end{document}

\nomenclature[a]{TPS}{Thermal protection System}
\nomenclature[L]{$v$}{Fluid velocity\nomunit{\metre\per\second}}
\nomenclature[G]{$\tau$}{Longitude \nomunit{\radian}}
\nomenclature[g]{$\tau$}{Dimensional time \nomunit{\second}}
\nomenclature[g]{$\delta$}{Geocentric Latitude \nomunit{\radian}}
\nomenclature[g]{$\delta$}{Differential Operator}
\nomenclature[g]{$\delta$}{Radius of Trust Region \nomunit{\meter}}
\nomenclature[g]{$\delta^*$}{Geodetic Latitude.  \nomunit{\radian}}

%%%% Ends by a blank page, congrats!
\newpage
\thispagestyle{empty}
\mbox{}

\end{document}

\nomenclature[a]{TPS}{Thermal protection System}
\nomenclature[L]{$v$}{Fluid velocity\nomunit{\metre\per\second}}
\nomenclature[G]{$\tau$}{Longitude \nomunit{\radian}}
\nomenclature[g]{$\tau$}{Dimensional time \nomunit{\second}}
\nomenclature[g]{$\delta$}{Geocentric Latitude \nomunit{\radian}}
\nomenclature[g]{$\delta$}{Differential Operator}
\nomenclature[g]{$\delta$}{Radius of Trust Region \nomunit{\meter}}
\nomenclature[g]{$\delta^*$}{Geodetic Latitude.  \nomunit{\radian}}

%%%% Ends by a blank page, congrats!
\newpage
\thispagestyle{empty}
\mbox{}

\end{document}

\nomenclature[a]{TPS}{Thermal protection System}
\nomenclature[L]{$v$}{Fluid velocity\nomunit{\metre\per\second}}
\nomenclature[G]{$\tau$}{Longitude \nomunit{\radian}}
\nomenclature[g]{$\tau$}{Dimensional time \nomunit{\second}}
\nomenclature[g]{$\delta$}{Geocentric Latitude \nomunit{\radian}}
\nomenclature[g]{$\delta$}{Differential Operator}
\nomenclature[g]{$\delta$}{Radius of Trust Region \nomunit{\meter}}
\nomenclature[g]{$\delta^*$}{Geodetic Latitude.  \nomunit{\radian}}

%%%% Ends by a blank page, congrats!
\newpage
\thispagestyle{empty}
\mbox{}

\end{document}